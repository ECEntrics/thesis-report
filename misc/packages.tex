% Packages used

% Used for flexibility, so other types of documents can have their own preambles (e.g. presentations)
\usepackage[subpreambles=true]{standalone}

% Used for all the files inside thesis directory
\usepackage{subfiles}	%TODO: possibly unused (remove?)

% Paper size and margins
\usepackage{geometry}

% --- Languages & Fonts ---
\usepackage{polyglossia}
\setmainfont{Linux Libertine O}
\setsansfont{Linux Libertine O}

\newfontfamily\greekfonttt[
	Script=Default,
	Scale=MatchLowercase,
	Path = assets/fonts/,
	Extension = .ttf,
	UprightFont = *-Regular,
	BoldFont=*-Bold,
	ItalicFont=*-Italic,
	BoldItalicFont=*-BoldItalic
]{Hack}

\setdefaultlanguage{greek}
\setotherlanguage{english}
\usepackage{fontawesome5}

% --- Styling ---
\usepackage{hyperref}							% Extensive support for hypertext
\usepackage{authblk}							% Support for footnote style author/affiliation
\usepackage{enumitem}							% For item lists
\usepackage{custom-packages/greek-enumerate}	% Greek enumeration for ordered item lists
\usepackage{float}								% For \begin{figure}[H]
\usepackage[font={footnotesize, it}]{caption}	% For captions under figures
\usepackage{tabularx}							% Support for break lines inside table cells
\usepackage{multirow, booktabs}					% Useful table styling commands
\usepackage[flushleft]{threeparttable}			% Table footnotes
\usepackage[dvipsnames]{xcolor}					% Text colors
\usepackage{minted}								% Source code highlighting (make sure to add -shell-escape flag!)
\usepackage [autostyle]{csquotes}
\usepackage{tcolorbox}							% Colored boxes
\tcbuselibrary{minted}							% Make tcolorbox work with minted
\usepackage{graphicx}
\usepackage{appendix}							% Appendix helpers

% --- TikZ and UML diagrams
\usepackage{pgf-umlsd}

% --- Bibliography ---
\usepackage[sorting=none]{biblatex}

% --- Custom commands ---
% Heavily based on this template:
% https://www.overleaf.com/latex/templates/university-of-lincoln-computer-science-thesis-template-unofficial/vdbphtsnwdqv

% Define variables
\makeatletter

\newcommand{\universityLogo}[1]{\def\@logo{#1}}
\newcommand{\programme}[1]{\def\@programmename{#1}}
\newcommand{\school}[1]{\def\@schoolname{#1}}
\newcommand{\university}[1]{\def\@universityname{#1}}
\newcommand{\supervisor}[1]{\def\@supervisor{#1}}

\makeatother

% Change the author concat text
\renewcommand*{\Authand}{ και }

% Make custom title
\makeatletter
\renewcommand\maketitle{
{\raggedright
\begin{center}
\thispagestyle{empty}

% Make the logo
\makeatletter
\centering\includegraphics[height=5cm]{\@logo}

% Make the title
\vspace{3cm}
{\huge \@title}

% The authors, school and university name
\vspace{3cm}
{\Large \@author}
\vspace{3cm}

\begin{tabular}{rl}
    \textit{Επιβλέπων} & \@supervisor
\end{tabular}

% Print out the supervisor
\vspace{1cm}
{\@programmename \\
\vspace{0.25cm} \@schoolname \\
\vspace{0.25cm} \@universityname}

% Then the date
\vspace{1.5cm}
{\footnotesize \@date}

\end{center}}
\newpage
}
\makeatother

 % Global settings for ordered lists (using enumitem and greek-enumerate packages)
\setenumerate[1]{label={\greek*)}} % First level
\setenumerate[2]{label={\roman*.}} % Second level
\newtcbinputlisting{\simplelisting}[2][]{
	listing file={assets/code/#2},
	title={},
	listing only,
	boxrule=1pt,
	minted language=javascript,
	minted style=default,
	minted options={breaklines, breaksymbol={}},
	#1
}
\newcommand{\logo}[2]{
	\begin{figure}[H]
		\centering
		\includegraphics[width=.12\textwidth]{assets/figures/#1}
		\caption{#2}
	\end{figure}
}
% Centered figure inside an item list
\newenvironment{enumitemcenteredfigure}
{
	\begin{minipage}{\linewidth}
		\centering
		\begin{figure}[H]
			\centering
		}
		{
		\end{figure}
	\end{minipage}
	\medskip
}

\newcommand{\sysReqItem}[6] {
    \item \ #1
    \begin{itemize}[label={}, leftmargin=0pt]
        \item \textbf{Περιγραφή}: #2
        \item \textbf{User Priority (#3/5)}: #4
        \item \textbf{Technical Priority (#5/5)}: #6
    \end{itemize}
    \medskip
}
\newcommand{\useCaseTable}[8] {{
    \begin{table}[H]
        \begin{center}
            \begin{tabularx}{\textwidth}{l X}
                \toprule
                \multicolumn{2}{c}{\textbf{#1}} \\ [0.5ex]
                \midrule
                Σύντομη περιγραφή         & #2 \\ [0.5ex]
                Αναφορά ΛΑ                & #3 \\ [0.5ex]
                Αναφορά ΜΛΑ               & #4 \\ [0.5ex]
                Πυροδότηση δραστηριότητας & #5 \\ [0.5ex]
                Προϋπόθεση                & #6 \\ [0.5ex]
                \bottomrule
            \end{tabularx}
        \end{center}
        \caption{#7}
        \label{table:#8}
    \end{table}
}}

\newcommand{\useCaseBaseFlowTable}[4] {{
    \begin{table}[H]
        \begin{center}
            \begin{tabularx}{\textwidth}{p{2.25cm} X X}
                \toprule
                \multicolumn{3}{c}{\textbf{Βασική ροή}} \\ [0.5ex]
                \midrule
                \textbf{Γραμμή}               & \textbf{Ενέργεια χρήστη συστήματος} & \textbf{Απάντηση Συστήματος} \\ [0.5ex]
                \midrule
                #1
                \midrule
                \textbf{Μετέπειτα κατάσταση:} & \multicolumn{2}{p{\dimexpr\textwidth-2\tabcolsep-0.8pt-2.25cm}}{#2} \\ [0.5ex]
                \bottomrule
            \end{tabularx}
        \end{center}
        \caption{#3}
        \label{table:#4}
    \end{table}
}}

\newcommand{\useCaseAlternateFlowTable}[7] {{
    \begin{table}[H]
        \begin{center}
            \begin{tabularx}{\textwidth}{l X X}
                \toprule
                \multicolumn{3}{l}{\textbf{Εναλλακτική ροή {#1}:} {#2}} \\ [0.5ex]
                \midrule
                \multicolumn{3}{p{\dimexpr\textwidth-2\tabcolsep-0.8pt}}{{#3}} \\ [0.5ex]
                \midrule
                \textbf{Γραμμή}               & \textbf{Ενέργεια χρήστη συστήματος} & \textbf{Απάντηση Συστήματος} \\ [0.5ex]
                \midrule
                #4 \\ [0.5ex]
                \midrule
                \multicolumn{3}{p{\dimexpr\textwidth-2\tabcolsep-0.8pt}}{{#5}} \\ [0.5ex]
                \bottomrule
            \end{tabularx}
        \end{center}
        \caption{#6}
        \label{table:#7}
    \end{table}
}}

% Command taken from this stackexchange answer
% https://tex.stackexchange.com/a/96913
\newcommand{\spheading}[3]{\rotatebox{#1}{\parbox{#2}{\raggedright #3}}}

% Usage:
% \spheading[<degrees_rotation>][<width>]{<stuff>}
\renewcommand{\appendixtocname}{Παραρτήματα}
\renewcommand{\appendixpagename}{Παραρτήματα}


% --- Custom styles ---
\renewcommand{\arraystretch}{1.2}	% Streches the table row height so text is not crammed between the lines
\MakeOuterQuote{"}	% For csquotes package

% Hyphenations
\begin{hyphenrules}{english}
	% Custom hyphenations go here
	% \hyphenation{ar-ti-fa-cts}
\end{hyphenrules}

