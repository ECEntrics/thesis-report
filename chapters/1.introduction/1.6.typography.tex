\section{Τυπογραφικές παραδοχές} \label{section:1-6-typography}

Το παρόν έγγραφο αποτυπώνεται με τη γραμματοσειρά Linux Libertine O\footnote{\url{https://libertine-fonts.org/}}, ενώ για τα κομμάτια κώδικα χρησιμοποιείται η Hack\footnote{\url{https://sourcefoundry.org/hack/}}. Το μέγεθος του κυρίως κειμένου είναι 12pt και το διάστιχό του είναι επαυξημένο του προκαθορισμένου κατά το ήμισυ για άνεση κατά την ανάγνωση.

Καταβάλλεται η μέγιστη δυνατή προσπάθεια για τη χρήση ελληνικών όρων, όπου αυτό είναι εφικτό, με τους αντίστοιχους αγγλικούς να τους συνοδεύουν σε ακόλουθες παρενθέσεις. Τα εισαγωγικά που χρησιμοποιούνται είναι τα διπλά γωνιώδη (« »), τόσο για ελληνικούς, όσο και για ξενόγλωσσους χαρακτηρισμούς.

Επίσης, αριθμούνται επί της συνολικής έκτασης της εργασίας οι λεζάντες των σχημάτων και των πινάκων, οι υποσημειώσεις και οι βιβλιογραφικές αναφορές, με τις τελευταίες να παρατίθενται στο τέλος του εγγράφου.

Τέλος, επισημαίνεται ότι η συγγραφή της αναφοράς πραγματοποιήθηκε στο ηλεκτρονικό τυπογραφικό σύστημα \LaTeX. Ο πηγαίος της κώδικας μπορεί να βρεθεί στο αντίστοιχο αποθετήριο κώδικα της διπλωματικής εργασίας\footnote{\url{https://gitlab.com/ecentrics/thesis-report}.}.
