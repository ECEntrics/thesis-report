\section{Ορισμός του προβλήματος}

Στις μέρες μας τα περισσότερα δεδομένα των χρηστών βρίσκονται υπό τον έλεγχο συγκεντρωτικών συστημάτων. Σε τέτοια συστήματα οι χρήστες δεν είναι κύριοι των δεδομένων τους, δεν έχουν εγγύηση για την αυθεντικότητα αυτών που βλέπουν και υπόκεινται σε λογοκρισία, ενώ τα συστήματα αυτά δεν είναι ασφαλή και μπορεί να σταματήσουν να λειτουργούν προσωρινά ή μόνιμα για τεχνικούς/οικονομικούς/νομικούς λόγους.

Οι περισσότερες διαδεδομένες, συγκεντρωτικές μορφές πλατφόρμας επικοινωνίας (mailing list, forum, κοινωνικά δίκτυα και άλλες) χρειάζονται, τυπικά, τουλάχιστον τις εξής τεχνολογίες:

\begin{itemize}
	\item μία πηγή επεξεργαστικής ισχύος (processing)
	\item μία βάση δεδομένων
	\item ένα πρωτόκολλο επικοινωνίας
\end{itemize}

Η επεξεργαστική ισχύς είναι αναγκαία για την περάτωση των λειτουργιών οι οποίες υλοποιούν τις υπηρεσίες της πλατφόρμας. Τις περισσότερες φορές η πηγή αυτή είναι ένας server ή μία cloud υπηρεσία.

Η βάση δεδομένων είναι απαραίτητη για την αποθήκευση της πληροφορίας. Σε μικρότερες εφαρμογές η βάση βρίσκεται στο ίδιο σύστημα που γίνεται και το processing, ενώ σε μεγαλύτερες ενδέχεται να υπάρχει για λόγους ασφάλειας ένα ξεχωριστό σύστημα αφιερωμένο στη βάση δεδομένων.

Το πρωτόκολλο επικοινωνίας αναλαμβάνει τη μετάδοση και ανάκτηση της πληροφορίας. Το πρωτόκολλό που χρησιμοποιείται σήμερα στη συντριπτική πλοιοψηφία των εφαρμογών είναι το HTTP.

Κάθε ένα από τα παραπάνω μέρη, εισάγει την ανάγκη ύπαρξης κεντρικών αρχών που τα διαχειρίζονται και τα συντηρούν. Η αρχή αυτή είναι συνήθως ο πάροχος της υπηρεσίας που διαχειρίζεται το processing και τη βάση δεδομένων, έχοντας έτσι πρόσβαση σε όλα τα δεδομένα που υπάρχουν στο σύστημα.
