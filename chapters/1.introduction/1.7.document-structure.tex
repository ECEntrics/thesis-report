\section{Οργάνωση κεφαλαίων}\label{section:1-7-document-structure}

Η παρούσα διπλωματική εργασία οργανώνεται σε κεφάλαια, ενότητες και υποενότητες, όπως αυτά διατυπώνονται στα \hyperref[toc]{\textbf{Περιεχόμενα}}.

Πιο συγκεκριμένα:

\begin{itemize}
	\item Το εισαγωγικό \hyperref[chapter:1-introduction]{\textbf{Κεφάλαιο 1}} έχει ως κεντρικά θέματα την ανάλυση του όρου "αποκέντρωση" (\ref{section:1-2-decentralization}), την περιγραφή του προβλήματος (\ref{section:1-3-problem-definition}), καθώς και την παρουσίαση του στόχου (\ref{section:1-4-thesis-goal}) και της μεθοδολογίας (\ref{section:1-5-methodology}) της εργασίας .
	\item Το \hyperref[chapter:2-theoretical-background]{\textbf{Κεφάλαιο 2}} σχετίζεται με το θεωρητικό υπόβαθρο, το οποίο περιλαμβάνει όλες τις έννοιες που είναι απαραίτητες για την κατανόηση των διαδικασιών της σχεδίασης και της υλοποίησης της εφαρμογής. Απαρτίζεται από επτά ενότητες, στις οποίες αναλύονται σε επαρκή λεπτομέρεια οι συναρτήσεις κατακερματισμού (\ref{section:2-1-hash-functions}), η ασύμμετρη κρυπτογραφία (\ref{section:2-2-asymmetric-cryptography}), τα δένδρα Merkle (\ref{section:2-3-merkle-trees}), τα δίκτυα ομότιμων κόμβων (\ref{section:2-4-p2p-networks}), το blockchain (\ref{section:2-5-blockchain}), το Ethereum (\ref{section:2-6-ethereum}) και το IPFS (\ref{section:2-7-ipfs}).
	\item Στο \hyperref[chapter:3-application-design]{\textbf{Κεφάλαιο 3}} περιγράφεται η διαδικασία της σχεδίασης της εφαρμογής. Συγκεκριμένα, ξεκινάει με τη σύλληψη της ιδέας (\ref{section:3-1-idea-conception}),την επιλογή της τεχνολογικής στοίβας (\ref{section:3-2-technology-stack}) και την παρουσίαση της μεθοδολογίας της σχεδίασης (\ref{section:3-3-design-methodology}). Συνεχίζει με τον ορισμό των κατηγοριών των χρηστών (\ref{section:3-4-user-categories}), των απαιτήσεων λογισμικού (\ref{section:3-5-software-requirements}) και των σεναρίων χρήσης (\ref{section:3-6-use-cases}). Κλείνει με τις ενότητες της αρχιτεκτονική σχεδίασης (\ref{section:3-7-architecture-design}) και της
	προδιαγραφής της μεθόδου υλοποίησης και του χρονοπρογραμματισμού (\ref{section:3-8-implementation-methodology-specification}).
	\item Στο \hyperref[chapter:4-application-implementation]{\textbf{Κεφάλαιο 4}} αναλύεται η διαδικασία υλοποίησης της πιλοτικής εφαρμογής Concordia. Αρχικά γίνεται μία περιγραφή της μεθοδολογίας που ακολουθήθηκε (\ref{subsection:4-1-implementation-methodology}), καθώς και των τεχνολογιών που χρησιμοποιήθηκαν (\ref{subsection:4-2-implementation-technology-stack}). Στην ενότητα \ref{section:4-3-implementation-architecture} παρουσιάζεται η αρχιτεκτονική του περιβάλλοντος ανάπτυξης και της εφαρμογής, μέσω της ανάλυσης των συνθετικών τους στοιχείων και του τρόπου που εκείνα διασυνδέονται. Έπειτα, επισημαίνονται τα προβλήματα που προέκυψαν κατά την ανάπτυξη (\ref{section:4-4-problems-faced}) και, τελικά, παρουσιάζονται τα υλοποιημένα χαρακτηριστικά της εφαρμογής, καθώς και οι διαφορές ανάμεσα στον σχεδιασμό και την υλοποίηση (\ref{section:4-5-implemented-parts}).
	\item Στο \hyperref[chapter:5-conclusions-open-areas]{\textbf{Κεφάλαιο 5}} διατυπώνονται τα συμπεράσματα της εργασίας (\ref{section:5-1-conclusions}) και προτείνονται ορισμένες πιθανές μελλοντικές επεκτάσεις (\ref{section:5-2-open-areas}).
	\item Το \hyperref[{appendix-a}]{\textbf{Παράρτημα Αʹ}} περιέχει στιγμιότυπα οθόνης της υλοποιημένης εφαρμογής.
	\item Το \hyperref[{appendix-b}]{\textbf{Παράρτημα Βʹ}} περιλαμβάνει πίνακες με στατιστικά στοιχεία του αναπτυχθέντα κώδικα.
	\item Τέλος, παρατίθενται οι \hyperref[{bibliography}]{\textbf{βιβλιογραφικές αναφορές}} που χρησιμοποιήθηκαν στο κείμενο.
\end{itemize}
