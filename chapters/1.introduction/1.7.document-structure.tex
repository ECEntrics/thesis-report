\section{Οργάνωση κεφαλαίων}\label{section:1-7-document-structure}

Η παρούσα διπλωματική εργασία οργανώνεται σε κεφάλαια, ενότητες και υποενότητες, όπως αυτά διατυπώνονται στα \hyperref[toc]{Περιεχόμενα}. Πιο συγκεκριμένα:

\begin{itemize}
	\item Στο εισαγωγικό \hyperref[chapter:1-introduction]{\textbf{Κεφάλαιο 1}} γίνεται μία σύντομη ανάλυση του όρου "αποκέντρωση", μία περιγραφή του προβλήματος και μία παρουσίαση του στόχου της εργασίας.
	\item Το \hyperref[chapter:2-theoretical-background]{\textbf{Κεφάλαιο 2}} σχετίζεται με το θεωρητικό υπόβαθρο, το οποίο περιλαμβάνει όλες τις έννοιες που είναι απαραίτητες για την κατανόηση των διαδικασιών της σχεδίασης και της υλοποίησης της εφαρμογής.
	\item Στο \hyperref[chapter:3-application-design]{\textbf{Κεφάλαιο 3}} αναλύεται η διαδικασία της σχεδίασης της εφαρμογής.
	\item Στο \hyperref[chapter:4-application-implementation]{\textbf{Κεφάλαιο 4}} περιγράφεται η διαδικασία υλοποίησης της πιλοτικής εφαρμογής Concordia.
	\item Στο \hyperref[chapter:5-conclusions-open-areas]{\textbf{Κεφάλαιο 5}} παρουσιάζονται τα συμπεράσματα της εργασίας (\ref{section:5-1-conclusions}), καθώς και διάφορες πιθανές μελλοντικές επεκτάσεις (\ref{section:5-2-open-areas}).
	\item Τέλος, το \hyperref[{screenshots-appendix}]{\textbf{Παράρτημα Αʹ}} περιέχει στιγμιότυπα οθόνης της υλοποιημένης εφαρμογής.
\end{itemize}
