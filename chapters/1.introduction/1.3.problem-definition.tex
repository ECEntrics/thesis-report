\section{Ορισμός του προβλήματος}\label{section:1-3-problem-definition}

Οι περισσότερες διαδεδομένες πλατφόρμες επικοινωνίας (κοινωνικά δίκτυα, mailing lists, forums κ.ά.) είναι ως επί το πλείστον συγκεντρωτικής μορφής, πράγμα το οποίο καθιστά αναγκαία την ύπαρξη κεντρικών αρχών που να τις διαχειρίζονται και να τις συντηρούν. 

Παρά τα θετικά της χαρακτηριστικά, η κεντροποιημένη λογική ενός τέτοιου συστήματος αφενός συνοδεύεται από ποικίλα μειονεκτήματα τεχνικής φύσεως (αρχιτεκτονικός συγκεντρωτισμός), αφετέρου εγείρει σοβαρούς προβληματισμούς σχετικά με τη διαχείριση των προσωπικών δεδομένων των χρηστών από τις κεντρικές αρχές (πολιτικός συγκεντρωτισμός). Τα βασικότερα από τα παραπάνω θα μπορούσαν να συνοψιστούν ως εξής: 

\begin{itemize}
	\item Έλλειψη \textbf{ασφάλειας}: Τα προσωπικά δεδομένα των χρηστών μπορεί να υποκλαπούν εξαιτίας κάποιας κυβερνοεπίθεσης.
	\item Έλλειψη \textbf{διαθεσιμότητας}: Το σύστημα μπορεί να σταματήσει να λειτουργεί προσωρινά ή μόνιμα για τεχνικούς, οικονομικούς ή νομικούς λόγους.
	\item Έλλειψη \textbf{εμπιστοσύνης}: Οι κεντρικές αρχές έχουν τη δυνατότητα να παρακολουθούν τους χρήστες, να διαβάζουν, ή ακόμα και να διαρρέουν τα προσωπικά τους δεδομένα εν αγνοία των τελευταίων. Οι δε χρήστες δε διαθέτουν κανέναν τρόπο με τον οποίον να μπορούν να τις εμπιστευθούν με βεβαιότητα.
	\item Έλλειψη εγγύησης της \textbf{αυθεντικότητας} των δεδομένων: Οι κεντρικές αρχές έχουν τη δυνατότητα να τροποποιούν τα δεδομένα κατά βούληση κάτι που έχει ως αποτέλεσμα να μην υπάρχει εγγύηση ως προς την αυθεντικότητα όσων βλέπουν οι χρήστες.
	\item Έλλειψη εγγύησης της \textbf{ελευθερίας του λόγου}: Οι κεντρικές αρχές έχουν τη δυνατότητα να λογοκρίνουν τα δεδομένα, είτε βάσει των συμφερόντων τους, είτε βάσει υποχρεώσεών τους προς τρίτους.
\end{itemize}

Επιπλέον, όπως γίνεται φανερό, οι αδυναμίες του συστήματος ως προς τον πολιτικό άξονα το καθιστούν ακατάλληλο να παρέχει στους χρήστες αυθεντικές και επικυρώσιμες δημοκρατικές διαδικασίες. Τέτοιου είδους διαδικασίες θα μπορούσε να ήταν από απλές ψηφοφορίες, μέχρι σύνθετες διαδικασίες αυτοδιαχείρισης της πλατφόρμας.