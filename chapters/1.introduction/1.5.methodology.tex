\section{Μεθοδολογία της διπλωματικής}\label{section:1-5-methodology}

Έναυσμα της παρούσας εργασίας αποτέλεσε η παρατηρήση της αρχιτεκτονικής δομής των σύγχρονων διαδικτυακών εφαρμογών και η ανάγκη διερεύνησης των επιπτώσεών της στον τελικό χρήστη. 

Αρχικά, ορίστηκε με σαφήνεια το πρόβλημα (\hyperref[section:1-3-problem-definition]{ενότητα 1.3}) και ο στόχος της διπλωματικής (\hyperref[section:1-4-thesis-goal]{ενότητα 1.4}), λαμβάνοντας την απόφαση να περιοριστεί στον τομέα των μέσων κοινωνικής δικτύωσης και της ψηφιακής δημοκρατίας.

Στη συνέχεια, πραγματοποιήθηκε έρευνα του χώρου των αποκεντρωμένων τεχνολογιών και ξεκίνησε η διαδικασία της σχεδίασης της εφαρμογής, μέσω της επιλογής του μοντέλου της τεχνολογικής στοίβας και του κατάλληλου λογισμικού για κάθε επίπεδό της.

Ακολούθησε η διαδικασία υλοποίησης της πιλοτικής πλατφόρμας Concordia, με στόχο να καταστεί ο αρχικός σχεδιασμός πραγματοποιήσιμος.

Τέλος, εξήχθησαν συμπεράσματα και διατυπώθηκαν πιθανές μελλοντικές επεκτάσεις για την εφαρμογή.
