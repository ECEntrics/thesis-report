\section{Γενικά}

Η αλματώδης ανάπτυξη του διαδικτύου διαμόρφωσε ένα ολοκαίνουργιο τοπίο σε κάθε τομέα της ανθρώπινης δραστηριότητας, παρέχοντας ένα αναρίθμητο πλήθος εφαρμογών και υπηρεσιών. Τα μέσα κοινωνικής δικτύωσης,
το ηλεκτρονικό ταχυδρομείο, η ψηφιακή ειδησεογραφία, ο διαμοιρασμός αρχείων και
οι υπηρεσίες πολυμέσων ροής, αποτελούν ορισμένα από τα σημαντικότερα - και πλέον αναπόσπαστα - κομμάτια,
που συνθέτουν την ψηφιακή πτυχή της σύγχρονης καθημερινότητας. 

Κατά κύριο λόγο, το μοντέλο που ακολουθούν οι παραπάνω τεχνολογίες είναι αυτό της αρχιτεκτονικής πελάτη-εξυπηρετητή (client–server architecture) και προσφέρονται από παρόχους υπηρεσιών υπολογιστικού νέφους (cloud computing service providers). Αυτό σημαίνει ότι οι απαραίτητες λειτουργίες τους, δηλαδή η επεξεργασία (processing), η αποθήκευση των δεδομένων (storage) και το πρωτόκολλο επικοινωνίας (communication protocol) υλοποιούνται επί ενός συγκεντρωτικού (centralized) πλαισίου, κάτι που τους προσδίδει ορισμένα αξιοσημείωτα πλεονεκτήματα (π.χ. ευκολία ανάπτυξης, συντήρησης και αποσφαλμάτωσης).

Στις μέρες μας, ωστόσο, παρατηρείται παράλληλα μία τάση δημιουργίας εφαρμογών που ακολουθούν αποκεντρωτικά μοντέλα λειτουργίας, στα οποία το processing και το storage κατανέμονται σε ένα σύνολο κόμβων που επικοινωνούν ομότιμα. Εντός, λοιπόν, αυτής της τάσης, αναπτύσσονται με ταχείς ρυθμούς διάφορα λογισμικά, τα οποία συνθέτουν ένα νέο, αποκεντρωτικό οικοσύστημα. Αυτό περιλαμβάνει (μεταξύ άλλων) τόσο καινοτόμα πρωτόκολλα αποθήκευσης δεδομένων (π.χ. IPFS), όσο και πλατφόρμες ανάπτυξης και εκτέλεσης αποκεντρωμένων εφαρμογών (π.χ. Ethereum blockchain).

\newpage