\section{Περί αποκέντρωσης}

Αν και ο όρος "αποκέντρωση" χρησιμοποιείται ευρέως στην επιστήμη των υπολογιστών και στα κρυπτοοικονομικά\footnote{Τα "κρυπτοοικονομικά" είναι η πρακτική επιστήμη της δημιουργίας κατανεμημένων συστημάτων, οι ιδιότητες των οποίων εξασφαλίζονται με οικονομικά κίνητρα, ενώ οι οικονομικοί τους μηχανισμοί είναι κρυπτογραφικά εγγυημένοι.\cite{1.2-ethereum-learn}} (cryptoeconomics), συνήθως ορίζεται αρκετά ασαφώς\cite{1.2-the-meaning-of-decentralization}. Στην πραγματικότητα η αποκέντρωση (ή, αντίστοιχα, ο συγκεντρωτισμός) μπορεί να τοποθετηθεί πάνω σε τρεις ξεχωριστούς άξονες, οι οποίοι είναι σε γενικές γραμμές ανεξάρτητοι ο ένας από τον άλλον. Αυτοί έχουν ως εξής:

\begin{enumerate}
	\item \textbf{Αρχιτεκτονική} αποκέντρωση: Από πόσους φυσικούς υπολογιστές αποτελείται ένα σύστημα; Πόσοι από αυτούς μπορούν, ανά πάσα στιγμή, να χαλάσουν και εκείνο να αντέξει;
	\item \textbf{Πολιτική} αποκέντρωση: Πόσα άτομα ή οργανισμοί ελέγχουν τους υπολογιστές από τους οποίους αποτελείται το σύστημα;
	\item \textbf{Λογική} αποκέντρωση: Η διεπαφή και οι δομές δεδομένων του συστήματος μοιάζουν περισσότερο με ένα μονολιθικό αντικείμενο ή ένα άμορφο σμήνος; Αν, δηλαδή, το σύστημα (συμπεριλαμβανομένων των παρόχων και των χρηστών) "κοπεί στη μέση", θα συνεχίσουν τα δύο μισά να λειτουργούν πλήρως ως ανεξάρτητες μονάδες; 
\end{enumerate}

Για παράδειγμα, το BitTorrent είναι αποκεντρωτικό ως προς όλους τους άξονες, ενώ ένα CDN (Content Delivery Network), είναι μόνο αρχιτεκτονικά και λογικά, αφού ελέγχεται από κάποια εταιρεία. Φυσικά, η έννοια μπορεί να γενικευθεί και να μιλάμε για αποκέντρωση μίας επιχείρησης (συνήθως πλήρως συγκεντρωτική) ή μίας γλώσσας (συνήθως πλήρως αποκεντρωτική).

Η επιλογή της δομής ενός συστήματος ως προς την αποκέντρωσή του βασίζεται στις εκάστοτε ανάγκες και στους στόχους του. Μερικά ισχυρά πλεονεκτήματα που διατυπώνονται συχνά για τα αποκεντρωτικά συστήματα είναι τα εξής:

\begin{itemize}
	\item \textbf{Ανοχή σε σφάλματα}: Τα αρχιτεκτονικά αποκεντρωμένα συστήματα είναι λιγότερο πιθανό να αποτύχουν τυχαία, επειδή βασίζονται σε πολλά ξεχωριστά στοιχεία που είναι απίθανο να παρουσιάσουν σφάλματα ταυτόχρονα.
	\item \textbf{Αντοχή σε επιθέσεις}: Το κόστος μίας επίθεσης, που έχει ως στόχο την καταστροφή ή τον χειρισμό ενός αποκεντρωτικού συστήματος, είναι πολύ ακριβό. Αυτό συμβαίνει επειδή δεν υπάρχει κάποιο ευαίσθητο κεντρικό σημείο στο οποίο να μπορεί να πραγματοποιηθεί μία επίθεση, η οποία να έχει κόστος πολύ χαμηλότερο από το οικονομικό μέγεθος του περιβάλλοντος συστήματος.
	\item \textbf{Απουσία ανάγκης εκχώρησης εμπιστοσύνης}: Σε ένα ιδανικό πολιτικά αποκεντρωμένο σύστημα οι χρήστες δε χρειάζεται να εμπιστεύονται κάποια κεντρική αρχή για την επεξεργασία και την αποθήκευση των δεδομένων. 
	\item \textbf{Αντίσταση σε συμπαιγνίες}: είναι πολύ πιο δύσκολο για τους συμμετέχοντες σε αποκεντρωμένα συστήματα να συνεργαστούν για να ενεργήσουν με τρόπο που τους ωφελεί σε βάρος άλλων συμμετεχόντων.
\end{itemize}

Ιδιαίτερα τα τελευταία χρόνια, παρατηρείται μία έντονη ανάγκη υλοποίησης αποκεντρωμένων εφαρμογών (decentralized applications), οι οποίες, πέρα από τα αρχιτεκτονικά πλεονεκτήματα που τις χαρακτηρίζουν (π.χ. σταθερότητα, ασφάλεια, επεκτασιμότητα), αποσκοπούν στην επίτευξη πολιτικής αποκέντρωσης. Αυτό πηγάζει τόσο από την ανάγκη προάσπισης των αρχών που καταστρατηγούνται όταν τα δεδομένα υπάγονται στον έλεγχο κάποιας κεντρικής διαχείρισης (π.χ. της ελευθερίας του λόγου, της ανωνυμίας και της ιδιωτικότητας του χρήστη), όσο και από την ανάγκη δημιουργίας διαδικασιών που απαιτούν εγκυρότητα και αυθεντικότητα, όπως όσων σχετίζονται με την αυτοδιαχείριση και την άμεση δημοκρατία. Ως απόγειο των παραπάνω, μπορούν να θεωρηθούν οι λεγόμενες \textit{αποκεντρωτικές αυτόνομες οργανώσεις} (decentralized autonomous organizations ή DAOs), οι οποίες αποτελούν μία μορφή αλγοκρατικής\footnote{Ο όρος "αλγοκρατία" (algocracy) αναφέρεται σε εναλλακτικές μορφές διακυβέρνησης που βασίζονται στη χρήση αλγορίθμων.\cite{1.2-virtual-migration}} οργάνωσης βασισμένης σε τεχνολογίες αποκέντρωσης και, κυρίως, στο blockchain.
