\section{Ορισμός του προβλήματος}

Η αλματώδης ανάπτυξη του διαδικτύου διαμόρφωσε ένα ολοκαίνουργιο τοπίο σε κάθε τομέα της ανθρώπινης δραστηριότητας, παρέχοντας ένα αναρίθμητο πλήθος εφαρμογών και υπηρεσιών. Τα μέσα κοινωνικής δικτύωσης,
το ηλεκτρονικό ταχυδρομείο, η ψηφιακή ειδησεογραφία, ο διαμοιρασμός αρχείων και
οι υπηρεσίες πολυμέσων ροής, αποτελούν ορισμένα από τα σημαντικότερα - και πλέον αναπόσπαστα - κομμάτια,
που συνθέτουν την ψηφιακή πτυχή της σύγχρονης καθημερινότητας. 

Κατά κύριο λόγο, το μοντέλο που ακολουθούν οι παραπάνω τεχνολογίες είναι αυτό της αρχιτεκτονικής πελάτη-εξυπηρετητή (client–server architecture) και προσφέρονται από παρόχους υπηρεσιών υπολογιστικού νέφους (cloud computing service providers). Αυτό σημαίνει ότι οι απαραίτητες λειτουργίες τους, δηλαδή η επεξεργασία (processing), η αποθήκευση των δεδομένων (storage) και το πρωτόκολλο επικοινωνίας (communication protocol) υλοποιούνται επί ενός συγκεντρωτικού (centralized) πλαισίου, κάτι που τους προσδίδει ορισμένα αξιοσημείωτα πλεονεκτήματα (π.χ. ευκολία ανάπτυξης, συντήρησης και αποσφαλμάτωσης). 

Από την άλλη πλευρά, η κεντροποιημένη λογική ενός τέτοιου συστήματος αφενός συνοδεύεται από ποικίλα μειονεκτήματα τεχνικής φύσεως, αφετέρου εγείρει σοβαρούς προβληματισμούς σχετικά με τη διαχείριση των προσωπικών δεδομένων των χρηστών από κεντρικές αρχές. Τα βασικότερα από τα παραπάνω θα μπορούσαν να συνοψιστούν ως εξής: 

\begin{itemize}
	\item Έλλειψη \textbf{εμπιστοσύνης}: Οι κεντρικές αρχές έχουν τη δυνατότητα να παρακολουθούν τους χρήστες, να διαβάζουν, ή ακόμα και να διαρρέουν τα προσωπικά τους δεδομένα εν αγνοία των τελευταίων. Οι δε χρήστες δε διαθέτουν κανέναν τρόπο με τον οποίον να μπορούν να τις εμπιστευθούν με βεβαιότητα.
	\item Έλλειψη εγγύησης της \textbf{αυθεντικότητας} των δεδομένων: Οι κεντρικές αρχές έχουν τη δυνατότητα να τροποποιούν τα δεδομένα κατά βούληση κάτι που έχει ως αποτέλεσμα να μην υπάρχει εγγύηση ως προς την αυθεντικότητα όσων βλέπουν οι χρήστες.
	\item Έλλειψη εγγύησης της \textbf{ελευθερίας του λόγου}: Οι κεντρικές αρχές έχουν τη δυνατότητα να λογοκρίνουν τα δεδομένα, είτε βάσει των συμφερόντων τους, είτε βάσει υποχρεώσεών τους ως προς τρίτους.
	\item Έλλειψη \textbf{ασφάλειας}: Τα προσωπικά δεδομένα των χρηστών μπορεί να υποκλαπούν εξαιτίας κάποιας κυβερνοεπίθεσης.
	\item Έλλειψη \textbf{διαθεσιμότητας}: Το σύστημα μπορεί να σταματήσει να λειτουργεί προσωρινά ή μόνιμα για τεχνικούς, οικονομικούς ή νομικούς λόγους.
\end{itemize}
