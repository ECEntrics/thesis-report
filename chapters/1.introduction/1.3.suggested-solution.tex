\section{Προτεινόμενη λύση}

Το Concordia είναι η εφαρμογή η οποία αναπτύσσουμε εμείς και στοχεύει να διορθώσει αυτά τα προβλήματα, επαναφέροντας στους χρήστες την κυριότητα των δεδομένων τους, εξασφαλίζοντας την πλήρη ελευθερία του λόγου και την αυθεντικότητα, ανοίγοντας τον δρόμο για αξιόπιστες ψηφοφορίες
Όλα αυτά μέσα από δημόσιες, αποκεντρωτικές διαδικασίες.

\subsection{Απαιτήσεις}

\subsection{Αποκέντρωση}

% Παλιό από Drive
Αποκέντρωση του συστήματος σημαίνει πρακτικά ότι το processing και η αποθήκευση των δεδομένων δε θα γίνονται από κάποια κεντρική αρχή αλλά θα είναι κατανεμημένα στο σύνολο των χρηστών (nodes). Με αυτόν τον τρόπο δεν υπάρχει ανάγκη για μία κεντρική αρχή και τα δεδομένα δεν είναι ελέγξιμα από κανέναν ατομικά, παρά μόνο από τη συναίνεση (consensus) του δικτύου.

Τα συγκεντρωτικά συστήματα έχουν μερικά θετικά χαρακτηριστικά που λείπουν από τα αποκεντρωτικά συστήματα, όπως ευκολία ανάπτυξης, συντήρησης και αποσφαλμάτωσης των εφαρμογών. Πάσχουν ωστόσο σε ό,τι αφορά την σταθερότητα, την ασφάλεια, την επεκτασιμότητα και την εξέλιξη, τομείς όπου τα αποκεντρωτικά συστήματα είναι ιδιαίτερα αποτελεσματικά.

Η ανάγκη για αποκέντρωση των εφαρμογών, ειδικά στην επικοινωνία, είναι μεγάλη και πηγάζει από την ανάγκη για ελευθερία του λόγου, ανωνυμία και ιδιωτικότητα. Χρησιμοποιώντας τεχνικές για κατανομή του processing, μία διανεμημένη βάση δεδομένων και αλγόριθμους κρυπτογραφίας δημόσιου κλειδιού μπορούμε να προστατεύσουμε την ανωνυμία του χρήστη αλλά και να εγγυηθούμε την ταυτοποίησή του.

\subsection{Αμεσοδημοκρατικές διαδικασίες και αυτοδιαχείριση}

% Παλιό από Drive
Για την πλήρη επίτευξη του στόχου απαιτείται επίσης ένα σύστημα διαχείρισης της πλατφόρμας αυτής καθ’ αυτής αλλά και των περιεχομένων της. Το σύστημα που επιλέγουμε για αυτούς τους σκοπούς είναι αυτό της άμεσης δημοκρατίας και αυτοδιαχείρισης. Αυτό σημαίνει ότι οι αποφάσεις θα παίρνονται μέσα από ψηφοφορίες στις οποίες θα μπορούν να συμμετέχουν όσα μέλη έχουν δικαίωμα ψήφου. Έτσι, λόγω της αποκέντρωσης και άρα της έλλειψης διοικούσας αρχής, η πλατφόρμα μπορεί να χρησιμοποιηθεί σαν μία εγγυημένα αμερόληπτη αρχή για ψηφοφορίες πάνω σε θέματα που αφορούν τη φοιτητική ζωή και όχι μόνο.
