\section{Λογικά μέρη} \label{section:3-1-logical-parts}

% Παλιό από Drive
Η πλατφόρμα μπορεί να διαχωριστεί σε δύο λογικά μέρη:

1) Το πρώτο μέρος αποτελεί μία αυτοτελή και πλήρως αποκεντρωτική πλατφόρμα που στόχος της είναι να παρέχει τη δυνατότητα ελευθερίας λόγου απρόσβλητου σε λογοκρισία και διαγραφή από κεντρικές οντότητες εξουσίας. Στην ουσία εδώ θα μπορεί - σε μια απλοποιημένη εκδοχή - οποιοσδήποτε να δημιουργήσει topics ή να απαντήσει σε άλλα. Επεξεργαστικός πυρήνας θα είναι ένα smart contract το οποίο θα δέχεται από τους χρήστες transactions και θα τα εγγράφει στο Storage Layer:

% TODO: insert diagram

Το μειονέκτημα αυτού του κομματιού είναι πως για να λειτουργήσει απαιτεί για κάθε \textenglish{transaction} οι χρήστες να καταβάλουν κάποιο τέλος για τα fees. Αν και υπάρχουν σχέδια από την ομάδα ανάπτυξης του Ethereum για τη μείωσή τους στο μέλλον (έως και 10-2), τα fees στη χρήση της EVM θα είναι αναπόφευκτα. Ωστόσο θα πρέπει να σημειωθεί ότι αφενός παρέχουν μια μορφή προστασίας απέναντι σε κακόβουλους χρήστες που θα πλημμύριζαν την εφαρμογή με ανεπιθύμητο ποσοτικά/ποιοτικά περιεχόμενο (θα τους ήταν οικονομικά ασύμφορη μια τέτοια επίθεση), αφετέρου υπάρχουν κάποια workarounds για να μειωθεί δραστικά η εμπλοκή του χρήστη με την καταβολή τελών, κάτι που περιγράφεται παρακάτω στις κατηγορίες χρηστών. Τα παραπάνω πρακτικά σημαίνουν ότι το Smart Contract θα πρέπει ανά διαστήματα να φορτίζεται (από οποιονδήποτε) με Ethereum ή οι χρήστες (βλ. κατηγορίες χρηστών) να καταβάλουν τα δικά τους έξοδα.

2) Το δεύτερο μέρος αποτελεί μια μερικώς αποκεντρωτική και μη αυτοτελή πλατφόρμα που έρχεται να λειτουργήσει επιπρόσθετα στην πρώτη. Το κομμάτι αυτό απευθύνεται αποκλειστικά σε επικυρωμένα μέλη του ΑΠΘ και συνιστά ένα αμεσοδημοκρατικό σύστημα ψηφοφορίας που θα εγγυάται σε υψηλό βαθμό την εγκυρότητα και την ανωνυμία των διαδικασιών του. Με λίγα λόγια, θα δημιουργούνται θέματα προς ψηφοφορία (στο πρώτο κομμάτι), πάνω στα οποία θα ψηφίζουν όσοι έχουν το δικαίωμα (αυτοί θα ορίζονται με την κατοχή ενός ανάλογου token στο Ethereum).

% TODO: insert diagram

O χρήστης μέσω ενός Frontend (μιας κλασικής ιστοσελίδας ουσιαστικά) θα μπορεί να πιστοποιήσει μέσω login στο it.auth.gr* την ακαδημαϊκή του ταυτότητα. Στη συνέχεια το TDS (Token Distribution Service, ελέγχοντας το admin account των tokens θα παρέχει στο χρήστη δύο tokens.  Ένα το οποίο θα του δίνει voting rights (verified user) και ένα που θα κάνει τη πλατφόρμα να του επιστρέφει τα τέλη τα οποία πληρώνουν σε κάθε post τους (trusted user).


*Ιδανικά, με τη συνεργασία του ΑΠΘ, το UAS θα αποτελεί υπηρεσία συνεργαζόμενη με την Υποδομή πιστοποίησης και εξουσιοδότησης (βλ. https://it.auth.gr/el/infrastructure/aai). Αυτό σημαίνει ότι οι χρήστες δε θα χρειάζεται να εμπιστευτούν τον UAS με τα it credential τους.
