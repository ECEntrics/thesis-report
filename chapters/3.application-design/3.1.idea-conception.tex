\section{Σύλληψη της ιδέας} \label{section:3-1-idea-conception}

Η σύλληψη της ιδέας για τη δημιουργία της εφαρμογής της παρούσας διπλωματικής εργασίας είχε ως εφαλτήριο την αναγνώριση ενός διδιάστατου προβλήματος.

Η πρώτη διάσταση εστιάζει στον χώρο των μέσων κοινωνικής δικτύωσης. Εκεί παρατηρείται αδιαμφισβήτητη επικράτηση πλατφορμών επικοινωνίας συγκεντρωτικής μορφής (π.χ. Facebook, Twitter, Instagram), ενώ προσπάθειες δημιουργίας αντίστοιχων αποκεντρωτικών εφαρμογών βρίσκονται σε πρώιμα στάδια, τόσο ανάπτυξης, όσο και υιοθέτησης από το ευρύ κοινό. Όπως αναλύθηκε και στην ενότητα \ref{section:1-3-problem-definition}, η τρέχουσα αυτή κατάσταση θέτει αξιοσημείωτα προβλήματα τεχνικής φύσεως (έλλειψη ασφάλειας και διαθεσιμότητας) και, κυρίως, πολιτικής (έλλειψη εμπιστοσύνης, εγγύησης της αυθεντικότητας των δεδομένων και της ελευθερίας του λόγου).

Η δεύτερη διάσταση εστιάζει στον χώρο της ψηφιακής δημοκρατίας (digital democracy). Συγκεκριμένα, παρατηρείται έλλειψη  εργαλείων, ικανών να παρέχουν τη δυνατότητα διενέργειας αυθεντικών δημοκρατικών διαδικασιών. Ψηφοφορίες και αυτοδιαχείριση εντός συστημάτων κεντροποιημένης λογικής αδυνατούν, για αρχιτεκτονικούς λόγους, να εξασφαλίσουν τις απαραίτητες θεμελιώδεις ιδιότητες τέτοιων διαδικασιών, δηλαδή την ανωνυμία και την επαληθευσιμότητα.

Αυτές οι παρατηρήσεις αποτέλεσαν την έμπνευση για τη δημιουργία μίας εφαρμογής, η οποία, μέσω ενός επιλεγμένου συνδυασμού αποκεντρωτικών τεχνολογιών, να ορίσει έναν ψηφιακό χώρο που θα έρθει αντιμέτωπος με το παραπάνω πρόβλημα. Έτσι, κεντρικός στόχος της πιλοτικής εφαρμογής Concordia, είναι να αποτελέσει μία αυτόνομη κοινωνική πλατφόρμα, που θα κατοχυρώνει στους χρήστες της ελευθερία του λόγου και πλήρη κυριότητα επί των δεδομένων τους. Επιπλέον, θα παρέχει τη δυνατότητα διενέργειας αυθεντικών, ανώνυμων ψηφοφοριών, κάτι που θα την καθιστά ένα αξιόπιστο δημοκρατικό βήμα για τη λήψη αποφάσεων εντός των αυτοδιαχειριζόμενων κοινοτήτων της.
