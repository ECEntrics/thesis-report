\section{Σενάρια χρήσης} \label{section:3-6-use-cases}

Βασικό μέρος της σχεδίασης της πλατφόρμας ήταν η καταγραφή των απαιτήσεων, η οποία έγινε στην προηγούμενη ενότητα (\ref{section:3-5-software-requirements}), καθώς και η σχεδίαση και ανάπτυξη των σεναρίων χρήσης. Τα σενάρια χρήσης αντιστοιχίζουν πιθανές ενέργειες των χρηστών με αποκρίσεις του συστήματος. Μέσω αυτής της αντιστοίχισης, παρουσιάζεται η λειτουργικότητα του συστήματος και περιγράφονται τόσο οι λειτουργικές, όσο και οι μη λειτουργικές του απαιτήσεις.

Στις επόμενες υποενότητες παρατίθενται τα σενάρια χρήσης (<ΣΧ>) που δίνουν τις απαραίτητες πληροφορίες για την κατανόηση της λειτουργίας του συστήματος.

\input{chapters/3.application-design/3.6.use-cases/3.6.1.use-case-sign-up}
\input{chapters/3.application-design/3.6.use-cases/3.6.2.use-case-sign-in}
\input{chapters/3.application-design/3.6.use-cases/3.6.3.use-case-create-topic}
% ===== =====
% Use case 4
% ===== =====
\subsection{Σενάριο χρήσης 4: Ανάκτηση θέματος} \label{subsection:3-6-use-case-fetch-topic}

Το σενάριο χρήσης 4, <ΣΧ-4>, περιγράφει τις διαδοχικές ενέργειες που εκτελούνται για την ανάκτηση ενός θέματος. Στους πίνακες \ref{table:3-6-use-case-fetch-topic} και \ref{table:3-6-use-case-fetch-topic-base-flow} παρατίθενται οι βασικές πληροφορίες του <ΣΧ-4> και οι ενέργειες της βασικής ροής αντίστοιχα, ενώ στο σχήμα \ref{figure:3-6-use-case-fetch-topic-base-flow-sequence-diagram} φαίνεται το διάγραμμα της βασικής ροής.

\useCaseTable
{Ανακτώ ένα θέμα}
{Στόχος του σεναρίου χρήσης είναι ο επισκέπτης ή ο χρήστης να μπορεί να ανακτήσει ένα θέμα.}
{\ref{srs:functional-srs-browse-community-topics}}
{-}
{Ο επισκέπτης ή χρήστης πατάει σε ένα από τα θέματα.}
{Ο επισκέπτης ή χρήστης πρέπει να έχει ανοίξει την σελίδα της εφαρμογής.}
{Σενάριο χρήσης 4, ανάκτηση θέματος.}
{\label{table:3-6-use-case-fetch-topic}}

% ===== Base flow =====

\useCaseBaseFlowTable
{
    1 & Ο χρήστης πατάει σε ένα από τα θέματα της λίστας. & Το σύστημα ανακτά τις πληροφορίες του θέματος από το blockchain. \\ [0.5ex]
    \midrule
    2 & -                                                 & Το σύστημα ανακτά τα μηνύματα του θέματος αντιγράφοντας τις προσωπικές βάσεις OrbitDb των συγγραφέων. \\ [0.5ex]
}
{Το σύστημα μεταβαίνει στην σελίδα του θέματος.}
{Σενάριο χρήσης 4 - Βασική ροή}
{\label{table:3-6-use-case-fetch-topic-base-flow}}

\begin{figure}[H]
    \centering
    \input{tikz/chapter-3/3-6-use-case-fetch-topic-sequence-diagram}
    \caption{Σενάριο χρήσης 4 - Διάγραμμα βασικής ροής}
    \label{figure:3-6-use-case-fetch-topic-base-flow-sequence-diagram}
\end{figure}

% ===== Alternate flow =====

Το <ΣΧ-4> περιέχει επίσης μία εναλλακτική ροή που μπορεί να προκύψει βάσει των επιλογών του χρήστη και η οποία περιγράφεται στον πίνακα \ref{table:3-6-use-case-fetch-topic-alternate-flow-1}. Η εναλλακτική ροή 1 φαίνεται επίσης στο σχήμα \ref{figure:3-6-use-case-fetch-topic-alternate-flow-1-sequence-diagram} όπου παρουσιάζεται το διάγραμμα ροής της.

\useCaseAlternateFlowTable
{1}
{Το θέμα περιέχει ψηφοφορία.}
{Εφόσον το θέμα που ανακτήθηκε στη γραμμή 1 της Βασικής Ροής περιέχει ψηφοφορία ανακτώνται οι πληροφορίες της.}
{
    1 & - & Το σύστημα ανακτά τα μηνύματα του θέματος αντιγράφοντας τις προσωπικές βάσεις OrbitDb των συγγραφέων. \\ [0.5ex]
    2 & - & Το σύστημα ανακτά την ψηφοφορία από το blockchain. \\ [0.5ex]
    3 & - & Το σύστημα ανακτά τις πληροφορίες της ψηφοφορίας αντιγράφοντας την προσωπική βάση OrbitDb του συγγραφέα. \\ [0.5ex]
    4 & - & Το σύστημα επιβεβαιώνει τις πληροφορίες της ψηφοφορίας με βάση το hash που έχει ανακτηθεί από το blockchain. \\ [0.5ex]
}
{Το σενάριο χρήσης τερματίζεται.}
{Σενάριο χρήσης 4 - Εναλλακτική ροή 1}
{\label{table:3-6-use-case-fetch-topic-alternate-flow-1}}

\begin{figure}[H]
    \centering
    \input{tikz/chapter-3/3-6-use-case-fetch-topic-alternate-flow-1-sequence-diagram}
    \caption{Σενάριο χρήσης 4 - Διάγραμμα εναλλακτικής ροής 1}
    \label{figure:3-6-use-case-fetch-topic-alternate-flow-1-sequence-diagram}
\end{figure}

% ===== =====
% Use case 5
% ===== =====
\newpage
\subsection{Σενάριο χρήσης 5: Δημιουργία νέου μηνύματος} \label{subsection:3-6-use-case-create-post}

Το σενάριο χρήσης 5 (<ΣΧ-5>) περιγράφει τις διαδοχικές ενέργειες που εκτελούνται για τη δημιουργία ενός μηνύματος. Στους πίνακες \ref{table:3-6-use-case-create-post} και \ref{table:3-6-use-case-create-post-base-flow} παρατίθενται οι βασικές πληροφορίες του <ΣΧ-5> και οι ενέργειες της βασικής ροής αντίστοιχα, ενώ στο σχήμα \ref{figure:3-6-use-case-create-post-base-flow-sequence-diagram} φαίνεται το διάγραμμα της βασικής ροής.

\useCaseTable
{Δημιουργώ νέο μήνυμα}
{Στόχος του σεναρίου χρήσης είναι ο χρήστης να μπορεί να δημιουργήσει νέο μήνυμα.}
{\ref{srs:functional-srs-create-post}}
{\ref{srs:non-functional-srs-minimize-fees}}
{Ο χρήστης πατάει το κουμπί δημιουργίας νέου μηνύματος.}
{Ο χρήστης να έχει συνδεθεί στην εφαρμογή και να βρίσκεται στη σελίδα ενός θέματος.}
{Σενάριο χρήσης 5, δημιουργία νέου μηνύματος}
{\label{table:3-6-use-case-create-post}}

% ===== Base flow =====

\useCaseBaseFlowTable
{
    1 & Ο χρήστης πατάει το κουμπί δημιουργίας νέου μηνύματος.           & Το σύστημα εμφανίζει τη φόρμα "Δημιουργία Μηνύματος". \\ [0.5ex]
    \midrule
    2 & Ο χρήστης συμπληρώνει τα πεδία και πατάει το κουμπί "Υποβολή". & Το σύστημα εισάγει νέο μήνυμα στο blockchain. \\ [0.5ex]
    \midrule
    3 & -                                                                & Το σύστημα εισάγει τις πληροφορίες του μηνύματος στην προσωπική OrbitDB βάση του χρήστη. \\ [0.5ex]
}
{Το σύστημα παραμένει στη σελίδα του θέματος εμφανίζοντας το νέο μήνυμα.}
{Σενάριο χρήσης 5 - Βασική ροή}
{\label{table:3-6-use-case-create-post-base-flow}}

\begin{figure}[H]
    \centering
    \begin{sequencediagram}
    \newthread{actor}{Actor}{}
    \newinst[3]{concordia}{:Concordia}{}
    \newinst[2]{eth}{:Ethereum}{}
    \newinst[1]{orbit}{:OrbitDb}{}

    \begin{call}{actor}{Create post}{concordia}{Post creation form}
    \end{call}

    \begin{call}{actor}{Submit}{concordia}{Updated topic page}

        \begin{call}{concordia}{Create post}{eth}{New post ID}
        \end{call}

        \begin{call}{concordia}{Save post information}{orbit}{}
        \end{call}

    \end{call}
\end{sequencediagram}

    \caption{Σενάριο χρήσης 5 - Διάγραμμα βασικής ροής}
    \label{figure:3-6-use-case-create-post-base-flow-sequence-diagram}
\end{figure}

% ===== Alternate flow =====

Το <ΣΧ-5> περιέχει επίσης μία εναλλακτική ροή που μπορεί να προκύψει βάσει των επιλογών του χρήστη και η οποία περιγράφεται στον πίνακα \ref{table:3-6-use-case-create-post-alternate-flow-1}.

\useCaseAlternateFlowTable
{1}
{Ο χρήστης πατάει το κουμπί "Άκυρο".}
{Εφόσον ο χρήστης στη γραμμή 2 της Βασικής Ροής επιλέξει "Άκυρο" το σύστημα επιστρέφει στη σελίδα του θέματος.}
{
    1 & Ο χρήστης πατάει το κουμπί "Άκυρο" & Το σύστημα επιστρέφει στη σελίδα του θέματος.
}
{Το σενάριο χρήσης τερματίζεται.}
{Σενάριο χρήσης 5 - Εναλλακτική ροή 1}
{\label{table:3-6-use-case-create-post-alternate-flow-1}}

% ===== =====
% Use case 6
% ===== =====
\subsection{Σενάριο χρήσης 6: Τροποποίηση μηνύματος} \label{subsection:3-6-use-case-modify-post}

Το σενάριο χρήσης 6 (<ΣΧ-6>) περιγράφει τις διαδοχικές ενέργειες που εκτελούνται για την τροποποίηση ενός μηνύματος. Στους πίνακες \ref{table:3-6-use-case-modify-post} και \ref{table:3-6-use-case-modify-post-base-flow} παρατίθενται οι βασικές πληροφορίες του <ΣΧ-6> και οι ενέργειες της βασικής ροής αντίστοιχα, ενώ στο σχήμα \ref{figure:3-6-use-case-modify-post-base-flow-sequence-diagram} φαίνεται το διάγραμμα της βασικής ροής.

\useCaseTable
{Τροποποιώ ένα μήνυμα}
{Στόχος του σεναρίου χρήσης είναι ο χρήστης να μπορεί να τροποποιήσει τα μηνύματά του.}
{\ref{srs:functional-srs-modify-post}}
{-}
{Ο χρήστης πατάει το κουμπί τροποποίησης του μηνύματος.}
{Ο χρήστης πρέπει να έχει συνδεθεί στην εφαρμογή και να βρίσκεται στη σελίδα του θέματος που περιέχει το μήνυμά του.}
{Σενάριο χρήσης 6, τροποποίηση μηνύματος}
{\label{table:3-6-use-case-modify-post}}

% ===== Base flow =====

\useCaseBaseFlowTable
{
    1 & Ο χρήστης πατάει το κουμπί τροποποίησης του μηνύματος.           & Το σύστημα εμφανίζει τη φόρμα "Τροποποίηση Μηνύματος". \\ [0.5ex]
    \midrule
    2 & Ο χρήστης συμπληρώνει τα πεδία και πατάει το κουμπί "Υποβολή". & Το σύστημα τροποποιεί τις πληροφορίες του μηνύματος στην προσωπική OrbitDB βάση του χρήστη. \\ [0.5ex]
}
{Το σύστημα παραμένει στη σελίδα του θέματος εμφανίζοντας το τροποποιημένο μήνυμα.}
{Σενάριο χρήσης 6 - Βασική ροή}
{\label{table:3-6-use-case-modify-post-base-flow}}

\begin{figure}[H]
    \centering
    \begin{sequencediagram}
    \newthread{actor}{Actor}{}
    \newinst[4]{concordia}{:Concordia}{}
    \newinst[1]{eth}{:Ethereum}{}
    \newinst[2]{orbit}{:OrbitDB}{}

    \begin{call}{actor}{Modify post}{concordia}{Post modification form}
    \end{call}

    \begin{call}{actor}{Submit}{concordia}{Updated topic page}

        \begin{call}{concordia}{Save modified post information}{orbit}{}
        \end{call}

    \end{call}
\end{sequencediagram}

    \caption{Σενάριο χρήσης 6 - Διάγραμμα βασικής ροής}
    \label{figure:3-6-use-case-modify-post-base-flow-sequence-diagram}
\end{figure}

% ===== Alternate flow =====

Το <ΣΧ-6> περιέχει επίσης μία εναλλακτική ροή που μπορεί να προκύψει βάσει των επιλογών του χρήστη και η οποία περιγράφεται στον πίνακα \ref{table:3-6-use-case-modify-post-alternate-flow-1}.

\useCaseAlternateFlowTable
{1}
{Ο χρήστης πατάει το κουμπί "Άκυρο".}
{Εφόσον ο χρήστης στη γραμμή 2 της Βασικής Ροής επιλέξει "Άκυρο" το σύστημα επιστρέφει στη σελίδα του θέματος.}
{
    1 & Ο χρήστης πατάει το κουμπί "Άκυρο" & Το σύστημα επιστρέφει στη σελίδα του θέματος.
}
{Το σενάριο χρήσης τερματίζεται.}
{Σενάριο χρήσης 6 - Εναλλακτική ροή 1}
{\label{table:3-6-use-case-modify-post-alternate-flow-1}}

% ===== =====
% Use case 7
% ===== =====
\subsection{Σενάριο χρήσης 7: Ψήφιση σε ψηφοφορία} \label{subsection:3-6-use-case-vote-in-poll}

Το σενάριο χρήσης 7, <ΣΧ-7>, περιγράφει τις διαδοχικές ενέργειες που εκτελούνται για την ψήφιση σε μία ψηφοφορία. Στους πίνακες \ref{table:3-6-use-case-vote-in-poll} και \ref{table:3-6-use-case-vote-in-poll-base-flow} παρατίθενται οι βασικές πληροφορίες του <ΣΧ-7> και οι ενέργειες της βασικής ροής αντίστοιχα, ενώ στο σχήμα \ref{figure:3-6-use-case-vote-in-poll-base-flow-sequence-diagram} φαίνεται το διάγραμμα της βασικής ροής.

\useCaseTable
{Ψηφίζω σε ψηφοφορία}
{Στόχος του σεναρίου χρήσης είναι ο χρήστης να μπορεί να ψηφίσει σε μία ψηφοφορία.}
{\ref{srs:functional-srs-vote-polls}}
{\ref{srs:non-functional-srs-minimize-fees}}
{Ο χρήστης πατάει το κουμπί ψηφοφορίας.}
{Ο χρήστης να έχει συνδεθεί στην εφαρμογή και να βρίσκεται στην σελίδα ενός θέματος το οποίο περιλαμβάνει ψηφοφορία.}
{Σενάριο χρήσης 7, ψήφιση σε ψηφοφορία.}
{\label{table:3-6-use-case-vote-in-poll}}

% ===== Base flow =====

\useCaseBaseFlowTable
{
    1 & Ο χρήστης πατάει το κουμπί της επιλογής που επιθυμεί να ψηφίσει και πατάει το κουμπί ``Υποβολή''. & Το σύστημα εισάγει νέα ψήφο στο blockchain. \\ [0.5ex]
}
{Το σύστημα ανανεώνει τις πληροφορίες της ψηφοφορίας.}
{Σενάριο χρήσης 7 - Βασική ροή}
{\label{table:3-6-use-case-vote-in-poll-base-flow}}

\begin{figure}[H]
    \centering
    \input{tikz/chapter-3/3-6-use-case-vote-in-poll-sequence-diagram}
    \caption{Σενάριο χρήσης 7 - Διάγραμμα βασικής ροής}
    \label{figure:3-6-use-case-vote-in-poll-base-flow-sequence-diagram}
\end{figure}

% ===== =====
% Use case 8
% ===== =====
\subsection{Σενάριο χρήσης 8: Ψήφιση μηνύματος} \label{subsection:3-6-use-case-vote-post}

Το σενάριο χρήσης 8 (<ΣΧ-8>) περιγράφει τις διαδοχικές ενέργειες που εκτελούνται για την ψήφιση σε ένα μήνυμα. Στους πίνακες \ref{table:3-6-use-case-vote-post} και \ref{table:3-6-use-case-vote-post-base-flow} παρατίθενται οι βασικές πληροφορίες του <ΣΧ-8> και οι ενέργειες της βασικής ροής αντίστοιχα, ενώ στο σχήμα \ref{figure:3-6-use-case-vote-post-base-flow-sequence-diagram} φαίνεται το διάγραμμα της βασικής ροής.

\useCaseTable
{Ψηφίζω σε μήνυμα}
{Στόχος του σεναρίου χρήσης είναι ο χρήστης να μπορεί να υπερψηφίσει ή καταψηφίσει ένα μήνυμα.}
{\ref{srs:functional-srs-vote-posts}}
{\ref{srs:non-functional-srs-minimize-fees}}
{Ο χρήστης πατάει το κουμπί υπερψήφισης ή καταψήφισης.}
{Ο χρήστης πρέπει να έχει συνδεθεί στην εφαρμογή και να βρίσκεται στη σελίδα ενός θέματος το οποίο περιλαμβάνει τουλάχιστον ένα μήνυμα το οποίο δεν έχει δημιουργήσει ο ίδιος.}
{Σενάριο χρήσης 8, ψήφιση μηνύματος}
{\label{table:3-6-use-case-vote-post}}

% ===== Base flow =====

\useCaseBaseFlowTable
{
    1 & Ο χρήστης πατάει στο κουμπί υπερψήφισης μηνύματος. & Το σύστημα εισάγει νέα ψήφο μηνύματος στο blockchain. \\ [0.5ex]
}
{Το σύστημα ανανεώνει τις ψήφους του μηνύματος.}
{Σενάριο χρήσης 8 - Βασική ροή}
{\label{table:3-6-use-case-vote-post-base-flow}}

\begin{figure}[H]
    \centering
    \begin{sequencediagram}
    \newthread{actor}{Actor}{}
    \newinst[3]{concordia}{:Concordia}{}
    \newinst[2]{eth}{:Ethereum}{}

    \begin{call}{actor}{Submit post vote}{concordia}{Updated topic page}

        \begin{call}{concordia}{Add post vote}{eth}{}
        \end{call}

    \end{call}
\end{sequencediagram}

    \caption{Σενάριο χρήσης 8 - Διάγραμμα βασικής ροής}
    \label{figure:3-6-use-case-vote-post-base-flow-sequence-diagram}
\end{figure}

\input{chapters/3.application-design/3.6.use-cases/3.6.9.delete-local-data}
% ===== =====
% Use case 10
% ===== =====
\subsection{Σενάριο χρήσης 10: Δημιουργία κοινότητας} \label{subsection:3-10-use-case-create-community}

Το σενάριο χρήσης 10 (<ΣΧ-10>) περιγράφει τις διαδοχικές ενέργειες που εκτελούνται για τη δημιουργία μίας κοινότητας. Στους πίνακες \ref{table:3-6-use-case-create-community} και \ref{table:3-6-use-case-create-community-base-flow} παρατίθενται οι βασικές πληροφορίες του <ΣΧ-10> και οι ενέργειες της βασικής ροής αντίστοιχα, ενώ στο σχήμα \ref{figure:3-6-use-case-create-community-base-flow-sequence-diagram} φαίνεται το διάγραμμα της βασικής ροής.

\useCaseTable
{Δημιουργώ νέα κοινότητα}
{Στόχος του σεναρίου χρήσης είναι ο χρήστης να μπορεί να δημιουργήσει μία νέα κοινότητα.}
{\ref{srs:functional-srs-create-communities}, \ref{srs:functional-srs-assign-community-contract}}
{\ref{srs:non-functional-srs-minimize-fees}}
{Ο χρήστης πατάει το κουμπί δημιουργίας νέας κοινότητας.}
{Ο χρήστης να έχει συνδεθεί στην εφαρμογή και να βρίσκεται στην αρχική σελίδα.}
{Σενάριο χρήσης 10, δημιουργία νέας κοινότητας}
{\label{table:3-6-use-case-create-community}}

% ===== Base flow =====

\useCaseBaseFlowTable
{
    1 & Ο χρήστης πατάει το κουμπί δημιουργίας νέας κοινότητας.          & Το σύστημα εμφανίζει τη φόρμα "Δημιουργία Κοινότητας". \\ [0.5ex]
    \midrule
    2 & Ο χρήστης συμπληρώνει τα πεδία και πατάει το κουμπί "Υποβολή". & Το σύστημα δημιουργεί νέα κοινότητα στο blockchain. \\ [0.5ex]
}
{Το σύστημα μεταβαίνει στη σελίδα της νέας κοινότητας.}
{Σενάριο χρήσης 10 - Βασική ροή}
{\label{table:3-6-use-case-create-community-base-flow}}

\begin{figure}[H]
    \centering
    \begin{sequencediagram}
    \newthread{actor}{Actor}{}
    \newinst[3]{concordia}{:Concordia}{}
    \newinst[2]{eth}{:Ethereum}{}
    \newinst[1]{orbit}{:OrbitDB}{}

    \begin{call}{actor}{Create community}{concordia}{Community creation form}
    \end{call}

    \begin{call}{actor}{Submit}{concordia}{Created community page}

        \begin{call}{concordia}{Create community}{eth}{New community ID}
        \end{call}

    \end{call}
\end{sequencediagram}

    \caption{Σενάριο χρήσης 10 - Διάγραμμα βασικής ροής}
    \label{figure:3-6-use-case-create-community-base-flow-sequence-diagram}
\end{figure}
\vspace{\baselineskip}

% ===== Alternate flow =====

Το <ΣΧ-10> περιέχει επίσης δύο εναλλακτικές ροές που μπορεί να προκύψουν βάσει των επιλογών του χρήστη και οι οποίες περιγράφονται στους πίνακες \ref{table:3-6-use-case-create-community-alternate-flow-1} και \ref{table:3-6-use-case-create-community-alternate-flow-2}. Η εναλλακτική ροή 1 φαίνεται επίσης στο σχήμα \ref{figure:3-6-use-case-create-community-alternate-flow-1-sequence-diagram} όπου παρουσιάζεται το διάγραμμα ροής της.

\useCaseAlternateFlowTable
{1}
{Ο χρήστης ορίζει εξωτερικό contract για την κοινότητα.}
{Εφόσον ο χρήστης στη γραμμή 2 της Βασικής Ροής επιλέξει "Προσθήκη Συμβολαίου" το σύστημα ανανεώνει τη σελίδα προσθέτοντας τα επιπλέον πεδία της φόρμας "Σύνδεση Συμβολαίου".}
{
    1 & Ο χρήστης, αφού συμπληρώσει τη φόρμα "Δημιουργία Κοινότητας", πατάει το κουμπί "Προσθήκη ψηφοφορίας" & Το σύστημα ανανεώνει τη σελίδα με τα πεδία της φόρμας "Σύνδεση Συμβολαίου". \\ [0.5ex]
    \midrule
    2 & Ο χρήστης συμπληρώνει τα πεδία και πατάει το κουμπί "Υποβολή".                                         & Το σύστημα δημιουργεί τη νέα κοινότητα στο blockchain και τη συνδέει με το εξωτερικό contract. \\ [0.5ex]
}
{Το σύστημα μεταβαίνει στη σελίδα της νέας κοινότητας.}
{Σενάριο χρήσης 10 - Εναλλακτική ροή 1}
{\label{table:3-6-use-case-create-community-alternate-flow-1}}

\begin{figure}[H]
    \centering
    \begin{sequencediagram}
    \newthread{actor}{Actor}{}
    \newinst[4]{concordia}{:Concordia}{}
    \newinst[4]{eth}{:Ethereum}{}
    \newinst{orbit}{:OrbitDb}{}

    \begin{call}{actor}{Create community}{concordia}{Community creation form}
    \end{call}

    \begin{call}{actor}{Add external contract}{concordia}{External contract form}
    \end{call}

    \begin{call}{actor}{Submit}{concordia}{Created community page}

        \begin{call}{concordia}{Create community}{eth}{New community ID}
        \end{call}

        \begin{call}{concordia}{Connect external contract}{eth}{}
        \end{call}
    \end{call}
\end{sequencediagram}

    \caption{Σενάριο χρήσης 3 - Διάγραμμα εναλλακτικής ροής 1}
    \label{figure:3-6-use-case-create-community-alternate-flow-1-sequence-diagram}
\end{figure}

\useCaseAlternateFlowTable
{2}
{Ο χρήστης πατάει το κουμπί "Άκυρο".}
{Εφόσον ο χρήστης στη γραμμή 2 της Βασικής Ροής ή στη γραμμή 2 της Εναλλακτικής Ροής 1 επιλέξει "Άκυρο" το σύστημα επιστρέφει στην αρχική σελίδα της εφαρμογής.}
{
    1 & Ο χρήστης πατάει το κουμπί "Άκυρο" & Το σύστημα επιστρέφει στην αρχική σελίδα της εφαρμογής.
}
{Το σενάριο χρήσης τερματίζεται.}
{Σενάριο χρήσης 10 - Εναλλακτική ροή 2}
{\label{table:3-6-use-case-create-community-alternate-flow-2}}

