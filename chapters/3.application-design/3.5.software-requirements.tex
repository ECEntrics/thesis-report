\section{Απαιτήσεις λογισμικού} \label{section:3-5-software-requirements}

Στην παρούσα ενότητα περιγράφονται οι βασικές απαιτήσεις λογισμικού ( \textenglish{software requirements}) της εφαρμογής.

Η πρώτη κατηγορία είναι αυτή των Λειτουργικών Απαιτήσεων (ΛΑ), η οποία αναφέρεται στη συμπεριφορά του συστήματος, δηλαδή στον τρόπο που θα αντιδρά και στις εξόδους που θα παράγει ανάλογα με τις εισόδους.

\begin{enumerate}[label=\textbf{<ΛΑ-\arabic*>}, leftmargin=\parindent, align=left, labelwidth=\parindent, labelsep=0pt]
	\sysReqItem
		{\label{srs:functional-srs-sign-up}}
		{Ο χρήστης πρέπει να μπορεί να εγγραφεί στην εφαρμογή με τον Ethereum λογαριασμό του.}
		{Ο χρήστης πρέπει να μπορεί να εγγραφεί στην εφαρμογή, πατώντας το κουμπί "Sign Up" και συμπληρώνοντας τα απαραίτητα πεδία σύμφωνα με τις οδηγίες. Το πεδίο "Username" είναι υποχρεωτικό να συμπληρωθεί με το επιθυμητό username, το οποίο ορίζεται με μοναδικό τρόπο. Σε περίπτωση που ο χρήστης εισάγει μη διαθέσιμο Username, το σύστημα θα πρέπει να μην επιτρέπει στον χρήστη να συνεχίσει και να προβάλει αντίστοιχο μήνυμα λάθους. Επιπλέον, υπάρχουν τα προαιρετικά πεδία "Profile picture URL" και "Location", στα οποία ο χρήστης μπορεί να εισάγει μία εικόνα προφίλ και την τοποθεσία του αντίστοιχα.}
		{5}{Η απαίτηση είναι ύψιστης προτεραιότητας για τους επισκέπτες, καθώς μόνο μέσω της εγγραφής μπορούν να χρησιμοποιήσουν τα υπόλοιπα χαρακτηριστικά της εφαρμογής (όπως φαίνεται στον πίνακα \ref{table:3-4-user-category-permissions}).}
		{5}{Η απαίτηση είναι ύψιστης σημασίας για το σύστημα, επειδή επηρεάζει τη λειτουργικότητά του.}

	\sysReqItem
		{\label{srs:functional-srs-sign-in}}
		{Ο χρήστης πρέπει να μπορεί συνδέεται στην εφαρμογή, εφόσον είναι εγγεγραμμένος.}
		{Το σύστημα πρέπει να διαπιστώνει αυτόματα εάν το τρέχον Ethereum address έχει λογαριασμό στην εφαρμογή και εάν ναι, να συνδέει να τον χρήστη, ανακτώντας το Username του από το blockchain και προβάλλοντας το στο μενού.}
		{5}{Αυτή η απαίτηση είναι ύψιστης προτεραιότητας για τους χρήστες, καθώς μέσω της σύνδεσης ενεργοποιούνται τα χαρακτηριστικά της δημιουργίας θεμάτων και δημοσίευσης μηνυμάτων.}
		{5}{Η απαίτηση είναι ύψιστης σημασίας για το σύστημα, επειδή επηρεάζει τη λειτουργικότητά του.}

	\sysReqItem
		{\label{srs:functional-srs-create-user-databases}}
		{Το σύστημα πρέπει να δημιουργεί τις απαραίτητες βάσεις δεδομένων και να τις συγχρονίζει με το δίκτυο.}
		{Το σύστημα πρέπει να δημιουργεί τις απαραίτητες OrbitDB βάσεις δεδομένων, εάν αυτές δεν υπάρχουν ήδη τοπικά. Όταν ο χρήστης ξεκλειδώνει τον Ethereum λογαριασμό του, το σύστημα θα πρέπει να τον προτρέπει να υπογράψει με το ιδιωτικό του κλειδί τη συναλλαγή δημιουργίας της OrbitDB Identity του. Αυτή θα εξασφαλίζει τη γνησιότητα των βάσεών του και των δεδομένων που εκείνες θα περιέχουν. Επιπλέον, τοπικές βάσεις δεδομένων θα πρέπει να συγχρονίζονται με τις βάσεις άλλων IPFS κόμβων και να διατηρούνται ενημερωμένες.}
		{5}{Η απαίτηση αυτή είναι ύψιστης σημασίας για τους χρήστες, καθώς η πλειοψηφία των δεδομένων της εφαρμογής διατηρούνται σε αυτές τις βάσεις.}
		{5}{Η παρούσα απαίτηση είναι ύψιστης σημασίας για το σύστημα, καθώς οι περισσότερες θεμελιώδεις λειτουργίες της εφαρμογής προϋποθέτουν την αποθήκευση δεδομένων σε OrbitDB βάσεις.}

	\sysReqItem
		{\label{srs:functional-srs-create-topic}}
		{Ο εγγεγραμμένος χρήστης πρέπει να μπορεί να δημιουργεί θέματα (topics).}
		{Ο εγγεγραμμένος χρήστης πρέπει να μπορεί να δημιουργεί νέα θέματα. Αυτό το επιτυγχάνει πατώντας το κουμπί "New Topic", συμπληρώνοντας τα υποχρεωτικά πεδία της φόρμας ("Topic subject" και "First post content"), πατώντας το κουμπί "Create Topic" και επιβεβαιώνοντας τη συναλλαγή στο Ethereum.}
		{5}{Αυτή η απαίτηση είναι υψηλής σημασίας καθώς επιτελεί έναν από τους βασικούς στόχους της πλατφόρμας.}
		{5}{Η απαίτηση είναι υψηλής σημασίας για τον ίδιο λόγο.}

	\sysReqItem
		{\label{srs:functional-srs-browse-community-topics}}
		{Ο χρήστης πρέπει να μπορεί να περιηγείται στα θέματα μίας κοινότητας.}
		{Το σύστημα πρέπει να μπορεί να προβάλλει τα δημιουργημένα θέματα μίας κοινότητας στην αρχική οθόνη της. Ο χρήστης πρέπει να μπορεί να περιηγείται σε αυτά πατώντας πάνω τους και, έπειτα, χρησιμοποιώντας τα βέλη πλοήγησης, να περιηγηθεί στο ιστορικό των μηνυμάτων του θέματος.}
		{5}{Η απαίτηση αυτή είναι υψηλής σημασίας, αφού επιτρέπει στους επισκέπτες να έχουν πρόσβαση στο δημοσιευμένο υλικό της πλατφόρμας.}
		{5}{Πρόκετια για απαίτηση υψηλής σημασίας, επειδή αποτελεί βασικό χαρακτηριστικό για τη χρηστικότητα της πλατφόρμας.}

	\sysReqItem
		{\label{srs:functional-srs-create-post}}
		{Ο εγγεγραμμένος χρήστης πρέπει να μπορεί να δημιουργεί μηνύματα (posts).}
		{Ο εγγεγραμμένος χρήστης πρέπει να μπορεί να δημιουργεί μηνύματα στο θέμα που επιθυμεί. Αυτό επιτυγχάνεται συμπληρώνοντας το πεδίο νέου μηνύματος στην οθόνη του θέματος, πατώντας το κουμπί "Post" και επιβεβαιώνοντας τη συναλλαγή στο Ethereum.}
		{5}{Αυτή η απαίτηση είναι ύψιστης σημασίας για τους χρήστες, επειδή αποτελεί ένα από τα βασικότερα χαρακτηριστικά της πλατφόρμας.}
		{5}{Η απαίτηση αυτή είναι ύψιστης σημασίας για το σύστημα, καθώς αποτελεί θεμελιώδες κομμάτι για την επίτευξη του βασικότερου στόχου της, δηλαδή της δημιουργίας διαλόγου.}

	\sysReqItem
		{\label{srs:functional-srs-modify-post}}
		{Ο χρήστης πρέπει να μπορεί να τροποποιεί τα μηνύματά του.}
		{Ο χρήστης πρέπει να μπορεί να τροποποιεί τα μηνύματά του. Αυτό το επιτυγχάνει επιλέγοντας το κουμπί επεξεργασίας στο εκάστοτε μήνυμα, πραγματοποιώντας τις επιθυμητές τροποποιήσεις και πατώντας το κουμπί επιβεβαίωσης. Στη συνέχεια, το σύστημα τροποποιεί το περιεχόμενο του μηνύματος στη βάση δεδομένων του χρήστη. Σε περίπτωση που ο χρήστης αλλάξει γνώμη κατά τη διάρκεια της διαδικασίας της επεξεργασίας, μπορεί να πατήσει το κουμπί ακύρωσης και να αναιρέσει τις αλλαγές που πραγματοποίησε.}
		{4}{Η απαίτηση αυτή αποτελεί σημαντικό χαρακτηριστικό, καθώς παρέχει στους χρήστες άμεσο έλεγχο επί των δεδομένων τους.}
		{3}{Αυτή η απαίτηση είναι μέτριας σημαντικότητας για το σύστημα, επειδή αυτό θα μπορούσε να είναι λειτουργικό χωρίς το χαρακτηριστικό της επεξεργασίας μηνυμάτων.}

	\sysReqItem
		{\label{srs:functional-srs-vote-posts}}
		{Ο εγγεγραμμένος χρήστης πρέπει να μπορεί να ψηφίζει σε μηνύματα άλλων χρηστών.}
		{Ο εγγεγραμμένος χρήστης πρέπει να μπορεί να υπερψηφίζει ή να καταψηφίζει μηνύματα άλλων χρηστών. Αυτό το επιτυγχάνει πατώντας τα παρακείμενα κουμπιά "+" ή "-" αντίστοιχα και επιβεβαιώνοντας τη συναλλαγή στο Ethereum (οι ψήφοι αποθηκεύονται εκεί). Η διαδικασία ισχύει και για την τροποποίηση ή την αφαίρεση μίας ψήφου από τον χρήστη.}
		{3}{Η παρούσα απαίτηση είναι μέτριας σημασίας για τους χρήστες, καθώς αποτελεί ένα χρήσιμο αλλά όχι απαραίτητο χαρακτηριστικό.}
		{2}{Η απαίτηση είναι χαμηλής σημασίας για τη λειτουργικότητα του συστήματος. Ωστόσο, τα δημιουργημένα δεδομένα μπορεί να είναι χρήσιμα σε μελλοντική επέκταση της εφαρμογής (π.χ. για τον υπολογισμό της εμπιστοσύνης των χρηστών).}

	\sysReqItem
		{\label{srs:functional-srs-create-polls}}
		{Ο εγγεγραμμένος χρήστης πρέπει να μπορεί να δημιουργεί ψηφοφορίες (polls).}
		{Ο εγγεγραμμένος χρήστης πρέπει να μπορεί να δημιουργεί ψηφοφορίες στις κοινότητες που του το επιτρέπουν. Αυτό το επιτυγχάνει πατώντας "Add Poll" στην οθόνη δημιουργία θέματος και συμπληρώνοντας τα απαραίτητα πεδία.}
		{5}{Η απαίτηση είναι ύψιστης σημασίας για τους χρήστες, καθώς οι δημοκρατικές διαδικασίες αποτελούν μία από τις κύριες χρήσεις της πλατφόρμας.}
		{5}{Η απαίτηση είναι ύψιστης σημασίας για το σύστημα, επειδή αποτελεί βασική προδιαγραφή του.}

	\sysReqItem
		{\label{srs:functional-srs-vote-polls}}
		{Ο εγγεγραμμένος χρήστης πρέπει να μπορεί να ψηφίζει σε ψηφοφορίες.}
		{Ο εγγεγραμμένος χρήστης πρέπει να μπορεί να ψηφίζει σε ψηφοφορίες, σύμφωνα με τους εκάστοτε κανόνες της. Σε κοινότητες που το απαιτούν, ο χρήστης θα πρέπει να διαθέτει το αντίστοιχο voting token για να έχει το δικαίωμα ψήφου.}
		{5}{Η απαίτηση είναι ύψιστης σημασίας για τους χρήστες, καθώς οι δημοκρατικές διαδικασίες αποτελούν μία από τις κύριες χρήσεις της πλατφόρμας.}
		{5}{Η απαίτηση είναι ύψιστης σημασίας για το σύστημα, επειδή αποτελεί βασική προδιαγραφή του.}

	\sysReqItem
		{\label{srs:functional-srs-delete-local-data}}
		{Ο χρήστης πρέπει να μπορεί να διαγράφει τα τοπικά δεδομένα.}
		{Ο χρήστης πρέπει να μπορεί να διαγράφει τα τοπικά δεδομένα. Αυτό το επιτυγχάνει πατώντας στο κουμπί "Clear databases" του μενού και επιβεβαιώνοντας τη διαγραφή μέσω ενός pop-up διαλόγου.}
		{2}{Η απαίτηση αυτή είναι χαμηλής σημασία για τους χρήστες, διότι αποτελεί απλά μία διευκόλυνση για τη διαγραφή των δεδομένων που έχουν αποθηκεύσει τοπικά.}
		{2}{Η απαίτηση αυτή είναι χαμηλής σημασίας για το σύστημα.}

	\sysReqItem
		{\label{srs:functional-srs-create-communities}}
		{Ο χρήστης πρέπει να μπορεί να δημιουργεί κοινότητες.}
		{Ο χρήστης πρέπει να μπορεί να δημιουργεί κοινότητες, πατώντας το κουμπί "Create community" και συμπληρώνοντας τα απαραίτητα πεδία.}
		{4}{Η απαίτηση είναι μεγάλης σημασίας για τους χρήστες, καθώς παρέχει την ευελιξία της δημιουργίας κοινοτήτων.}
		{4}{Πρόκειται για απαίτηση μεγάλης σημασίας για την πλατφόρμα, επειδή έτσι γενικεύει τη χρήση της σε περισσότερες κοινότητες, προσελκύοντας μεγαλύτερο αριθμό χρηστών.}

	\sysReqItem
		{\label{srs:functional-srs-assign-community-contract}}
		{Κατά τη δημιουργία κοινότητας, ο χρήστης πρέπει να έχει τη δυνατότητα να ορίσει ένα contract που θα παρέχει προσαρμοσμένα tokens για αυτήν.}
		{Κατά τη δημιουργία κοινότητας, ο χρήστης πρέπει να έχει τη δυνατότητα να ορίσει ένα contract που θα παρέχει προσαρμοσμένα tokens για αυτήν. Τα tokens αυτά θα διαμοιράζονται με τον τρόπο που επιθυμεί η κοινότητα και θα είναι εκείνα τα οποία θα καθορίζουν τους έγκυρους ψηφοφόρους της.}
		{4}{Αυτή η απαίτηση είναι μεγάλης σημασίας, καθώς παρέχει στις κοινότητες τη δυνατότητα διενέργειας επιβεβαιώσιμων ανώνυμων ψηφοφοριών.}
		{4}{Η απαίτηση είναι μεγάλης σημασίας για το σύστημα, διότι παρέχει στις κοινότητες την απαιτούμενη αυτονομία στον ορισμό των δημοκρατικών διαδικασιών τους.}
\end{enumerate}

Η δεύτερη κατηγορία είναι αυτή των Μη Λειτουργικών Απαιτήσεων (ΜΛΑ). Περιλαμβάνει απαιτήσεις αρχιτεκτονικής σημασίας, οι οποίες καθορίζουν κριτήρια ή περιορισμούς του τρόπου λειτουργίας του συστήματος και σχετίζονται με χαρακτηριστικά όπως η αποδοτικότητα, η αξιοπιστία και η ευχρηστία του.

\begin{enumerate}[label=\textbf{<ΜΛΑ-\arabic*>}, leftmargin=\parindent, align=left, labelwidth=\parindent, labelsep=0pt]
	\sysReqItem
		{\label{srs:non-functional-srs-maximum-decentraliztion}}
		{Η πλατφόρμα πρέπει να είναι κατά το δυνατόν αρχιτεκτονικά αποκεντρωμένη.}
		{Οι τεχνολογίες στις οποίες βασίζεται η πλατφόρμα πρέπει ιδανικά να μη δημιουργούν κεντρικά σημεία. Επιπλέον, ο κώδικας και η δημόσια διάθεση του πρέπει να γίνονται με αποκεντρωμένο τρόπο.}
		{5}{Η αρχιτεκτονική αποκέντρωση της πλατφόρμας αποτελεί απαίτηση ύψιστης προτεραιότητας για τον χρήστη, καθώς διασφαλίζει την πολιτική αποκέντρωση και, έτσι, τους κύριους στόχους που έχουν οριστεί.}
		{5}{Η αρχιτεκτονική αποκέντρωση της πλατφόρμας αποτελεί, απαίτηση ύψιστης σημασίας για το σύστημα, καθώς καθιστά το ίδιο ασφαλές σε επιθέσεις και τα δεδομένα μόνιμα διαθέσιμα στους χρήστες.}

	\sysReqItem
		{\label{srs:non-functional-srs-minimize-fees}}
		{Τα fees για τη χρήση του Ethereum blockchain πρέπει να ελαχιστοποιούνται.}
		{Τα τέλη συναλλαγών που πρέπει να καταβάλλονται για τη χρήση του Ethereum blockchain εξαρτώνται άμεσα τόσο από τον όγκο των δεδομένων προς αποθήκευση, όσο και από τους κύκλους επεξεργασίας των smart contracts της εφαρμογής. Ως προς τα δεδομένα, οι προγραμματιστές θα πρέπει να μεριμνούν ώστε ο κύριος όγκος τους να αποθηκεύεται επί του IPFS, ενώ επί του blockchain να αποθηκεύονται μόνο όσα πραγματικά χρειάζονται. Ως προς την απαιτούμενη επεξεργαστική ισχύ, πρέπει να βελτιστοποιείται ο κώδικας των smart contracts, έτσι ώστε οι διάφορες λειτουργίες τους να εκτελούνται με τους λιγότερους δυνατούς επεξεργαστικούς κύκλους.}
		{4}{Η απαίτηση αυτή είναι μεγάλης σημασίας για τους χρήστες καθώς ναι μεν δεν είναι απαραίτητη για τη χρήση της αλλά είναι ιδιαίτερα σημαντική για την ένταξη χρηστών με χαμηλότερες οικονομικές δυνατότητες.}
		{5}{Η απαίτηση αυτή είναι μεγάλης σημασίας για το σύστημα διότι αποτελεί σημαντικό παράγοντα που επιδρά στην προσέλκυση και τη διατήρηση ενεργών χρηστών.}

	\sysReqItem
		{\label{srs:non-functional-srs-upgrade-contracts}}
		{Τα contracts της εφαρμογής πρέπει να είναι αναβαθμίσιμα.}
		{Τα contracts της εφαρμογής πρέπει μπορούν να αναβαθμιστούν, έτσι ώστε να μπορούν να προστίθενται λειτουργίες και να διορθώνονται σφάλματα. Η αναβαθμισιμότητά τους θα πρέπει να επιτυγχάνεται με μεθόδους που να μην υπονομεύουν τη λειτουργικότητα των  προηγούμενων εκδόσεων.}
		{2}{Η απαίτηση αυτή είναι χαμηλής σημασίας για τους χρήστες, καθώς αφορά την ανάπτυξη και όχι τη χρήση της.}
		{5}{Η απαίτηση αυτή είναι υψηλής σημασίας για το σύστημα, επειδή προσφέρει τη δυνατότητα αποσφαλμάτωσης του, καθώς και την υλοποίηση νέων χαρακτηριστικών.}
\end{enumerate}
