\section{Προδιαγραφή μεθόδου υλοποίησης και χρονοπρογραμματισμός} \label{section:3-6-implementation-methodology-specification}

\subsection{Προδιαγραφή κύκλων}

Εποπτικά, η διαδικασία της υλοποίησης περιγράφεται ως εξής:

% TODO: insert diagram

\subsection{Πρώτη φάση}

% Παλιό από Drive
Στήνεται ένα Ethereum Private Network ως βάση πάνω στην οποία θα δουλέψουμε. Πάνω σε αυτό γράφουμε τα contracts που θα είναι υπεύθυνα για διεκπεραίωση ή μη των posts.
Στη συνέχεια αναπτύσσεται ο απαραίτητος κώδικας που υλοποιεί το posting χρησιμοποιώντας τις βιβλιοθήκες που δίνονται από το IPFS για την επικοινωνία μεταξύ των κόμβων του δικτύου και αυτές που δίνονται από τη BigChainDB για την αποθήκευση των πληροφοριών με διανεμημένο τρόπο.
Γίνονται δοκιμές για την εξακρίβωση της σωστής λειτουργίας του αποτελέσματος και διορθώνονται τυχόν λάθη στο κώδικα.

\subsection{Δεύτερη φάση}

% Παλιό από Drive
Υλοποιείται το δικαίωμα ψήφου και posting χωρίς fees. Αυτό γίνεται μέσω δύο contracts που θα δημιουργούν δύο διαφορετικά tokens (voting token, feeless token) και θα τα αποδίδουν στον εκάστοτε χρήστη που πρέπει να πάρει το δικαίωμα.
Αναπτύσσεται κώδικας που να υλοποιεί τη διαδικασία ψηφοφορίας.
Γίνονται δοκιμές για την εξακρίβωση της σωστής λειτουργίας του αποτελέσματος και διορθώνονται τυχόν λάθη στο κώδικα. Σε αυτή τη φάση η απόδοση των tokens θα γίνει χειροκίνητα για το σκοπό της δοκιμής.

\subsection{Τρίτη φάση}

% Παλιό από Drive
Υλοποιείται ένα σύστημα απόδοσης εμπιστοσύνης (ΣΑΠ).
Αναπτύσσονται τα contracts που είναι απαραίτητα για τη λειτουργία του ΣΑΠ καθώς και για την αυτόματη απόδοση feeless token στους trusted χρήστες.
Γίνονται δοκιμές για την εξακρίβωση της σωστής λειτουργίας του αποτελέσματος και διορθώνονται τυχόν λάθη στο κώδικα.
Εφόσον η εφαρμογή περάσει το στάδιο των δοκιμών είναι έτοιμη για alpha deployment, είναι δηλαδή έτοιμη για χρήση από το κοινό, υπολείπονται όμως χαρακτηριστικά που είναι ιδιαίτερα θεμιτά αλλά όχι απαραίτητα για τη λειτουργία.

\subsection{Τέταρτη φάση}

% Παλιό από Drive
Αναπτύσσεται ο κώδικας του (μοναδικού) συγκεντρωτικού τμήματος του συστήματος το οποίο ανήκει στο δεύτερο κομμάτι - του UAS: Έτσι αυτοματοποιείται η διαδικασία απόδοσης των token, που στην προηγούμενη φάση έγινε χειροκίνητα.
Γίνονται δοκιμές για την εξακρίβωση της σωστής λειτουργίας του αποτελέσματος και διορθώνονται τυχόν λάθη στο κώδικα.
Εφόσον η εφαρμογή περάσει το στάδιο των δοκιμών είναι έτοιμη για ένα beta deployment, ώστε να γίνει πιο ευρύς έλεγχος από μία ομάδα δοκιμών και να παρθεί feedback για την εμπειρία χρήστη.

Για το τελικό deployment θα μπορούσε να τεθεί ως στόχος η κατά το δυνατόν μείωση των τελών για τη λειτουργία της πλατφόρμας, ανεπτυγμένα χαρακτηριστικά επικοινωνίας όπως δόμηση των συζητήσεων σε κατηγορίες, προφίλ χρηστών και άλλα χαρακτηριστικά ευκολίας χρήσης.
