\section{Μεθοδολογία σχεδίασης} \label{section:3-3-design-methodology}

Στον χώρο της τεχνολογίας λογισμικού υπάρχουν διάφορες μεθοδολογίες σχεδίασης, οι οποίες έχουν μεταξύ τους κοινά στοιχεία. Αυτό καθιστά δύσκολο τον προσδιορισμό μίας μόνο μεθοδολογίας η οποία να ακολουθείται πιστά σε κάθε έργο. Συνήθως, οι ομάδες που αναπτύσσουν το λογισμικό ακολουθούν μία μίξη από διάφορα εργαλεία, όπου αυτά κρίνονται ευνοϊκά για τους στόχους της ομάδας. % todo: need reference for this

Κατά τη σχεδίαση και την υλοποίηση του κώδικα ακολουθήθηκαν διάφορες τεχνικές και μοτίβα ανάπτυξης. Κατά βάση χρησιμοποιήθηκαν Agile μέθοδοι, όπως το Kanban και το Scrum και, αργότερα στην ανάπτυξη, το DevOps μοντέλο για διαρκή ενσωμάτωση (Continuous Integration) και διαρκή εγκατάσταση (Continuous Deployment).

Για την παρούσα εργασία, πραγματοποιήθηκε ανάλυση και σχεδιασμός των επιμέρους μονάδων εργασίας (tasks) πριν την έναρξη της διαδικασίας ανάπτυξης του κώδικα. Τα task που προδιαγράφηκαν ήταν συνήθως epic\footnote{Τα epic είναι μεγάλες μονάδες εργασίας, οι οποίες αφορούν σε κάποιο βασικό χαρακτηριστικό. Ο διαχωρισμός τους σε επιμέρους task αναβάλλεται με σκοπό την καλύτερη κατανόηση των αναγκών τους.} τα οποία αργότερα χωρίστηκαν σε επιμέρους, μικρότερα task. Ορίστηκαν επίσης ορόσημα (milestones) τα οποία βοήθησαν ιδιαίτερα στην ιεράρχηση και προτεραιοποίηση των task.

Το Kanban είναι μία μέθοδος οργάνωσης έργων και οπτικοποίησης των μονάδων εργασίας που απαιτούνται για την ολοκλήρωσή τους. Στο Kanban ορίζονται τα βασικά στάδια της ροής ενός task και χρησιμοποιούνται οπτικά μέσα, ώστε να γίνει ιχνηλάτηση τόσο της συνολικής κατάστασης του έργου, όσο και συγκεκριμένων-μεμονωμένων task καθώς αυτά προοδεύουν. Για κάθε στάδιο ολοκλήρωσης ορίζεται μία ξεχωριστή ουρά εργασιών (στήλη), για παράδειγμα "σε αναμονή", "σε εξέλιξη", "ολοκληρωμένο". Χρησιμοποιούνται οπτικά σινιάλα (χρώματα, tags και άλλα) για τον διαχωρισμό και τη γρήγορη κατανόηση των σημαντικότερων γνωρισμάτων των task, για παράδειγμα ξεχωριστό tag για κάθε υπηρεσία στην οποία αναφέρεται το task. Επίσης, ορίζονται όρια στον αριθμό των task που μπορούν να είναι ταυτόχρονα σε εξέλιξη.

Μία άλλη Agile μέθοδος είναι το Scrum. Το Scrum χρησιμοποιεί και επεκτείνει το Kanban. Η βασικές διαφορές του με το Kanban είναι ότι στο Scrum υπάρχουν πιο αυστηρές διαδικασίες. Ορίζονται προγραμματιστικοί κύκλοι (Sprints), οι οποίοι έχουν συγκεκριμένες ημερομηνίες έναρξης και λήξης και συγκεκριμένους στόχους, οι οποίοι αντικατοπτρίζονται σε στόχους ολοκλήρωσης ορισμένων task. Οι ρόλοι είναι σαφέστεροι, με κάθε μέλος της ομάδας να αναλαμβάνει διαφορετικές ευθύνες στην οργάνωση και εκτέλεση. Για τη διαδικασία ανάπτυξης, αποδείχθηκε πολύ χρήσιμη η χρήση του Scrum σε περιόδους που ήταν αναγκαία η ταχύτατη ανάπτυξη καίριων μερών του συστήματος. Αυτό λόγω της αυστηρότητας που επιβάλλει, ειδικά σε ό,τι αφορά στις προθεσμίες ολοκλήρωσης τόσο των επιμέρους task, όσο και του συνολικού Sprint.

Καθώς η αναπτυξιακή διαδικασία ωριμάζει και η πλατφόρμα μετατρέπεται σε βιώσιμο προϊόν, είναι χρήσιμη η ύπαρξη ενός συστήματος που να διευκολύνει τη δημιουργία και τη δημοσίευση καινούργιων εκδόσεων. Μερικές εξαιρετικές μέθοδοι για την απρόσκοπτη και αυτοματοποιημένη επίτευξη αυτού του στόχου ορίζονται από το DevOps. Με τον όρο DevOps (development operations) αναφέρεται μία κουλτούρα σχεδίασης και ανάπτυξης λογισμικού που ορίζει τους ρόλους, τις διαδικασίες και τις τεχνολογίες της, με σκοπό τη συνεχή δημιουργία αξίας για τους χρήστες. Το DevOps έχει πολύ στενή σχέση με το Agile και αποτελεί τη συνέχιση αυτής της νοοτροπίας στον χώρο.

Μία από τις πιο χρήσιμες πτυχές του DevOps, η οποία χρησιμοποιήθηκε στην διπλωματική, είναι το CI/CD, το οποίο περιγράφει τις διαδικασίες αυτοματοποίησης των εργασιών ενσωμάτωσης (integration), ελέγχου (testing), παράδοσης (delivery) και εγκατάστασης (deployment) του προϊόντος. Μέσω του CI/CD αφαιρείται η ανάγκη ανθρώπινης αλληλεπίδρασης για την ολοκλήρωση αυτών των σταδίων, ενώ επιτυγχάνεται η διαρκής και απρόσκοπτη διάθεση της τελευταίες έκδοσης της πλατφόρμας στους χρήστες. Ορίζονται επίσης διαδικασίες δημιουργίας περιβάλλοντος ελέγχου (staging deploys), κάτι που αποτελεί σημαντικό βοήθημα στον έγκαιρο εντοπισμό λαθών του κώδικα.
