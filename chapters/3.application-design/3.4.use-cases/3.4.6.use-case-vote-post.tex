% ===== =====
% Use case 6
% ===== =====
\subsection{Σενάριο χρήσης 6: Ψήφιση μηνύματος} \label{subsection:3-4-use-case-vote-post}

\useCaseTable
{Ψηφίζω σε μήνυμα}
{Στόχος του σεναρίου χρήσης είναι ο χρήστης να μπορεί να υπερψηφίσει ή καταψηφίσει ένα μήνυμα.}
{TODO}
{TODO}
{Ο επισκέπτης πατάει το κουμπί υπερψήφισης ή καταψήφισης.}
{Ο χρήστης να έχει συνδεθεί στην εφαρμογή και να βρίσκεται στην σελίδα ενός θέματος το οποίο περιλαμβάνει τουλάχιστον ένα μήνυμα το οποίο δεν έχει δημιουργήσει ο ίδιος.}
{Σενάριο χρήσης 6, ψήφιση μηνύματος.}
{\label{table:3-4-use-case-vote-post}}

\useCaseBaseFlowTable
{
    1 & Ο χρήστης πατάει στο κουμπί υπερψήφισης μηνύματος. & Το σύστημα εισάγει νέα ψήφο μηνύματος στο blockchain. \\ [0.5ex]
}
{Το σύστημα ανανεώνει τις ψήφους του μηνύματος.}
{Σενάριο χρήσης 6 - Βασική ροή}
{\label{table:3-4-use-case-vote-post-base-flow}}

\begin{figure}[H]
    \centering
    \input{tikz/chapter-3/3-4-use-case-vote-post-sequence-diagram}
    \caption{Σενάριο χρήσης 6 - Διάγραμμα βασικής ροής}
    \label{figure:3-4-use-case-vote-post-base-flow-sequence-diagram}
\end{figure}
