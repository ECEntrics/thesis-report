% ===== =====
% Use case 4
% ===== =====
\subsection{Σενάριο χρήσης 4: Δημιουργία νέου μηνύματος} \label{subsection:3-4-use-case-create-post}

\useCaseTable
{Δημιουργώ νέο μήνυμα}
{Στόχος του σεναρίου χρήσης είναι ο χρήστης να μπορεί να δημιουργήσει νέο μήνυμα.}
{TODO}
{TODO}
{Ο χρήστης πατάει το κουμπί δημιουργίας νέου μηνύματος.}
{Ο χρήστης να έχει συνδεθεί στην εφαρμογή και να βρίσκεται στην σελίδα ενός θέματος.}
{Σενάριο χρήσης 4, δημιουργία νέου μηνύματος.}
{3-4-use-case-create-post}

\useCaseBaseFlowTable
{
    1 & Ο χρήστης πατάει το κουμπί δημιουργίας νέου μηνύματος.           & Το σύστημα εμφανίζει την φόρμα ``Δημιουργία Μηνύματος''. \\ [0.5ex]
    \midrule
    2 & Ο χρήστης συμπληρώνει τα πεδία και πατάει το κουμπί ``Υποβολή''. & Το σύστημα εισάγει νέο μήνυμα στο blockchain. \\ [0.5ex]
    \midrule
    3 & -                                                                & Το σύστημα εισάγει τις πληροφορίες του μηνύματος στην προσωπική βάση OrbitDb του χρήστη. \\ [0.5ex]
}
{Το σύστημα παραμένει στη σελίδα του θέματος εμφανίζοντας το νέο μήνυμα.}
{Σενάριο χρήσης 4 - Βασική ροή}
{3-4-use-case-create-post-base-flow}

\useCaseAlternateFlowTable
{1}
{Ο χρήστης πατάει το κουμπί ``Άκυρο''.}
{Εφόσον ο χρήστης στη γραμμή 2 της Βασικής Ροής επιλέξει ``Άκυρο'' το σύστημα επιστρέφει στη σελίδα του θέματος.}
{
    1 & Ο χρήστης πατάει το κουμπί ``Άκυρο'' & Το σύστημα επιστρέφει στη σελίδα του θέματος.
}
{Το σενάριο χρήσης τερματίζεται.}
{Σενάριο χρήσης 4 - Εναλλακτική ροή 1}
{3-4-use-case-create-post-alternate-flow-1}

\begin{figure}[H]
    \centering
    \input{tikz/chapter-3/3-4-use-case-create-post-sequence-diagram}
    \caption{Σενάριο χρήσης 4 - Διάγραμμα βασικής ροής}
    \label{figure:3-4-use-case-create-post-base-flow-sequence-diagram}
\end{figure}
