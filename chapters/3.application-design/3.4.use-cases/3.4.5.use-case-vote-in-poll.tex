% ===== =====
% Use case 5
% ===== =====
\subsection{Σενάριο χρήσης 5: Ψήφιση σε ψηφοφορία} \label{subsection:3-4-use-case-vote-in-poll}

Το σενάριο χρήσης 5, <ΣΧ-5>, περιγράφει τις διαδοχικές ενέργειες που εκτελούνται για την ψήφιση σε μία ψηφοφορία. Στους πίνακες \ref{table:3-4-use-case-vote-in-poll} και \ref{table:3-4-use-case-vote-in-poll-base-flow} παρατίθενται οι βασικές πληροφορίες του <ΣΧ-5> και οι ενέργειες της βασικής ροής αντίστοιχα, ενώ στο σχήμα \ref{figure:3-4-use-case-vote-in-poll-base-flow-sequence-diagram} φαίνεται το διάγραμμα της βασικής ροής.

\useCaseTable
{Ψηφίζω σε ψηφοφορία}
{Στόχος του σεναρίου χρήσης είναι ο χρήστης να μπορεί να ψηφίσει σε μία ψηφοφορία.}
{TODO}
{TODO}
{Ο χρήστης πατάει το κουμπί ψηφοφορίας.}
{Ο χρήστης να έχει συνδεθεί στην εφαρμογή και να βρίσκεται στην σελίδα ενός θέματος το οποίο περιλαμβάνει ψηφοφορία.}
{Σενάριο χρήσης 5, ψήφιση σε ψηφοφορία.}
{\label{table:3-4-use-case-vote-in-poll}}

% ===== Base flow =====

\useCaseBaseFlowTable
{
    1 & Ο χρήστης πατάει το κουμπί της επιλογής που επιθυμεί να ψηφίσει και πατάει το κουμπί ``Υποβολή''. & Το σύστημα εισάγει νέα ψήφο στο blockchain. \\ [0.5ex]
}
{Το σύστημα ανανεώνει τις πληροφορίες της ψηφοφορίας.}
{Σενάριο χρήσης 5 - Βασική ροή}
{\label{table:3-4-use-case-vote-in-poll-base-flow}}

\begin{figure}[H]
    \centering
    \begin{sequencediagram}
    \newthread{actor}{Actor}{}
    \newinst[3]{concordia}{:Concordia}{}
    \newinst[2]{eth}{:Ethereum}{}

    \begin{call}{actor}{Submit poll vote}{concordia}{Updated topic page}

        \begin{call}{concordia}{Add poll vote}{eth}{}
        \end{call}

    \end{call}
\end{sequencediagram}

    \caption{Σενάριο χρήσης 5 - Διάγραμμα βασικής ροής}
    \label{figure:3-4-use-case-vote-in-poll-base-flow-sequence-diagram}
\end{figure}
