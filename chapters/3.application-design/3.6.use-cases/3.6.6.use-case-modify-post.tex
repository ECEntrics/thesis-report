% ===== =====
% Use case 6
% ===== =====
\subsection{Σενάριο χρήσης 6: Τροποποίηση μηνύματος} \label{subsection:3-6-use-case-modify-post}

Το σενάριο χρήσης 6 (<ΣΧ-6>) περιγράφει τις διαδοχικές ενέργειες που εκτελούνται για την τροποποίηση ενός μηνύματος. Στους πίνακες \ref{table:3-6-use-case-modify-post} και \ref{table:3-6-use-case-modify-post-base-flow} παρατίθενται οι βασικές πληροφορίες του <ΣΧ-6> και οι ενέργειες της βασικής ροής αντίστοιχα, ενώ στο σχήμα \ref{figure:3-6-use-case-modify-post-base-flow-sequence-diagram} φαίνεται το διάγραμμα της βασικής ροής.

\useCaseTable
{Τροποποιώ ένα μήνυμα}
{Στόχος του σεναρίου χρήσης είναι ο χρήστης να μπορεί να τροποποιήσει τα μηνύματά του.}
{\ref{srs:functional-srs-modify-post}}
{-}
{Ο χρήστης πατάει το κουμπί τροποποίησης του μηνύματος.}
{Ο χρήστης πρέπει να έχει συνδεθεί στην εφαρμογή και να βρίσκεται στη σελίδα του θέματος που περιέχει το μήνυμά του.}
{Σενάριο χρήσης 6, τροποποίηση μηνύματος}
{\label{table:3-6-use-case-modify-post}}

% ===== Base flow =====

\useCaseBaseFlowTable
{
    1 & Ο χρήστης πατάει το κουμπί τροποποίησης του μηνύματος.           & Το σύστημα εμφανίζει τη φόρμα "Τροποποίηση Μηνύματος". \\ [0.5ex]
    \midrule
    2 & Ο χρήστης συμπληρώνει τα πεδία και πατάει το κουμπί "Υποβολή". & Το σύστημα τροποποιεί τις πληροφορίες του μηνύματος στην προσωπική OrbitDB βάση του χρήστη. \\ [0.5ex]
}
{Το σύστημα παραμένει στη σελίδα του θέματος εμφανίζοντας το τροποποιημένο μήνυμα.}
{Σενάριο χρήσης 6 - Βασική ροή}
{\label{table:3-6-use-case-modify-post-base-flow}}

\begin{figure}[H]
    \centering
    \begin{sequencediagram}
    \newthread{actor}{Actor}{}
    \newinst[4]{concordia}{:Concordia}{}
    \newinst[1]{eth}{:Ethereum}{}
    \newinst[2]{orbit}{:OrbitDb}{}

    \begin{call}{actor}{Modify post}{concordia}{Post modification form}
    \end{call}

    \begin{call}{actor}{Submit}{concordia}{Updated topic page}

        \begin{call}{concordia}{Save modified post information}{orbit}{}
        \end{call}

    \end{call}
\end{sequencediagram}

    \caption{Σενάριο χρήσης 6 - Διάγραμμα βασικής ροής}
    \label{figure:3-6-use-case-modify-post-base-flow-sequence-diagram}
\end{figure}

% ===== Alternate flow =====

Το <ΣΧ-6> περιέχει επίσης μία εναλλακτική ροή που μπορεί να προκύψει βάσει των επιλογών του χρήστη και η οποία περιγράφεται στον πίνακα \ref{table:3-6-use-case-modify-post-alternate-flow-1}.

\useCaseAlternateFlowTable
{1}
{Ο χρήστης πατάει το κουμπί "Άκυρο".}
{Εφόσον ο χρήστης στη γραμμή 2 της Βασικής Ροής επιλέξει "Άκυρο" το σύστημα επιστρέφει στη σελίδα του θέματος.}
{
    1 & Ο χρήστης πατάει το κουμπί "Άκυρο" & Το σύστημα επιστρέφει στη σελίδα του θέματος.
}
{Το σενάριο χρήσης τερματίζεται.}
{Σενάριο χρήσης 6 - Εναλλακτική ροή 1}
{\label{table:3-6-use-case-modify-post-alternate-flow-1}}
