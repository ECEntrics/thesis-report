% ===== =====
% Use case 10
% ===== =====
\subsection{Σενάριο χρήσης 10: Δημιουργία κοινότητας} \label{subsection:3-10-use-case-create-community}

Το σενάριο χρήσης 10, <ΣΧ-10>, περιγράφει τις διαδοχικές ενέργειες που εκτελούνται για την δημιουργία μίας κοινότητας. Στους πίνακες \ref{table:3-6-use-case-create-community} και \ref{table:3-6-use-case-create-community-base-flow} παρατίθενται οι βασικές πληροφορίες του <ΣΧ-10> και οι ενέργειες της βασικής ροής αντίστοιχα, ενώ στο σχήμα \ref{figure:3-6-use-case-create-community-base-flow-sequence-diagram} φαίνεται το διάγραμμα της βασικής ροής.

\useCaseTable
{Δημιουργώ νέα κοινότητα}
{Στόχος του σεναρίου χρήσης είναι ο χρήστης να μπορεί να δημιουργήσει νέα κοινότητα.}
{\ref{srs:functional-srs-create-communities}, \ref{srs:functional-srs-assign-community-contract}}
{\ref{srs:non-functional-srs-minimize-fees}}
{Ο χρήστης πατάει το κουμπί δημιουργίας νέας κοινότητας.}
{Ο χρήστης να έχει συνδεθεί στην εφαρμογή και να βρίσκεται στην αρχική σελίδα.}
{Σενάριο χρήσης 10, δημιουργία νέας κοινότητας.}
{\label{table:3-6-use-case-create-community}}

% ===== Base flow =====

\useCaseBaseFlowTable
{
    1 & Ο χρήστης πατάει το κουμπί δημιουργίας νέας κοινότητας.          & Το σύστημα εμφανίζει την φόρμα ``Δημιουργία Κοινότητας''. \\ [0.5ex]
    \midrule
    2 & Ο χρήστης συμπληρώνει τα πεδία και πατάει το κουμπί ``Υποβολή''. & Το σύστημα δημιουργεί νέα κοινότητα στο blockchain. \\ [0.5ex]
}
{Το σύστημα μεταβαίνει στην σελίδα της νέας κοινότητας.}
{Σενάριο χρήσης 10 - Βασική ροή}
{\label{table:3-6-use-case-create-community-base-flow}}

\begin{figure}[H]
    \centering
    \begin{sequencediagram}
    \newthread{actor}{Actor}{}
    \newinst[3]{concordia}{:Concordia}{}
    \newinst[2]{eth}{:Ethereum}{}
    \newinst[1]{orbit}{:OrbitDb}{}

    \begin{call}{actor}{Create community}{concordia}{Community creation form}
    \end{call}

    \begin{call}{actor}{Submit}{concordia}{Created community page}

        \begin{call}{concordia}{Create community}{eth}{New community ID}
        \end{call}

    \end{call}
\end{sequencediagram}

    \caption{Σενάριο χρήσης 10 - Διάγραμμα βασικής ροής}
    \label{figure:3-6-use-case-create-community-base-flow-sequence-diagram}
\end{figure}

% ===== Alternate flow =====

Το <ΣΧ-10> περιέχει επίσης δύο εναλλακτικές ροές που μπορεί να προκύψουν βάσει των επιλογών του χρήστη και οι οποίες περιγράφονται στους πίνακες \ref{table:3-6-use-case-create-community-alternate-flow-1} και \ref{table:3-6-use-case-create-community-alternate-flow-2}. Η εναλλακτική ροή 1 φαίνεται επίσης στο σχήμα \ref{figure:3-6-use-case-create-community-alternate-flow-1-sequence-diagram} όπου παρουσιάζεται το διάγραμμα ροής της.

\useCaseAlternateFlowTable
{1}
{Ο χρήστης ορίζει εξωτερικό contract για την κοινότητα.}
{Εφόσον ο χρήστης στη γραμμή 2 της Βασικής Ροής επιλέξει ``Προσθήκη Συμβολαίου'' το σύστημα ανανεώνει την σελίδα προσθέτοντας τα επιπλέον πεδία της φόρμας ``Σύνδεση Συμβολαίου''.}
{
    1 & Ο χρήστης, αφού συμπληρώσει τη φόρμα ``Δημιουργία Κοινότητας'', πατάει το κουμπί ``Προσθήκη ψηφοφορίας'' & Το σύστημα ανανεώνει τη σελίδα με τα πεδία της φόρμας ``Σύνδεση Συμβολαίου''. \\ [0.5ex]
    \midrule
    2 & Ο χρήστης συμπληρώνει τα πεδία και πατάει το κουμπί ``Υποβολή''.                                         & Το σύστημα δημιουργεί την νέα κοινότητα στο blockchain και την συνδέει με το εξωτερικό contract. \\ [0.5ex]
}
{Το σύστημα μεταβαίνει στην σελίδα της νέας κοινότητας.}
{Σενάριο χρήσης 10 - Εναλλακτική ροή 1}
{\label{table:3-6-use-case-create-community-alternate-flow-1}}

\begin{figure}[H]
    \centering
    \begin{sequencediagram}
    \newthread{actor}{Actor}{}
    \newinst[4]{concordia}{:Concordia}{}
    \newinst[4]{eth}{:Ethereum}{}
    \newinst{orbit}{:OrbitDb}{}

    \begin{call}{actor}{Create community}{concordia}{Community creation form}
    \end{call}

    \begin{call}{actor}{Add external contract}{concordia}{External contract form}
    \end{call}

    \begin{call}{actor}{Submit}{concordia}{Created community page}

        \begin{call}{concordia}{Create community}{eth}{New community ID}
        \end{call}

        \begin{call}{concordia}{Connect external contract}{eth}{}
        \end{call}
    \end{call}
\end{sequencediagram}

    \caption{Σενάριο χρήσης 3 - Διάγραμμα εναλλακτικής ροής 1}
    \label{figure:3-6-use-case-create-community-alternate-flow-1-sequence-diagram}
\end{figure}

\useCaseAlternateFlowTable
{2}
{Ο χρήστης πατάει το κουμπί ``Άκυρο''.}
{Εφόσον ο χρήστης στη γραμμή 2 της Βασικής Ροής ή στη γραμμή 2 της Εναλλακτικής Ροής 1 επιλέξει ``Άκυρο'' το σύστημα επιστρέφει στην αρχική σελίδα της εφαρμογής.}
{
    1 & Ο χρήστης πατάει το κουμπί ``Άκυρο'' & Το σύστημα επιστρέφει στην αρχική σελίδα της εφαρμογής.
}
{Το σενάριο χρήσης τερματίζεται.}
{Σενάριο χρήσης 10 - Εναλλακτική ροή 2}
{\label{table:3-6-use-case-create-community-alternate-flow-2}}
