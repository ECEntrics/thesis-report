% ===== =====
% Use case 1
% ===== =====
\subsection{Σενάριο χρήσης 2: Σύνδεση χρήστη} \label{subsection:3-6-use-case-signin}

Το σενάριο χρήσης 2 (<ΣΧ-2>) περιγράφει τις διαδοχικές ενέργειες που εκτελούνται για τη σύνδεση ενός χρήστη στο σύστημα. Στους πίνακες \ref{table:3-6-use-case-sign-in} και \ref{table:3-6-use-case-sign-in-base-flow} παρατίθενται οι βασικές πληροφορίες του <ΣΧ-2> και οι ενέργειες της βασικής ροής αντίστοιχα, ενώ στο σχήμα \ref{figure:3-6-use-case-sign-in-base-flow-sequence-diagram} φαίνεται το διάγραμμα της βασικής ροής.

\useCaseTable
{Συνδέομαι στο σύστημα}
{Στόχος του σεναρίου χρήσης είναι ο χρήστης να συνδέεται αυτόματα στο σύστημα.}
{\ref{srs:functional-srs-sign-in}}
{-}
{-}
{Ο χρήστης πρέπει να έχει ανοίξει τη σελίδα της εφαρμογής.}
{Σενάριο χρήσης 2, σύνδεση χρήστη στο σύστημα}
{\label{table:3-6-use-case-sign-in}}

% ===== Base flow =====

\useCaseBaseFlowTable
{
    1 & - & Το σύστημα ανακτά τις πληροφορίες του χρήστη από το blockchain. \\ [0.5ex]
    \midrule
    2 & - & Το σύστημα δημιουργεί τις προσωπικές OrbitDB βάσεις δεδομένων του χρήστη. \\ [0.5ex]
}
{Το σύστημα παραμένει στην αρχική σελίδα της εφαρμογής.}
{Σενάριο χρήσης 2 - Βασική ροή}
{\label{table:3-6-use-case-sign-in-base-flow}}

\begin{figure}[H]
    \centering
    \begin{sequencediagram}
    \newthread{actor}{Actor}{}
    \newinst[2]{concordia}{:Concordia}{}
    \newinst[2]{eth}{:Ethereum}{}
    \newinst[1]{orbit}{:OrbitDb}{}

    \begin{call}{actor}{Fetch page}{concordia}{}

        \begin{call}{concordia}{Get user}{eth}{User information}
        \end{call}

        \begin{call}{concordia}{Create databases}{orbit}{}
        \end{call}

    \end{call}
\end{sequencediagram}

    \caption{Σενάριο χρήσης 2 - Διάγραμμα βασικής ροής}
    \label{figure:3-6-use-case-sign-in-base-flow-sequence-diagram}
\end{figure}
\vspace{\baselineskip}