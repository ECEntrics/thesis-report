% ===== =====
% Use case 8
% ===== =====
\subsection{Σενάριο χρήσης 8: Ψήφιση μηνύματος} \label{subsection:3-6-use-case-vote-post}

Το σενάριο χρήσης 8 (<ΣΧ-8>) περιγράφει τις διαδοχικές ενέργειες που εκτελούνται για την ψήφιση σε ένα μήνυμα. Στους πίνακες \ref{table:3-6-use-case-vote-post} και \ref{table:3-6-use-case-vote-post-base-flow} παρατίθενται οι βασικές πληροφορίες του <ΣΧ-8> και οι ενέργειες της βασικής ροής αντίστοιχα, ενώ στο σχήμα \ref{figure:3-6-use-case-vote-post-base-flow-sequence-diagram} φαίνεται το διάγραμμα της βασικής ροής.

\useCaseTable
{Ψηφίζω σε μήνυμα}
{Στόχος του σεναρίου χρήσης είναι ο χρήστης να μπορεί να υπερψηφίσει ή καταψηφίσει ένα μήνυμα.}
{\ref{srs:functional-srs-vote-posts}}
{\ref{srs:non-functional-srs-minimize-fees}}
{Ο χρήστης πατάει το κουμπί υπερψήφισης ή καταψήφισης.}
{Ο χρήστης πρέπει να έχει συνδεθεί στην εφαρμογή και να βρίσκεται στη σελίδα ενός θέματος το οποίο περιλαμβάνει τουλάχιστον ένα μήνυμα το οποίο δεν έχει δημιουργήσει ο ίδιος.}
{Σενάριο χρήσης 8, ψήφιση μηνύματος}
{\label{table:3-6-use-case-vote-post}}

% ===== Base flow =====

\useCaseBaseFlowTable
{
    1 & Ο χρήστης πατάει στο κουμπί υπερψήφισης μηνύματος. & Το σύστημα εισάγει νέα ψήφο μηνύματος στο blockchain. \\ [0.5ex]
}
{Το σύστημα ανανεώνει τις ψήφους του μηνύματος.}
{Σενάριο χρήσης 8 - Βασική ροή}
{\label{table:3-6-use-case-vote-post-base-flow}}

\begin{figure}[H]
    \centering
    \begin{sequencediagram}
    \newthread{actor}{Actor}{}
    \newinst[3]{concordia}{:Concordia}{}
    \newinst[2]{eth}{:Ethereum}{}

    \begin{call}{actor}{Submit post vote}{concordia}{Updated topic page}

        \begin{call}{concordia}{Add post vote}{eth}{}
        \end{call}

    \end{call}
\end{sequencediagram}

    \caption{Σενάριο χρήσης 8 - Διάγραμμα βασικής ροής}
    \label{figure:3-6-use-case-vote-post-base-flow-sequence-diagram}
\end{figure}
