% ===== =====
% Use case 1
% ===== =====
\subsection{Σενάριο χρήσης 1: Εγγραφή χρήστη} \label{subsection:3-6-use-case-signup}

Το σενάριο χρήσης 1, <ΣΧ-1>, περιγράφει τις διαδοχικές ενέργειες που εκτελούνται για την εγγραφή ενός χρήστη στο σύστημα. Στους πίνακες \ref{table:3-6-use-case-sign-up} και \ref{table:3-6-use-case-sign-up-base-flow} παρατίθενται οι βασικές πληροφορίες του <ΣΧ-1> και οι ενέργειες της βασικής ροής αντίστοιχα, ενώ στο σχήμα \ref{figure:3-6-use-case-sign-up-base-flow-sequence-diagram} φαίνεται το διάγραμμα της βασικής ροής.

\useCaseTable
{Εγγράφομαι στο σύστημα}
{Στόχος του σεναρίου χρήσης είναι ο επισκέπτης να μπορεί να εγγραφεί στο σύστημα ως χρήστης.}
{\ref{srs:functional-srs-sign-up}, \ref{srs:functional-srs-create-user-databases}}
{\ref{srs:non-functional-srs-minimize-fees}}
{Ο επισκέπτης πατάει το κουμπί εγγραφή.}
{Ο επισκέπτης πρέπει να έχει ανοίξει την σελίδα της εφαρμογής.}
{Σενάριο χρήσης 1, εγγραφή χρήστη στο σύστημα.}
{\label{table:3-6-use-case-sign-up}}

% ===== Base flow =====

\useCaseBaseFlowTable
{
    1 & Ο χρήστης πατάει το κουμπί εγγραφή.                                                    & Το σύστημα εμφανίζει την φόρμα ``Εγγραφή Χρήστη''. \\ [0.5ex]
    \midrule
    2 & Ο χρήστης συμπληρώνει τα πεδία και πατάει το κουμπί ``Υποβολή''.                       & Το σύστημα εισάγει νέο χρήστη στο blockchain. \\ [0.5ex]
    \midrule
    3 & -                                                                                      & Το σύστημα δημιουργεί τις προσωπικές βάσεις βάσεις δεδομένων OrbitDb του χρήστη. \\ [0.5ex]
    \midrule
    4 & -                                                                                      & Το σύστημα εμφανίζει την φόρμα ``Πληροφορίες Χρήστη''. \\ [0.5ex]
    \midrule
    5 & Ο χρήστης συμπληρώνει τις προσωπικές του πληροφορίες και πατάει το κουμπί ``Υποβολή''. & Το σύστημα εισάγει τις πληροφορίες χρήστη στην προσωπική του βάση OrbitDb. \\ [0.5ex]
}
{Το σύστημα μεταβαίνει στην αρχική σελίδα της εφαρμογής.}
{Σενάριο χρήσης 1 - Βασική ροή}
{\label{table:3-6-use-case-sign-up-base-flow}}

\begin{figure}[H]
    \centering
    \begin{sequencediagram}
    \newthread{actor}{Actor}{}
    \newinst[4]{concordia}{:Concordia}{}
    \newinst[2]{eth}{:Ethereum}{}
    \newinst[1]{orbit}{:OrbitDB}{}

    \begin{call}{actor}{Sign up}{concordia}{Sign up form}
    \end{call}

    \begin{call}{actor}{Submit}{concordia}{Personal information form}

        \begin{call}{concordia}{Create user}{eth}{New user ID}
        \end{call}

        \begin{call}{concordia}{Create databases}{orbit}{}
        \end{call}

    \end{call}

    \begin{call}{actor}{Submit}{concordia}{Home page}

        \begin{call}{concordia}{Save personal information}{orbit}{}
        \end{call}

    \end{call}
\end{sequencediagram}

    \caption{Σενάριο χρήσης 1 - Διάγραμμα βασικής ροής}
    \label{figure:3-6-use-case-sign-up-base-flow-sequence-diagram}
\end{figure}

% ===== Alternate flows =====

Το <ΣΧ-1> περιέχει επίσης τρεις εναλλακτικές ροές που μπορεί να προκύψουν βάσει των επιλογών του χρήστη και οι οποίες περιγράφονται στους πίνακες \ref{table:3-6-use-case-sign-up-alternate-flow-1}, \ref{table:3-6-use-case-sign-up-alternate-flow-2} και \ref{table:3-6-use-case-sign-up-alternate-flow-3}.

\useCaseAlternateFlowTable
{1}
{Τα στοιχεία χρήστη είναι λανθασμένα.}
{Εφόσον ο χρήστης στη γραμμή 2 δεν συμπληρώσει το πεδίο ονόματος χρήστη ή συμπληρώσει ένα όνομα χρήστη το οποίο είναι ήδη σε χρήση στο σύστημα, το σύστημα πρέπει να επιστρέψει σχετικό μήνυμα σφάλματος.}
{
    1 & - & Το σύστημα εμφανίζει μήνυμα σφάλματος.
}
{Το σύστημα επιστρέφει στη γραμμή 1 της βασικής ροής.}
{Σενάριο χρήσης 1 - Εναλλακτική ροή 1}
{\label{table:3-6-use-case-sign-up-alternate-flow-1}}

\useCaseAlternateFlowTable
{2}
{Ο χρήστης πατάει το κουμπί ``Άκυρο''.}
{Εφόσον ο χρήστης στη γραμμή 2 της Βασικής Ροής επιλέξει ``Άκυρο'' το σύστημα επιστρέφει στην αρχική σελίδα της εφαρμογής.}
{
    1 & Ο χρήστης πατάει το κουμπί ``Άκυρο'' & Το σύστημα επιστρέφει στην αρχική σελίδα της εφαρμογής.
}
{Το σενάριο χρήσης τερματίζεται.}
{Σενάριο χρήσης 1 - Εναλλακτική ροή 2}
{\label{table:3-6-use-case-sign-up-alternate-flow-2}}

\useCaseAlternateFlowTable
{3}
{Ο χρήστης πατάει το κουμπί ``Παράληψη''.}
{Εφόσον ο χρήστης στη γραμμή 5 της Βασικής Ροής επιλέξει ``Παράληψη'' το σύστημα επιστρέφει στην αρχική σελίδα της εφαρμογής.}
{
    1 & Ο χρήστης πατάει το κουμπί ``Παράληψη'' & Το σύστημα επιστρέφει στην αρχική σελίδα της εφαρμογής.
}
{Το σενάριο χρήσης τερματίζεται.}
{Σενάριο χρήσης 1 - Εναλλακτική ροή 3}
{\label{table:3-6-use-case-sign-up-alternate-flow-3}}
