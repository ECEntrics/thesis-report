% ===== =====
% Use case 3
% ===== =====
\subsection{Σενάριο χρήσης 3: Δημιουργία νέου θέματος} \label{subsection:3-6-use-case-create-topic}

Το σενάριο χρήσης 3, <ΣΧ-3>, περιγράφει τις διαδοχικές ενέργειες που εκτελούνται για την δημιουργία ενός θέματος. Στους πίνακες \ref{table:3-6-use-case-create-topic} και \ref{table:3-6-use-case-create-topic-base-flow} παρατίθενται οι βασικές πληροφορίες του <ΣΧ-3> και οι ενέργειες της βασικής ροής αντίστοιχα, ενώ στο σχήμα \ref{figure:3-6-use-case-create-topic-base-flow-sequence-diagram} φαίνεται το διάγραμμα της βασικής ροής.

\useCaseTable
{Δημιουργώ νέο θέμα}
{Στόχος του σεναρίου χρήσης είναι ο χρήστης να μπορεί να δημιουργήσει νέο θέμα.}
{\ref{srs:functional-srs-create-topic}, \ref{srs:functional-srs-create-polls}}
{\ref{srs:non-functional-srs-minimize-fees}}
{Ο χρήστης πατάει το κουμπί δημιουργίας νέου θέματος.}
{Ο χρήστης να έχει συνδεθεί στην εφαρμογή και να βρίσκεται στην αρχική σελίδα.}
{Σενάριο χρήσης 3, δημιουργία νέου θέματος.}
{\label{table:3-6-use-case-create-topic}}

% ===== Base flow =====

\useCaseBaseFlowTable
{
    1 & Ο χρήστης πατάει το κουμπί δημιουργίας νέου θέματος.             & Το σύστημα εμφανίζει την φόρμα "Δημιουργία Θέματος". \\ [0.5ex]
    \midrule
    2 & Ο χρήστης συμπληρώνει τα πεδία και πατάει το κουμπί "Υποβολή". & Το σύστημα εισάγει νέο θέμα στο blockchain. \\ [0.5ex]
    \midrule
    3 & -                                                                & Το σύστημα εισάγει τις πληροφορίες του θέματος στην προσωπική βάση OrbitDb του χρήστη. \\ [0.5ex]
}
{Το σύστημα μεταβαίνει στην σελίδα του νέου θέματος.}
{Σενάριο χρήσης 3 - Βασική ροή}
{\label{table:3-6-use-case-create-topic-base-flow}}

\begin{figure}[H]
    \centering
    \begin{sequencediagram}
    \newthread{actor}{Actor}{}
    \newinst[4]{concordia}{:Concordia}{}
    \newinst[2]{eth}{:Ethereum}{}
    \newinst[1]{orbit}{:OrbitDb}{}

    \begin{call}{actor}{Create topic}{concordia}{Topic creation form}
    \end{call}

    \begin{call}{actor}{Submit}{concordia}{New topic page}

        \begin{call}{concordia}{Create topic}{eth}{New topic ID}
        \end{call}

        \begin{call}{concordia}{Save topic information}{orbit}{}
        \end{call}

    \end{call}
\end{sequencediagram}

    \caption{Σενάριο χρήσης 3 - Διάγραμμα βασικής ροής}
    \label{figure:3-6-use-case-create-topic-base-flow-sequence-diagram}
\end{figure}

% ===== Alternate flow =====

Το <ΣΧ-3> περιέχει επίσης δύο εναλλακτικές ροές που μπορεί να προκύψουν βάσει των επιλογών του χρήστη και οι οποίες περιγράφονται στους πίνακες \ref{table:3-6-use-case-create-topic-alternate-flow-1} και \ref{table:3-6-use-case-create-topic-alternate-flow-2}. Η εναλλακτική ροή 1 φαίνεται επίσης στο σχήμα \ref{figure:3-6-use-case-create-topic-alternate-flow-1-sequence-diagram} όπου παρουσιάζεται το διάγραμμα ροής της.

\useCaseAlternateFlowTable
{1}
{Ο χρήστης δημιουργεί ψηφοφορία.}
{Εφόσον ο χρήστης στη γραμμή 2 της Βασικής Ροής επιλέξει "Προσθήκη Ψηφοφορίας" το σύστημα ανανεώνει την σελίδα προσθέτοντας τα επιπλέον πεδία της φόρμας "Δημιουργία Ψηφοφορίας".}
{
    1 & Ο χρήστης, αφού συμπληρώσει τη φόρμα "Δημιουργία Θέματος", πατάει το κουμπί "Προσθήκη ψηφοφορίας" & Το σύστημα ανανεώνει τη σελίδα με τα πεδία της φόρμας "Δημιουργία Ψηφοφορίας". \\ [0.5ex]
    \midrule
    2 & Ο χρήστης συμπληρώνει τα πεδία και πατάει το κουμπί "Υποβολή".                                      & Το σύστημα εισάγει το νέο θέμα καθώς και τη νέα ψηφοφορία στο blockchain. \\ [0.5ex]
    \midrule
    3 & -                                                                                                     & Το σύστημα εισάγει τις πληροφορίες του θέματος και της ψηφοφορίας στις προσωπικές βάσεις OrbitDb του χρήστη.
}
{Το σύστημα μεταβαίνει στην σελίδα του νέου θέματος.}
{Σενάριο χρήσης 3 - Εναλλακτική ροή 1}
{\label{table:3-6-use-case-create-topic-alternate-flow-1}}

\begin{figure}[H]
    \centering
    \begin{sequencediagram}
    \newthread{actor}{Actor}{}
    \newinst[4]{concordia}{:Concordia}{}
    \newinst[2]{eth}{:Ethereum}{}
    \newinst[1]{orbit}{:OrbitDB}{}

    \begin{call}{actor}{Create topic}{concordia}{Topic creation form}
    \end{call}

    \begin{call}{actor}{Add poll}{concordia}{Poll creation form}
    \end{call}

    \begin{call}{actor}{Submit}{concordia}{New topic-poll page}

        \begin{call}{concordia}{Create topic}{eth}{New topic ID}
        \end{call}

        \begin{call}{concordia}{Add poll to topic}{eth}{}
        \end{call}

        \begin{call}{concordia}{Save topic information}{orbit}{}
        \end{call}

        \begin{call}{concordia}{Save poll information}{orbit}{}
        \end{call}

    \end{call}
\end{sequencediagram}

    \caption{Σενάριο χρήσης 3 - Διάγραμμα εναλλακτικής ροής 1}
    \label{figure:3-6-use-case-create-topic-alternate-flow-1-sequence-diagram}
\end{figure}

\useCaseAlternateFlowTable
{2}
{Ο χρήστης πατάει το κουμπί "Άκυρο".}
{Εφόσον ο χρήστης στη γραμμή 2 της Βασικής Ροής ή στη γραμμή 2 της Εναλλακτικής Ροής 1 επιλέξει "Άκυρο" το σύστημα επιστρέφει στην αρχική σελίδα της εφαρμογής.}
{
    1 & Ο χρήστης πατάει το κουμπί "Άκυρο" & Το σύστημα επιστρέφει στην αρχική σελίδα της εφαρμογής.
}
{Το σενάριο χρήσης τερματίζεται.}
{Σενάριο χρήσης 3 - Εναλλακτική ροή 2}
{\label{table:3-6-use-case-create-topic-alternate-flow-2}}
