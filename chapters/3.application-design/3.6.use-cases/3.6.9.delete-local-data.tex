% ===== =====
% Use case 9
% ===== =====
\subsection{Σενάριο χρήσης 9: Διαγραφή τοπικών δεδομένων} \label{subsection:3-6-use-case-delete-local-data}

Το σενάριο χρήσης 9, <ΣΧ-9>, περιγράφει τις διαδοχικές ενέργειες που εκτελούνται για τη διαγραφή των τοπικών δεδομένων. Στους πίνακες \ref{table:3-6-use-case-delete-local-data} και \ref{table:3-6-use-case-delete-local-data-base-flow} παρατίθενται οι βασικές πληροφορίες του <ΣΧ-9> και οι ενέργειες της βασικής ροής αντίστοιχα, ενώ στο σχήμα \ref{figure:3-6-use-case-delete-local-data-base-flow-sequence-diagram} φαίνεται το διάγραμμα της βασικής ροής.

\useCaseTable
{Διαγράφω τα τοπικά δεδομένα}
{Στόχος του σεναρίου χρήσης είναι ο επισκέπτης να μπορεί να διαγράψει τα τοπικά δεδομένα που αποθηκεύονται στο σύστημά του από την εφαρμογή.}
{\ref{srs:functional-srs-delete-local-data}}
{-}
{Ο επισκέπτης πατάει το κουμπί διαγραφής των τοπικών δεδομένων.}
{Ο επισκέπτης πρέπει να έχει ανοίξει την σελίδα της εφαρμογής.}
{Σενάριο χρήσης 9, διαγραφή τοπικών δεδομένων.}
{\label{table:3-6-use-case-delete-local-data}}

% ===== Base flow =====

\useCaseBaseFlowTable
{
    1 & Ο επισκέπτης πατάει το κουμπί διαγραφής των τοπικών δεδομένων.      & Το σύστημα εμφανίζει την φόρμα ``Επιβεβαίωση Διαγραφής Τοπικών Δεδομένων''. \\ [0.5ex]
    \midrule
    2 & Ο επισκέπτης συμπληρώνει το πεδίο και πατάει το κουμπί ``Υποβολή''. & Το σύστημα διαγράφει όλες τις τοπικές βάσεις OrbitDb που χρησιμοποιούνται από την εφαρμογή. \\ [0.5ex]
}
{Το σύστημα παραμένει πραγματοποιεί ανανέωση της σελίδας.}
{Σενάριο χρήσης 9 - Βασική ροή}
{\label{table:3-6-use-case-delete-local-data-base-flow}}

\begin{figure}[H]
    \centering
    \begin{sequencediagram}
    \newthread{actor}{Actor}{}
    \newinst[4]{concordia}{:Concordia}{}
    \newinst[1]{eth}{:Ethereum}{}
    \newinst[1]{orbit}{:OrbitDB}{}

    \begin{call}{actor}{Delete local data}{concordia}{Delete confirmation form}
    \end{call}

    \begin{call}{actor}{Submit}{concordia}{}

        \begin{call}{concordia}{Delete local DBs}{orbit}{}
        \end{call}

    \end{call}
\end{sequencediagram}

    \caption{Σενάριο χρήσης 9 - Διάγραμμα βασικής ροής}
    \label{figure:3-6-use-case-delete-local-data-base-flow-sequence-diagram}
\end{figure}
