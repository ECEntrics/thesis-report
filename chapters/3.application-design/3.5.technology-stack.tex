\section{Τεχνολογίες} \label{section:3-5-technologies}

\subsection{Ethereum}

Ξεκινώντας την σχεδίαση της πλατφόρμας πραγματοποιήσαμε έρευνα ώστε να ανακαλύψουμε τις πιθανές επιλογές για το κομμάτι της διανεμημένης επεξεργασίας (\textenglish{distributed computing}). Αναλογιστήκαμε τα προτερήματα και μειονεκτήματα διάφορων επιλογών, συμπεριλαμβανομένων των ... 

Επιλέξαμε να προχωρήσουμε με το Ethereum και όχι κάποια άλλη πλατφόρμα επειδή ...

Το Ethereum είναι ...
Παρέχει Smart Contracts ακολουθώντας το μοντέλο ...
Proof of work είναι ...
Ο τρόπος που υπολογίζεται και πληρώνεται η καταναλώμενη επεξεργαστική ισχύς είναι ...
Αυτά εισάγουν τους εξής περιορισμούς που πρέπει να ληφθούν υπόψιν κατά την υλοποίηση ...

% Παλιό από Drive
Προχωρώντας την τεχνολογία του blockchain ένα βήμα παραπέρα, ξεκίνησαν να δημιουργούνται προγραμματιστικές πλατφόρμες για την ανάπτυξη αποκεντρωτικών εφαρμογών (\textenglish{Decentralized Applications} ή DApps). Η πρώτη και, μέχρι τώρα, πιο δημοφιλής, ισχυρή και λειτουργική πλατφόρμα είναι το Ethereum.

Στο Ethereum υπάρχουν δύο είδη λογαριασμών: οι Externally Owned Accounts και οι \textenglish{Contracts Accounts}. Η διαφορά τους είναι ότι ενώ οι πρώτοι ελέγχονται από τους χρήστες, οι δεύτεροι διαθέτουν ένα αμετάβλητο (immutable) κομμάτι κώδικα το οποίο αποτελεί ένα \textenglish{Smart Contract}. Όταν μια συναλλαγή σταλεί σε ένα Smart Contract, ο συσχετιζόμενος κώδικας εκτελείται από την  “Ethereum Virtual Machine (EVM)” σε κάθε κόμβο (και εν τέλει όλοι έρχονται σε consensus). Φυσικά, για την εκτέλεση των operations στην EVM (όπως και για τα απλά \textenglish{transactions}) χρειάζεται να δοθούν κάποια fees στους miners ως ανταμοιβή για την εργασία τους.


\subsection{IPFS, OrbitDB}

Όπως η επιλογή του Blockchain, που περιγράφηκε στο προηγούμενο κεφάλαιο (\textenglish{insert reference}), ομοίως και η επιλογή του λογισμικού που θα χρησιμοποιηθεί για την κατανεμημένη αποθήκευση δεδομένων ξεκίνησε με μία έρευνα των επιλογών που υπάρχουν. Αναλογιστήκαμε τα προτερήματα και μειονεκτήματα διάφορων επιλογών, συμπεριλαμβανομένων των ... 

Επιλέξαμε να προχωρήσουμε με το IPFS και την OrbitDB έναντι άλλων λύσεων επειδή ...

Το IPFS είναι ...
Παρέχει ... με τον εξής τρόπο ...
Δωρεάν ...
Αυτά τα χαρακτηριστικά εισάγουν τους εξής περιορισμούς που πρέπει να ληφθούν υπόψιν κατά την υλοποίηση ...

Η OrbitDB είναι ... και χρησιμοποιεί το IPFS για να καταφέρει τα εξής χαρακτηριστικά ...
Περιορισμοί πάλι κλπ ...

% Παλιό από Drive
Ένα από τα προβλήματα που προκύπτουν με την αποκέντρωση εφαρμογών είναι αυτό της εύρεσης των resources που χρειαζόμαστε. Το πρόβλημα έγκειται στο γεγονός ότι δεν υπάρχει ένας server και άρα μία μοναδική διεύθυνση IP από την οποία μπορούμε να πάρουμε το αντικείμενο που ψάχνουμε, διότι όλα είναι διανεμημένα στο δίκτυο. Επιπλέον η αποθήκευση μεγάλου όγκου πληροφοριών στο Ethereum on-chain έχει απαγορευτικά μεγάλο κόστος οπότε απορρίπτεται.

Το πρόβλημα ανάγεται σε αυτό της επικοινωνίας μεταξύ των κόμβων. Αν καθοριστεί ένα πρότυπο επικοινωνίας μεταξύ των κόμβων τότε τα αντικείμενα μπορούν να βρεθούν ζητώντας τα από τους γειτονικούς κόμβους, οι οποίοι με τη σειρά τους θα τα ζητήσουν από τους δικούς τους γείτονες και ου το κάθε εξής, μέχρι το αντικείμενο να βρεθεί και να σταλεί στον κόμβο που αρχικά το ζήτησε.

Το IPFS είναι ένα νέο πρωτόκολλο μεταφοράς υπερκειμένου που βρίσκεται ακόμα υπό ανάπτυξη. Όπως το γνωστό πρωτόκολλο HTTP, για συγκεντρωτικά συστήματα, έτσι και το IPFS, για αποκεντρωτικά συστήματα, καθορίζει το πρότυπο το οποίο θα χρησιμοποιούν τα τερματικά του δικτύου για να επικοινωνούν μεταξύ τους. Χρησιμοποιεί κρυπτογραφικές τεχνικές, όπως αυτές που είδαμε παραπάνω, καθώς και διάφορες άλλες τεχνολογίες για να:

\begin{itemize}
    \item συντονίσει τη παράδοση περιεχομένου
    \item υλοποιήσει στρώμα σύνδεσης μέσω οποιουδήποτε πρωτοκόλλου δικτύου
    \item ορίσει ένα σύστημα ονομάτων τομέων (DNS)
    \item υλοποιήσει στρώμα δρομολόγησης
    \item διαμοιράσει αρχεία με peer-to-peer (P2P) τεχνικές
\end{itemize}

Έτσι το IPFS ορίζει ένα νέο δίκτυο υπολογιστών που συμπληρώνει τα ήδη υπάρχοντα (www, Tor, .bit) διατηρώντας πλήρως αποκεντρωμένη αρχιτεκτονική και κανένα κεντρικό σημείο αποτυχίας.

Συνοπτικά, δηλαδή, στο IPFS αποθηκεύονται διευθυνσιοδοτημένα βάσει περιεχομένου (\textenglish{content addressable}) αρχεία, τα οποία ανακτώνται βάσει του hash των περιεχομένων τους (αντί για την τοποθεσία τους), ενώ ανακαλύπτονται και διαμοιράζονται μέσω του παρεχόμενου P2P network layer. Αξίζει, ωστόσο, να σημειωθεί πως το layer αυτό δεν εγγυάται από μόνο του το hosting των αρχείων και το διαθέσιμο bandwidth για την ανάκτηση τους, πράγμα που σημαίνει ότι θα πρέπει να γίνει παροχή υποδομής από τους χρήστες ικανής να υποστηρίξει έναν ελάχιστο αριθμό κόμβων αποθήκευσης.

Πρόκειται για συμπληρωματικές τεχνολογίες στις παραπάνω που στόχο έχουν την οργάνωση των αποθηκευμένων πληροφοριών σε μορφή βάσεων δεδομένων.
