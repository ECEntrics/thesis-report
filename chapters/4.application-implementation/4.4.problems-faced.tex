\section{Προβλήματα ανάπτυξης} \label{section:4-4-problems-faced}

Σε αυτήν την ενότητα περιγράφονται οι μεγαλύτερες δυσκολίες που αντιμετωπίστηκαν κατά την ανάπτυξη της πλατφόρμας. Αυτές μπορεί να αναφέρονται σε τεχνικά θέματα, αλλά και στις κοινωνικές και πολιτισμικές συνθήκες που επικρατούν στον χώρο των DApps και των crypto γενικότερα.

Μία από τις μεγαλύτερες τροχοπέδες που καθυστέρησε σοβαρά την ανάπτυξη ήταν η πρωιμότητα των βιβλιοθηκών και των εργαλείων ανάπτυξης. Οι βασικότερες βιβλιοθήκες που χρησιμοποιήθηκαν ήταν σε πρώτο ή δεύτερο πειραματικό στάδιο (alpha και beta phase αντίστοιχα). Συγκεκριμένα:

\begin{itemize}
	\item Όλα τα εργαλεία της σουίτας Truffle ήταν σε alpha phase κατά την ανάπτυξη (κάποια έχουν περάσει σε beta πλέον).
	\item Το IPFS (συγκεκριμένα η βιβλιοθήκη js-ipfs) βρίσκεται ακόμα σε alpha έκδοση.
	\item Η OrbitDB βρίσκεται ακόμα σε alpha phase.
	\item Η γλώσσα των contracts, Solidity, ακόμα δεν έχει βγάλει version 1.0 καθώς αλλάζει διαρκώς με breaking changes\footnote{Από τη σελίδα του πηγαίου κώδικα \url{https://github.com/ethereum/solidity}}.
\end{itemize}

Αυτή η έλλειψη ώριμων βιβλιοθηκών και εργαλείων προκάλεσε μείζονα προβλήματα. Συχνά έπρεπε να διορθωθούν προβλήματα των βιβλιοθηκών, ή να γίνει δουλειά που να τα παρακάμπτει. Άλλες φορές απαιτήθηκαν πολλές ώρες αποσφαλμάτωσης και δοκιμών ώστε να λειτουργήσουν τα χαρακτηριστικά που υπόσχονταν τα εργαλεία.

Ένα άλλο πρόβλημα ήταν η έλλειψη εργαλείων για ορισμένες διαδικασίες. Δύο βασικά παραδείγματα αυτού αποτελούν πρώτον η έλλειψη υποστήριξης για integration/end-to-end testing των contracts κατά την ανάπτυξη (πλέον υπάρχουν κάποιες λύσεις) και δεύτερον η έλλειψη έτοιμων διαδικασιών, plugins και integrations του Jenkins με τα εργαλεία ανάπτυξης και ειδικά με τη σουίτα Truffle.

Σε παρόμοια κατάσταση βρίσκεται και η γενική συναίνεση σχετικά με τα best practices. Σε διάφορα μέρη της ανάπτυξης παρατηρήθηκε ότι δεν υπήρχε κάποια διαμορφωμένη άποψη στην κοινότητα και κάθε ομάδα ανάπτυξης εφάρμοζε την δική της ιδέα. Αυτό καθιστά δύσκολη την ανάπτυξη από αρχάριους προγραμματιστές χωρίς καθοδήγηση. Ένα άλλο σχετικό πρόβλημα που παρατηρήθηκε είναι ότι στον χώρο υπάρχει ακόμα πολύς θόρυβος, δηλαδή σημαντικό μέρος των πηγών που βρίσκονται στο διαδίκτυο είναι αντικρουόμενες ή σε πολλές περιπτώσεις οι προτάσεις τους απορρίπτονται από την κοινότητα.

Τελικώς, ένα μη τεχνικό ζήτημα που έπρεπε να αντιμετωπιστεί είναι η αβεβαιότητα της βιωσιμότητας, εξέλιξης και αποδοχής της τεχνολογίας blockchain και των εφαρμογών που βασίζονται σε αυτήν από το ευρύ κοινό. Αυτό συναίνεσε αρνητικά, καθώς δημιούργησε μία επιτακτικότητα προσοχής της εμπειρίας του χρήστη (UX), κάτι που φυσιολογικά δεν αποτελεί σημαντικό μέρος της ανάπτυξης ενός PoC. Η ανάγκη για προσοχή του UX πηγάζει από την ανάγκη για συγκράτηση των χρηστών (user retention) με στόχο την αντιστροφή της αβεβαιότητας και την παροχή μίας γνησίως ευχάριστης εμπειρίας.
