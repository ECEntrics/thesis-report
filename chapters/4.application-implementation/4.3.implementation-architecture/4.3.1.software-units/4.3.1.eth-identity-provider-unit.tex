\subsubsection{Άρθρωμα eth-identity-provider} \label{subsubsection:4-3-1-eth-identity-provider-unit}

Η λειτουργία της βάσης OrbitDB επιτρέπει τη χρήση προσαρμοσμένων  orbit-db-identity-provider, οι οποίοι θα δημιουργούν και θα επικυρώνουν
τα μοναδικά αναγνωριστικά των χρήστών (OrbitDB Identity) βάσει προσαρμοσμένων εξωτερικών αναγνωριστικών (external identifier), όπως παρουσιάζεται στο σχήμα \ref{figure:4-2-4-2-orbit-db-identity}. 

Στην περίπτωση της εφαρμογής Concordia είναι χρήσιμο να μπορούν να υπολογιστούν με ντετερμινιστικό τρόπο οι OrbitDB βάσεις δεδομένων του κάθε χρήστη, για λόγους απλότητας και εξοικονόμησης αποθηκευτικού χώρου επί του blockchain. Έτσι, αφού κάθε χρήστης ορίζεται μοναδικά μέσω της διεύθυνσης Ethereum με την οποία εγγράφεται και συνδέεται, αυτή θα πρέπει να αποτελεί και το εξωτερικό αναγνωριστικό στο πεδίο id της OrbitDB Identity.

Για αυτόν το λόγο υλοποιήθηκε το άρθρωμα eth-identity-provider, το οποίο:

\begin{itemize}
	\item Παράγει ένα OrbitDB Identity για τον χρήστη, με id τον συνδυασμο του Ethereum address του και του address του κεντρικού contract της εφαρμογής\footnote{Το δεύτερο εισήχθη για την αποφυγή προβλημάτων σε πολλαπλές αναπτύξεις συμβολαίων.}. Αυτό επιτυγχάνεται με την υπογραφή μίας συναλλαγής με το Ethereum private key του χρήστη, μέσω του MetaMask.
	\item Επικυρώνει τις OrbitDB Identity που απαιτούνται, εξασφαλίζοντας ότι υπογράφηκαν από τα Ethereum private key των κατόχων τους. 
	\item Διασφαλίζει ντετερμινιστικές, υπολογίσιμες διευθύνσεις OrbitDB βάσεων για τον κάθε χρήστη.
\end{itemize}

Αυτό το άρθρωμα γίνεται διαθέσιμο για χρήση με τη μορφή βιβλιοθήκης μέσω του μητρώου λογισμικού npm (\url{https://www.npmjs.com/package/@ecentrics/eth-identity-provider}), ενώ το αποθετήριό του βρίσκεται στο GitLab (\url{https://gitlab.com/ecentrics/eth-identity-provider}).
