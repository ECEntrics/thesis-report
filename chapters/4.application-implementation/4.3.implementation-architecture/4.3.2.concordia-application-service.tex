\subsection{Concordia Application} \label{subsection:4-3-2-concordia-application-service}

\subsubsection{Περιγραφή - Στόχοι υπηρεσίας}

Η εφαρμογή Concordia (Concordia Application) εκθέτει τις γραφικές διεπαφές μέσω των οποίων αλληλεπιδρούν οι χρήστες με το σύστημα. Αποτελεί τον δίαυλο επικοινωνίας του τελικού χρήστη με το blockchain και με τη βάση OrbitDB. Η αρχιτεκτονική της υπηρεσίας φαίνεται στο σχήμα \ref{figure:4-3-concordia-application-architecture}. Μέσω της εφαρμογής Concordia οι χρήστες μπορούν να:

\begin{itemize}
    \item περιηγηθούν και διαβάσουν το περιεχόμενο της πλατφόρμας

    \item δημιουργήσουν λογαριασμό χρήστη

    \item δημοσιεύσουν και τροποποιήσουν προσωπικές τους πληροφορίες όπως η τοποθεσία και η εικόνα προφίλ

    \item δημιουργήσουν θέματα (topics)

    \item δημιουργήσουν ψηφοφορίες (polls), καθώς και να ψηφίσουν σε αυτές

    \item δημιουργήσουν και τροποποιήσουν μηνύματα (posts)

    \item υπερψηφίσουν (up-vote) ή καταψηφίσουν (down-vote) μηνύματα άλλων χρηστών
\end{itemize}

Η υπηρεσία αποτελείται από κώδικα γραμμένο σε Javascript ο οποίος γίνεται διαθέσιμος στους τελικούς χρήστες με τη μορφή εφαρμογής διαδικτύου (web application) μέσω ενός διακομιστή (server). Παρόλο που η υπηρεσία προσφέρει τη γραφική διεπαφή χρήστη μόνο στην αγγλική γλώσσα, έχει παραμετροποιηθεί ώστε να είναι δυνατή η εύκολη μεταγλώττιση της χωρίς την ανάγκη πραγματοποίησης μεγάλων αλλαγών στον κώδικα.

Χρησιμοποιείται η βιβλιοθήκη React για την οργάνωση και ανάπτυξη των συνθετικών τμημάτων (components) του γραφικού περιβάλλοντος. Για το γραφικό περιβάλλον γίνεται χρήση του framework της Semantic UI. Χρησιμοποιείται η βιβλιοθήκη Redux για τη διαχείριση κατάστασης της εφαρμογής (state management), % todo: find a better greek translation
καθώς και η βιβλιοθήκη Redux-Saga για τη διαχείριση ασύγχρονων παράπλευρων ενεργειών (side-effects) σε ένα σύστημα βασισμένο σε συμβάντα (event-based). Άλλες βιβλιοθήκες χρησιμοποιούνται για διάφορα μέρη της υπηρεσίας, ενώ χρησιμοποιούνται επίσης τα αρθρώματα που περιγράφηκαν προηγουμένως για την επίτευξη διαφορετικών στόχων. Ο πλήρης κατάλογος των βιβλιοθηκών και αρθρωμάτων μπορεί να βρεθεί στον κώδικα της υπηρεσίας στο παράρτημα. % todo: add reference to the appendix containing the code or a link to it in the repo

\begin{figure}[H]
    \centering
    \includegraphics[width=.7\textwidth]{assets/figures/chapter-4/4.3.architecture-4.3.2.concordia-application-architecture.png}
    \caption{Αρχιτεκτονική υπηρεσίας Concordia Application}
    \label{figure:4-3-concordia-application-architecture}
\end{figure}

Για τη λειτουργία της υπηρεσία Concordia Application είναι απαραίτητα τα αντικείμενα (artifacts) που προκύπτουν από τη μεταγλώττιση των contracts και τη μεταφόρτωση/δημοσίευσή τους στο blockchain. Για την εισαγωγή των artifacts στην υπηρεσία έχουν αναπτυχθεί δύο μέθοδοι.

Η πρώτη μέθοδος είναι η μεταγλώττιση και μεταφόρτωση των contracts πριν την παραγωγή του πακέτου λογισμικού της υπηρεσίας για τελική χρήση (production build). Με αυτό τον τρόπο η υπηρεσία θα έχει πρόσβαση στα artifacts μέσω της βιβλιοθήκης που παράγεται από το άρθρωμα concordia-contracts. Αυτή η μέθοδος έχει το μειονέκτημα ότι το τελικό πακέτο λογισμικού (production build) ``δένεται'' με όποια συγκεκριμένη έκδοση των contracts είναι διαθέσιμη κατά τη δημιουργία του πακέτου. Αυτό σημαίνει ότι σε ενδεχόμενη ενημέρωση των contracts πρέπει αναγκαστικά να δημιουργηθεί και νέα έκδοση του πακέτου λογισμικού της υπηρεσίας Concordia Application.

Για την αποφυγή του παραπάνω προβλήματος αναπτύχθηκε η δεύτερη μέθοδος προσκόμισης των contract artifacts, η οποία είναι η λήψη τους (download) από μία άλλη τοποθεσία στο διαδίκτυο. Σε αυτή τη μέθοδο, η εφαρμογή κατά την εκκίνησή της πραγματοποιεί ένα HTTP αίτημα (HTTP request) σε διεύθυνση η οποία δίνεται ως μεταβλητή περιβάλλοντος (environment variable). Η απάντηση του αιτήματος αναμένεται να περιέχει τα artifacts ώστε η εφαρμογή να τα χρησιμοποιήσει.

\subsubsection{Διανομή}

Η υπηρεσία Concordia Application πακετάρεται μαζί με τον διακομιστή nginx και γίνεται διαθέσιμη για χρήση ως εικόνα docker (docker image) μέσω του αποθετηρίου εικόνων dockerhub. Κατά την εκτέλεση της εικόνας οι χρήστες μπορούν μέσω μεταβλητών περιβάλλοντος να ορίσουν παραμέτρους της εκτέλεσης όπως η διεύθυνση του εξυπηρετητή (host location) της εφαρμογής και οι τοποθεσίες των υπηρεσιών Rendezvous Server και Contracts Provider.