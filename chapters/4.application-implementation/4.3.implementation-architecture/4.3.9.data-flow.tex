\subsection{Ροή πληροφορίας} \label{subsection:4-3-9-data-flow}

Στην παρούσα υποενότητα θα αναλυθεί η ροή της πληροφορίας στο σύστημα. Λόγω των πολλαπλών υπηρεσιών, της κατάτμησης την πληροφορίας και των διαφορετικών σημείων αποθήκευσης της, η ροή της πληροφορίας στο σύστημα ακολουθεί ένα σχετικά περίπλοκο μονοπάτι (σε σχέση με κλασσικές, μονολιθικές, κεντροποιημένες εφαρμογές).

Αρχικά θα γίνει αναφορά στη διαδικασία αποθήκευσης των νέων πληροφοριών. Η μοναδική πηγή παραγωγής δεδομένων στο σύστημα είναι οι χρήστες και κατ' επέκταση η υπηρεσία Concordia Application, εφόσον είναι η μοναδική υπηρεσία με την οποία αυτοί αλληλεπιδρούν. Τα δεδομένα που δημιουργούν οι χρήστες (πληροφορίες χρηστών, τίτλοι θεμάτων και περιεχόμενο μηνυμάτων) κατατέμνονται πριν αποθηκευτούν. Η πληροφορία που εισάγεται στο σύστημα διαχωρίζεται σε δύο μέρη. Στο blockchain αποθηκεύεται ένας δείκτης προς τα δεδομένα, ενώ τα πραγματικά δεδομένα αποθηκεύονται στη βάση OrbitDB. Ο δείκτης εκτός από την άμεση χρησιμότητα στην εύρεση των δεδομένων, παρέχει και την έμμεση λειτουργικότητα της δημιουργίας απαραίτητων μεταδομένων όπως ο αριθμός των θεμάτων στο σύστημα ή των μηνυμάτων σε ένα θέμα.

Από την πλευρά της εύρεση των πληροφοριών στο σύστημα, η ροή είναι ως εξής. Αρχικά, είναι απαραίτητη η αναζήτηση στο blockchain για την εύρεση του δείκτη προς τα δεδομένα. Έπειτα, τα δεδομένα μπορούν να ανακτηθούν μέσω του IPFS από τον εκάστοτε χρήστη ή από κάποιον Pinner.

Τέλος, παρακάτω δίνεται ένα παράδειγμα εισαγωγής πληροφορίας στο σύστημα και έπειτα ανάκτησης της ίδιας πληροφορίας.

Έστω, χρήστης που δημιουργεί νέο θέμα. Τα δεδομένα που παράγονται είναι ο τίτλος του θέματος και το περιεχόμενο του πρώτου μηνύματος. Μεταδεδομένα της δημιουργίας είναι η διεύθυνση του δημιουργού του θέματος. Για την αποθήκευση του θέματος στο σύστημα δημιουργείται πρώτα συναλλαγή στο blockchain ώστε να δημιουργηθεί μία νέα εγγραφή στον πίνακα των θεμάτων. Η εγγραφή αυτή δεν περιέχει τίποτα παρά μόνο τη διεύθυνση του δημιουργού χρήστη. Αν η συναλλαγή είναι επιτυχής, θα επιστραφεί ο αύξων αριθμός του νέου θέματος. Έπειτα, στην προσωπική βάση OrbitDB του χρήστη και στον πίνακα των θεμάτων θα προστεθεί εγγραφή με αναγνωριστικό τον αύξων αριθμό του θέματος όπου θα αποθηκευτούν τα δεδομένα του τίτλου και πρώτου μηνύματος. Στο σχήμα \ref{figure:4-3-data-flow-insert} παρουσιάζεται γραφικά η διαδικασία.

\begin{figure}[H]
    \centering
    \begin{sequencediagram}
    \newthread{actor}{Actor}{}
    \newinst[4]{concordia}{:Concordia}{}
    \newinst[2]{eth}{:Ethereum}{}
    \newinst[1]{orbit}{:OrbitDb}{}

    \begin{call}{actor}{Create topic}{concordia}{Topic creation form}
    \end{call}

    \begin{call}{actor}{Submit}{concordia}{New topic page}

        \begin{call}{concordia}{Create topic}{eth}{New topic ID}
        \end{call}

        \begin{call}{concordia}{Save topic information}{orbit}{}
        \end{call}

    \end{call}
\end{sequencediagram}

    \caption{Διάγραμμα ακολουθίας δημιουργίας θέματος}
    \label{figure:4-3-data-flow-insert}
\end{figure}

Έστω, χρήστης που επιθυμεί να διαβάσει το προηγούμενο θέμα. Αρχικά, πρέπει να διαβαστούν τα μεταδεδομένα του συγκεκριμένου θέματος από το blockchain. Έπειτα, διαβάζονται από το blockchain οι αύξοντες αριθμοί των μηνυμάτων που έχουν δημοσιευτεί στο θέμα αυτό. Σε μία τελευταία ανάκτηση από το blockchain διαβάζονται τα μεταδομένα του κάθε μηνύματος. Έπειτα, η πληροφορία αυτή εμπλουτίζεται από τα δεδομένα του θέματος και των μηνυμάτων, τα οποία ανακτώνται από τις προσωπικές βάσεις Orbit κάθε χρήστη. Στο σχήμα \ref{figure:4-3-data-flow-read} φαίνεται το διάγραμμα ροής της πληροφορίας κατά την ανάκτηση πληροφοριών από το σύστημα.

\begin{figure}[H]
    \centering
    \begin{sequencediagram}
    \newthread{actor}{Actor}{}
    \newinst[2]{concordia}{:Concordia}{}
    \newinst[2]{eth}{:Ethereum}{}
    \newinst[1]{orbit}{:OrbitDB}{}

    \begin{call}{actor}{Fetch topic}{concordia}{Topic}

        \begin{call}{concordia}{Get topic}{eth}{Topic}
        \end{call}

        \begin{call}{concordia}{Get topic post IDs}{eth}{Post IDs}
        \end{call}

        \begin{call}{concordia}{Get posts loop}{concordia}{Posts}

            \begin{call}{concordia}{Get post}{eth}{Post}
            \end{call}

        \end{call}

        \begin{call}{concordia}{Retrieve topic information}{orbit}{}
        \end{call}

        \begin{call}{concordia}{Retrieve posts information loop}{concordia}{Posts information}

            \begin{call}{concordia}{Retrieve post information}{orbit}{}
            \end{call}

        \end{call}
    \end{call}
\end{sequencediagram}

    \caption{Διάγραμμα ακολουθίας εύρεσης και ανάκτησης θέματος}
    \label{figure:4-3-data-flow-read}
\end{figure}
