\section{Μεθοδολογία υλοποίησης}

Για την επίτευξη των στόχων που ορίστηκαν και την οργάνωση της εργασίας που απαιτείται σε διαχειρίσιμα μέρη, σχεδιάστηκε η χρήση διάφορων εργαλείων και μεθόδων ανάπτυξης λογισμικού, όπως το σύστημα ελέγχου εκδόσεων (version control system) Git και η μέθοδος οργάνωσης Scrum. Τα εργαλεία αυτά είναι δοκιμασμένα και έχουν εδραιωθεί στη σύγχρονη ανάπτυξη λογισμικού.

Το Git είναι δωρεάν λογισμικό ανοιχτού κώδικα το οποίο επιτρέπει και επικουρεί την απρόσκοπτη ανάπτυξη λογισμικού από πολλαπλά μέλη μίας ομάδας, ταυτόχρονα και διανεμημένα. Αυτό επιτυγχάνεται παρέχοντας ένα πλαίσιο από εργαλεία τα οποία βοηθούν την διαχείριση και ενσωμάτωση των διαφορετικών εκδόσεων του κώδικα τις οποίες αναπτύσσει κάθε μέλος της ομάδας ξεχωριστά. Υπάρχουν διάφορα μοντέλα χρήσης του Git και πιο συγκεκριμένα της δυνατότητας που δίνει για δημιουργία, ανάπτυξη και ένωση κλαδιών (branches). Για τους σκοπούς της παρούσας διπλωματικής χρησιμοποιήθηκε το μοντέλο GitHub flow\cite{4.2-github-flow}. Το μοντέλο αυτό ορίζει ότι κάθε προγραμματιστής θα ανοίγει ένα νέο branch για τη ανάπτυξη ενός νέου χαρακτηριστικού της εφαρμογής ή τη διόρθωση ενός μέρους του κώδικα. Έπειτα, όταν η δουλειά έχει ολοκληρωθεί, το branch ενώνεται (merge) με το βασικό branch της εφαρμογής.

Το Scrum είναι μία μέθοδος οργάνωσης στην οποία ο επιμελητής του Scrum (Scrum master) διαχωρίζει τα ανεξάρτητα μέρη εργασίας (tasks) που πρέπει να υλοποιηθούν για την ολοκλήρωση των στόχων ενός project. Τα μέρη αυτά περιγράφονται αναλυτικά μαζί με τις απαιτήσεις τους και κατατίθενται σε μία λίστα εργασιών (backlog). Έπειτα, μέσα από συσκέψεις (meetings), επιλέγεται ένας αριθμός από μέρη εργασίας τα οποία θα αποτελέσουν το επόμενο Sprint. Κάθε μέρος εργασίας ανατίθεται σε κάποιο μέλος για υλοποίηση και ορίζεται για το Sprint μία χρονική διάρκεια, στόχος της οποίας είναι η περάτωση όλων των μερών εργασίας πριν τη λήξη της. Στο τέλος προθεσμίας που ορίστηκε για το Sprint τα μέλη της ομάδας αποτιμούν τα αποτελέσματα και ορίζουν το επόμενο Sprint. Η διαδικασία επαναλαμβάνεται έως ότου το έργο ολοκληρωθεί.

Μέσα από την χρήση των παραπάνω εργαλείων επιτυγχάνεται η ομαλή συνεργασία στην ανάπτυξη του λογισμικού. Κάθε μέλος της ομάδας δύναται να εργαστεί ανεξάρτητα και χωρίς την ανάγκη διαρκούς επικοινωνίας με τα υπόλοιπα μέλη. Οι στόχοι είναι ορισμένοι, σαφείς και χωρισμένοι σε διαχειρίσιμα μέρη τα οποία δεν καταβάλουν τα μέλη. Ταυτόχρονα, έχοντας ως έδρα καθιερωμένα πρότυπα ανάπτυξης, παρέχεται φορμαλισμός και έτοιμες μέθοδοι επίλυσης προβλημάτων, γεγονός που λειτουργεί καταλυτικά και βοηθά στην αποφυγή τελμάτων κατά τη συγγραφή του κώδικα.

TODO: add continuous integration
