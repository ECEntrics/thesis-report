\section{Αρχιτεκτονική υλοποίησης} \label{section:4-4-implementation-architecture}

Το σύστημα υλοποιήθηκε χρησιμοποιώντας το μοντέλο αρχιτεκτονικής των μικροϋπηρεσιών. Το μοντέλο των μικροϋπηρεσιών βασίζεται στην αποδόμηση του συστήματος σε μικρές μονάδες, οι οποίες συνεργάζονται ώστε να προσφέρουν ένα ενιαίο αποτέλεσμα. Η προσέγγιση αυτή έχει πολλά πλεονεκτήματα σε σύγκριση με την παραδοσιακή ανάπτυξη μονολιθικών εφαρμογών % todo: add reference
. Ο βασικός λόγος για τον οποίο επιλέχθηκε η αρχιτεκτονική μικροϋπηρεσιών είναι η ευκολία που προσφέρει στη γρήγορη ανάπτυξη καινούριων χαρακτηριστικών, ταυτόχρονα από διαφορετικά μέλη μίας ομάδας, ασύγχρονα και χωρίς την ανάγκη συνεχής επικοινωνίας και συνεννόησης μεταξύ τους. Αυτό συμβαίνει επειδή κάθε μέρος του συστήματος (υπηρεσία) είναι αυτόνομο και η ανάπτυξή του είναι διαχωρισμένη από το υπόλοιπο σύστημα με το οποίο είναι αδύναμα συνδεδεμένο (loosely coupled).

Το σύστημα συντίθεται από διάφορες μικροϋπηρεσίες, κάποιες από τις οποίες αναπτύχθηκαν στα πλαίσια αυτής της εργασίας ενώ άλλες αποτελούν δωρεάν λογισμικό ανοιχτού κώδικα. Οι μικροϋπηρεσίες αυτές συνοψίζονται στον παρακάτω πίνακα (πίνακας \ref{table:4-4-microservice-summary}).

\begin{table}[h!]
\begin{center}
\begin{tabular}{|l l|} 
 \hline
 \textbf{Μικροϋπηρεσία} & \textbf{Σύντομη περιγραφή - Αντικείμενο/Στόχος} \\
 \hline\hline
 Concordia Application & Υπηρεσία με την οποία αλληλεπιδρούν οι χρήστες. \\ [0.5ex]
 \hline
 Concordia Contracts Migrator & Υπηρεσία μεταφόρτωσης των συμβολαίων (contracts) στο blockchain. \\ [0.5ex]
 \hline
 Concordia Pinner & Υπηρεσία καρφιτσώματος δεδομένων. \\ [0.5ex]
 \hline
 Concordia Contracts Provider & Υπηρεσία που διαμοιράζει τα contracts της εφαρμογής μέσω HTTP. \\ [0.5ex]
 \hline
 Ganache & Τοπικό, ιδιωτικό Ethereum blockchain. \\ [0.5ex]
 \hline
 Rendezvous Server & Υπηρεσία εύρεσης ομότιμων χρηστών. \\ [0.5ex]
 \hline
\end{tabular}
\end{center}
\caption{Σύντομη περιγραφή υπηρεσιών συστήματος.}
\label{table:4-4-microservice-summary}
\end{table}

Στα πλαίσια της εργασίας αναπτύχθηκαν επίσης διάφορα αρθρώματα, κυρίως με τη μορφή βιβλιοθηκών Javascript. Τα αρθρώματα χρησιμοποιούνται από τις υπηρεσίες για την επίτευξη των επιμέρους εργασιών. Η ανάπτυξη του λογισμικού σε ξεχωριστά αρθρώματα επιτρέπει την εύκολη επαναχρησιμοποίηση του κώδικα καθώς και τον διαχωρισμό των αυτόνομων τμημάτων κώδικα. Τα αρθρώματα συνοψίζονται στον παρακάτω πίνακα (πίνακας \ref{table:4-4-software-units-summary}).

\begin{table}[h!]
\begin{center}
\begin{tabular}{|l l|} 
 \hline
 \textbf{Άρθρωμα} & \textbf{Σύντομη περιγραφή - Αντικείμενο/Στόχος} \\
 \hline\hline
 Άρθρωμα concordia-shared & Χρήσιμα εργαλεία και σταθερές συστήματος. \\ [0.5ex]
 \hline
 Άρθρωμα concordia-contracts & Μεταγλώττιση των contracts και διάθεση των artifacts. \\ [0.5ex]
 \hline
 Άρθρωμα eth-identity-provider & Δημιουργία μοναδικού αναγνωριστικού για χρήση στην βάση OrbitDB. \\ [0.5ex]
 \hline
 Άρθρωμα drizzle & Βελτιωμένη προγραμματιστική διεπαφή επικοινωνίας με το blockchain. \\ [0.5ex]
 \hline
 Άρθρωμα breeze & Βελτιωμένη προγραμματιστική διεπαφή χρήσης της βάση OrbitDB. \\ [0.5ex]
 \hline
\end{tabular}
\end{center}
\caption{Σύντομη περιγραφή υπηρεσιών συστήματος.}
\label{table:4-4-software-units-summary}
\end{table}

Τα αρθρώματα και οι υπηρεσίες θα περιγραφούν σε μεγαλύτερη ανάλυση στα επόμενα κεφάλαια.

% ===== =====
% Common software units
% ===== =====
\subsection{Αρθρώματα} \label{subsection:4-4-software-units}

Στο κεφάλαιο αυτό θα περιγραφούν με μεγαλύτερη λεπτομέρεια τα αρθρώματα που αναπτύχθηκαν.

\vspace{0.5cm}
\textbf{Άρθρωμα concordia-shared}

Το άρθρωμα concordia-shared αποτελεί μία βιβλιοθήκη χρήσιμων εργαλείων και σταθερών. Εδώ περιέχεται όλο το λογισμικό το οποίο πρέπει ή είναι επιθυμητό να συμπεριφέρεται με τον ίδιο τρόπο συνολικά στο σύστημα, όπως για παράδειγμα οι μέθοδοι παραμετροποίησης των υπηρεσιών και μέθοδοι καταγραφής (logging). Το άρθρωμα αυτό χρησιμοποιείται από το άρθρωμα concordia-contracts καθώς και από τις υπηρεσίες Concordia Application, Concordia Pinner και Concordia Contracts Provider.

\vspace{0.5cm}
\textbf{Άρθρωμα concordia-contracts}

Το άρθρωμα αυτό επιτελεί δύο ενέργειες. Αρχικά, είναι το άρθρωμα στο οποίο αναπτύσσονται τα contracts που χρησιμοποιούνται από την εφαρμογή. Στο άρθρωμα αυτό τα contracts μεταγλωττίζονται από κώδικα γλώσσας Solidity, στην κατάλληλη τελική μορφή JSON. Παρέχονται επίσης σενάρια ενεργειών (scripts) ώστε τα contracts να μεταφορτωθούν σε blockchain καθώς και στην υπηρεσία Concordia Contracts Provider. Το άρθρωμα αποτελεί επίσης βιβλιοθήκη η οποία μετά τη μεταγλώττιση και μεταφόρτωση των contracts σε blockchain παρέχει τα contract artifacts. Το άρθρωμα αυτό χρησιμοποιείται από τις υπηρεσίες Concordia Application και Concordia Pinner.

\vspace{0.5cm}
\textbf{Άρθρωμα eth-identity-provider}

Η λειτουργία της βάση OrbitDB απαιτεί τη δημιουργία ενός μοναδικού αναγνωριστικού (identity) χρήστη. Για την εύκολη εξαγωγή ενός αναγνωριστικού χρήστη το οποίο να είναι μεν μοναδικό αλλά να είναι δυνατός ο επανυπολογισμός, χρησιμοποιήθηκε ο συνδυασμός της διεύθυνσης του χρήστη στο δίκτυο Ethereum με τη διεύθυνση του βασικού contract που χρησιμοποιεί η εφαρμογή, ώστε να παραχθεί το αναγνωριστικό χρήστη. Ο υπολογισμός του συνδυασμού αυτού υλοποιείται από αυτό το άρθρωμα.

\vspace{0.5cm}
\textbf{Άρθρωμα drizzle}

Το άρθρωμα drizzle που χρησιμοποιείται στην υπηρεσία Concordia Application είναι μία τροποποιημένη έκδοση της Javascript βιβλιοθήκης Drizzle που προσφέρεται από τη σουίτα εργαλείων Truffle. Η τροποποιημένη βιβλιοθήκη αναπτύχθηκε στα πλαίσια της διπλωματικής με στόχο τη διευκόλυνση της χρήσης του Drizle και την επιδιόρθωση προβληματικών σημείων της πρωτότυπης βιβλιοθήκης.

Το άρθρωμα drizzle υλοποιεί τις προγραμματιστικές διεπαφές μέσω των οποίων πραγματοποιείται η επικοινωνία της εφαρμογής με το blockchain. Για την επίτευξη της επικοινωνίας αυτής, η βιβλιοθήκη χρησιμοποιεί τη συλλογή βιβλιοθηκών web3.js η οποία αποτελεί τον πιο διαδεδομένο τρόπο διεπαφής με το blockchain σε αποκεντρωτικές εφαρμογές.

\vspace{0.5cm}
\textbf{Άρθρωμα breeze}

Το άρθρωμα αυτό αποτελεί μία βιβλιοθήκη περίβλημα (wrapper) της βιβλιοθήκης OrbitDB. Η OrbitDB είναι μία βιβλιοθήκη η οποία προσφέρει τις απαραίτητες προγραμματιστικές διεπαφές για τη χρήση της βάσης δεδομένων με το ίδιο όνομα. Μέσα από τη χρήση των βιβλιοθηκών που προσφέρονται από το IPFS για την αποθήκευση δεδομένων, η OrbitDB καταφέρνει να υλοποιήσει μία αποκεντρωμένη βάση δεδομένων.

Το άρθρωμα breeze κάνει χρήση της βιβλιοθήκης OrbitDB, προσφέρει ωστόσο συγκεκριμένες προγραμματιστικές διεπαφές που διευκολύνουν τόσο την παραμετροποίηση της βάσης όσο και τη χρήση της, ενώ όπως και στο άρθρωμα drizzle το άρθρωμα breeze αναλαμβάνει να διορθώσει ορισμένα προβλήματα της πρωτότυπης βιβλιοθήκης.

% ===== =====
% concordia-app microservice
% ===== =====
\subsection{Concordia Application} \label{subsection:4-4-concordia-application-service}

Η εφαρμογή Concordia (Concordia Application) εκθέτει τις γραφικές διεπαφές μέσω των οποίων αλληλεπιδρούν οι χρήστες με το σύστημα. Αποτελεί τον δίαυλο επικοινωνίας του τελικού χρήστη με το blockchain και με τη βάση OrbitDB.

Η υπηρεσία αποτελείται από κώδικα γραμμένο σε Javascript ο οποίος γίνεται διαθέσιμος στους τελικούς χρήστες με τη μορφή εφαρμογής διαδικτύου (web application) μέσω ενός διακομιστή (server). Παρόλο που η υπηρεσία προσφέρει τη γραφική διεπαφή χρήστη μόνο στην αγγλική γλώσσα, έχει παραμετροποιηθεί ώστε να είναι δυνατή η εύκολη μεταγλώττιση της χωρίς την ανάγκη πραγματοποίησης μεγάλων αλλαγών στον κώδικα.

Χρησιμοποιείται η βιβλιοθήκη React για την ανάπτυξη και οργάνωση των συνθετικών τμημάτων (components) του γραφικού περιβάλλοντος. Για το γραφικό περιβάλλον γίνεται χρήση του framework της Semantic UI. Χρησιμοποιείται η βιβλιοθήκη Redux για τη διαχείριση κατάστασης της εφαρμογής (state management), % todo: find a better greek translation
καθώς και η βιβλιοθήκη Redux-Saga για τη διαχείριση ασύγχρονων παράπλευρων ενεργειών (side-effects) σε ένα σύστημα βασισμένο σε συμβάντα (event-based). Άλλες βιβλιοθήκες χρησιμοποιούνται για διάφορα μέρη της υπηρεσίας, ενώ χρησιμοποιούνται επίσης τα αρθρώματα που περιγράφηκαν προηγουμένως για την επίτευξη διαφορετικών στόχων. Ο πλήρης κατάλογος των βιβλιοθηκών και αρθρωμάτων μπορεί να βρεθεί στον κώδικα της υπηρεσίας στο παράρτημα. % todo: add reference to the appendix containing the code or a link to it in the repo

Για τη λειτουργία της υπηρεσία Concordia Application είναι απαραίτητα τα αντικείμενα (artifacts) που προκύπτουν από τη μεταγλώττιση των contracts και τη μεταφόρτωση/δημοσίευσή τους στο blockchain. Για την εισαγωγή των artifacts στην υπηρεσία έχουν αναπτυχθεί δύο μέθοδοι. Η μία μέθοδος είναι η μεταγλώττιση και μεταφόρτωση των contracts πριν την παραγωγή του πακέτου λογισμικού της υπηρεσίας για τελική χρήση (production build). Με αυτό τον τρόπο η υπηρεσία θα έχει διαθέσιμα τα artifacts μέσω της βιβλιοθήκης που παράγεται από το άρθρωμα concordia-contracts. Αυτή η μέθοδος έχει το μειονέκτημα ότι το τελικό πακέτο λογισμικού (production build) ``δένεται'' με μία συγκεκριμένη έκδοση των contracts η οποία είναι διαθέσιμη κατά τη δημιουργία του πακέτου. Αυτό σημαίνει ότι σε ενδεχόμενη ενημέρωση των contracts πρέπει αναγκαστικά να δημιουργηθεί και νέα έκδοση του πακέτου λογισμικού της υπηρεσίας Concordia Application.

Για την αποφυγή του παραπάνω προβλήματος αναπτύχθηκε η δεύτερη μέθοδος προσκόμισης των contract artifacts, η οποία είναι η λήψη τους (download) από μία άλλη τοποθεσία στο διαδίκτυο. Σε αυτή τη μέθοδο, η εφαρμογή κατά την εκκίνησή της πραγματοποιεί ένα HTTP αίτημα (HTTP request) σε διεύθυνση η οποία δίνεται ως μεταβλητή περιβάλλοντος (environment variable). Η απάντηση του αιτήματος αναμένεται να περιέχει τα artifacts ώστε η εφαρμογή να τα χρησιμοποιήσει.

% ===== =====
% concordia-contracts-migrator microservice
% ===== =====
\subsection{Concordia Contracts Migrator} \label{subsection:4-4-concordia-contracts-migrator}

Η υπηρεσία αυτή αποτελείται από ένα απλό εκτελέσιμο πρόγραμμα γραμμής εντολών βασισμένο στο άρθρωμα concordia-contracts που αναλύθηκε σε προηγούμενο κεφάλαιο (κεφάλαιο \ref{subsection:4-4-software-units}). Το πρόγραμμα μεταγλωττίζει τα contracts και έπειτα τα μεταφορτώνει στο blockchain το οποίο είναι ορισμένο με χρήση μεταβλητών περιβάλλοντος. Τέλος, αν οι κατάλληλες μεταβλητές περιβάλλοντος είναι ορισμένς το πρόγραμμα μεταφορτώνει τα τελικά artifacts σε αποθετήριο.

% ===== =====
% concordia-pinner microservice
% ===== =====
\subsection{Concordia Pinner} \label{subsection:4-4-concordia-pinner-service}

Η υπηρεσία καρφιτσώματος περιεχομένου (Concordia Pinner) αποτελεί μία εφαρμογή τερματικού (temrinal application/cmd application) η οποία στοχεύει στο καρφίτσωμα (pinning) του περιεχομένου που αποθηκεύεται στο IPFS μέσω της βάσης OrbitDB. Η υπηρεσία είναι γραμμένη στη γλώσσα προγραμματισμού Javascript.

Λόγω του τρόπου λειτουργίας % todo: insert reference
του IPFS, το περιεχόμενο που αναρτούν οι χρήστες πρέπει να καρφιτσώνεται από άλλους χρήστες ή αυτόνομες εφαρμογές, όπως η υπηρεσία Concordia Pinner, ώστε να είναι διαθέσιμο. Αν το περιεχόμενο δεν καρφιτσωθεί τότε μόνο ο/η χρήστης που το δημιούργησε μπορεί να το διαθέσει στους υπόλοιπους χρήστες, έτσι αν ο/η δημιουργός δεν είναι διαθέσιμος/διαθέσιμη, ούτε το περιεχόμενο θα είναι. Η υπηρεσία αυτή υλοποιήθηκε για να εγγυηθεί η διαθεσιμότητα του περιεχομένου του συστήματος που αποθηκεύεται στο IPFS (τίτλοι topics, τίτλοι posts και άλλα).

Η υπηρεσία συνδέεται στο blockchain από όπου παρακολουθεί την κατάσταση του συστήματος και ``ακούει'' για νέους χρήστες, θέματα και μηνύματα. Η υπηρεσία συνδέεται επίσης στο IPFS, έτσι όταν δημιουργηθεί νέο περιεχόμενο στο σύστημα, η υπηρεσία το καρφιτσώνει αυτόματα. Με αυτό τον τρόπο, διατηρώντας την υπηρεσία διαθέσιμη για πάντα, εκτελώντας τη για παράδειγμα σε ένα διακομιστή (server), διαβεβαιώνεται η διαθεσιμότητα του περιεχομένου.

% ===== =====
% concordia-contracts-provider microservice
% ===== =====
\subsection{Concordia Contracts Provider} \label{subsection:4-4-concordia-contracts-provider-service}

Η υπηρεσία Contracts Provider αποτελεί μία βοηθητική υπηρεσία η οποία υλοποιεί ένα απλό αποθετήριο για τα contract artifacts. Είναι γραμμένη σε Javascript και διαθέτει δύο HTTP endpoints, ένα για τη μεταφόρτωση (upload) των artifacts προς την υπηρεσία και ένα για τη λήψη (download) από την υπηρεσία. Η υπηρεσία υποστηρίζει επίσης την επισύναψη ετικετών στα artifacts, όπως η έκδοση (version) ή το κλαδί ανάπτυξης (branch, για παράδειγμα master/develop).

Η υπηρεσία χρησιμοποιείται σε μία προσπάθεια αποσύνδεσης της βασικής εφαρμογής που υλοποιεί η υπηρεσία Concordia Application από μία συγκεκριμένη έκδοση των contracts. Οι λόγοι που αυτό είναι επιθυμητό αναπτύχθηκαν στην περιγραφή της υπηρεσίας Concordia Application (κεφάλαιο \ref{subsection:4-4-concordia-application-service}). Ωστόσο, η υπηρεσία Contracts Provider αποτελεί σημείο κεντροποίησης του συστήματος, για το λόγο αυτό θεωρείται προσωρινή λύση η οποία θα μπορούσε να αντικατασταθεί από αποκεντρωτικές λύσεις όπως η μεταφόρτωση των artifacts στο IPFS και ο διαμοιρασμός τους από εκεί.

% ===== =====
% rendezvous-ganache microservice
% ===== =====
\subsection{Ganache} \label{subsection:4-4-ganache-service}

Η υπηρεσία Ganache αποτελεί μία εφαρμογή τερματικού η οποία είναι μέρος της δωρεάν σουίτας ανοιχτού λογισμικού Truffle. Η εφαρμογή δημιουργεί ένα τοπικό, ιδιωτικό blockchain το οποίο ακολουθεί το πρότυπο του Ethereum. Επίσης, η εφαρμογή δρα ως minner στο δίκτυο, διεκπεραιώνοντας όλες τις συναλλαγές.

% ===== =====
% rendezvous-server microservice
% ===== =====
\subsection{Rendezvous Server} \label{subsection:4-4-rendezvous-server-service}

Η υπηρεσία αποτελεί δωρεάν λογισμικό ανοιχτού κώδικα το οποίο χρησιμοποιήθηκε (αλλά δεν αναπτύχθηκε) στα πλαίσια της διπλωματικής και υλοποιεί το πρωτόκολλο rendezvous για την εύρεση ομότιμων χρηστών (peers). Η υπηρεσία είναι απαραίτητη για τη λειτουργία του IPFS, ώστε οι ομότιμοι χρήστες (peers) να μπορούν να ανακαλύψουν τις διευθύνσεις των υπόλοιπων χρηστών του δικτύου.

% ===== =====
% microservice communication
% ===== =====
\subsection{Διασύνδεση υπηρεσιών} \label{subsection:4-4-service-communication}

% ===== =====
% data flow
% ===== =====
\subsection{Ροή πληροφορίας} \label{subsection:4-4-data-flow}
