\subsubsection{Jenkins} \label{subsection:4-3-1-3-jenkins}

\logo{chapter-4/4.3.jenkins-logo}{Jenkins logo}

Το Jenkins είναι ένας πλήρως παραμετροποιήσιμος και επεκτάσιμος διακομιστής αυτοματοποίησης (automation server). Ο διακομιστής μπορεί να αυτοματοποιήσει τις διαδικασίες ελέγχου, ολοκλήρωσης, παράδοσης και εγκατάστασης του κώδικα, υλοποιώντας έτσι βασικές διαδικασίες που ορίζει το DevOps, συνεχή έλεγχο (continuous testing), συνεχή ολοκλήρωση (continuous integration), συνεχή παράδοση (continuous delivery) και συνεχή εγκατάσταση (continuous deployment). Επίσης, το Jenkins μπορεί να παραμετροποιηθεί μέσω των ρυθμίσεων που προσφέρει και των επεκτάσεων (plugins) που υπάρχουν ώστε να παρέχει τις δυνατότητες αυτές για οποιαδήποτε πλατφόρμα, γλώσσα και περιβάλλον ανάπτυξης.

Στο Jenkins είναι δυνατός ο ορισμός με χρήση κώδικα (σε Groovy και στο DSL που παρέχεται από το Jenkins) πολλαπλών γραμμών εργασιών (pipeline). Οι γραμμές εργασιών συντίθενται από πολλαπλά βήματα τα οποία επιτελούν ξεχωριστούς στόχους προς το τελικό αποτέλεσμα της γραμμής. Τα βήματα μπορούν να τρέχουν σειριακά ή παράλληλα. Ενώ δίνεται η δυνατότητα εκτέλεσης σε πολλαπλά, διανεμημένα συστήματα καθώς και άλλες προχωρημένες λειτουργικότητες.
