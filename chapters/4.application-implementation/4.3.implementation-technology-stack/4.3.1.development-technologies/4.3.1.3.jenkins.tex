\subsubsection{Jenkins} \label{subsection:4-3-1-3-jenkins}

\logo{chapter-4/4.3.jenkins-logo}{Jenkins logo}

Το Jenkins είναι ένας πλήρως παραμετροποιήσιμος και επεκτάσιμος διακομιστής αυτοματοποίησης (automation server). Ο διακομιστής μπορεί να αυτοματοποιήσει τις διαδικασίες ελέγχου, ολοκλήρωσης, παράδοσης και εγκατάστασης του κώδικα, υλοποιώντας έτσι βασικές διαδικασίες που ορίζει το DevOps, συνεχή έλεγχο (continuous testing), συνεχή ολοκλήρωση (continuous integration), συνεχή παράδοση (continuous delivery) και συνεχή εγκατάσταση (continuous deployment). Επίσης, το Jenkins μπορεί να παραμετροποιηθεί μέσω των ρυθμίσεων που προσφέρει και των επεκτάσεων (plugins) που υπάρχουν ώστε να παρέχει τις δυνατότητες αυτές για οποιαδήποτε πλατφόρμα, γλώσσα και περιβάλλον ανάπτυξης.

Στο Jenkins είναι δυνατός ο ορισμός με χρήση κώδικα (σε Groovy και στο DSL που παρέχεται από το Jenkins) πολλαπλών γραμμών εργασιών (pipeline). Οι γραμμές εργασιών συντίθενται από πολλαπλά βήματα τα οποία επιτελούν ξεχωριστούς στόχους προς το τελικό αποτέλεσμα της γραμμής. Τα βήματα μπορούν να τρέχουν σειριακά ή παράλληλα. Ενώ δίνεται η δυνατότητα εκτέλεσης σε πολλαπλά, διανεμημένα συστήματα καθώς και άλλες προχωρημένες λειτουργικότητες.

Το Jenkins συνδυάζεται αποτελεσματικά με την πλατφόρμα του Docker που περιγράφηκε προηγουμένως. Μέσω του συνδυασμού δίνεται η ευκαιρία της αυτοματοποίησης του μεγαλύτερου μέρους του DevOps σε ένα απολύτως προβλέψιμο περιβάλλον το οποίο παραμένει σταθερό από την ανάπτυξη του κώδικα μέχρι την τελική εγκατάσταση. Με αυτή την μέθοδο βελτιώνεται σημαντικά η αποτελεσματικότητα των ομάδων ανάπτυξης κώδικα.

% example citations
% Shahin, Mojtaba, et al. “Continuous Integration, Delivery and Deployment: A Systematic Review on Approaches, Tools, Challenges and Practices.” IEEE Access, vol. 5, 2017, pp. 3909–3943.
% Meyer, Mathias. “Continuous Integration and Its Tools.” IEEE Software, vol. 31, no. 3, 2014, pp. 14–16.
% Virmani, Manish. “Understanding DevOps & Bridging the Gap from Continuous Integration to Continuous Delivery.” Fifth International Conference on the Innovative Computing Technology (INTECH 2015), 2015, pp. 78–82.
