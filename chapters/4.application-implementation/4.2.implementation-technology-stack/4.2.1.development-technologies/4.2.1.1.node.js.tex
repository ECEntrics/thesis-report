\subsubsection{Node.js} \label{subsection:4-2-1-1-node.js}

\logo{chapter-4/4.2.node.js-logo}{Node.js logo}

Το Node.js\footnote{\url{https://nodejs.org/}} είναι ένα περιβάλλον χρόνου εκτέλεσης Javascript πολλαπλών πλατφορμών, το οποίο εκτελείται στη μηχανή V8\footnote{\url{https://v8.dev/}} και παρέχει τη δυνατότητα εκτέλεσης κώδικα Javascript εκτός περιηγητών ιστού. Επιτρέπει στους προγραμματιστές να χρησιμοποιούν Javascript για τη σύνταξη εργαλείων γραμμής εντολών και τη δημιουργία κλιμακωτών διαδικτυακών εφαρμογών (κυρίως για εξυπηρετητές). Έχει αρχιτεκτονική βασισμένη σε συμβάντα (event-driven architecture), με δυνατότητα ασύγχρονης εισόδου/εξόδου (asynchronous I/O).\cite{4.2-node.js}

Ένα από τα σημαντικότερα χαρακτηριστικά του Node.js είναι ο ενσωματωμένος διαχειριστής πακέτων του, ο οποίος ονομάζεται npm. Με τον npm γίνεται εφικτή η εγκατάσταση πακέτων (βιβλιοθηκών) από το μητρώο npm (npm registry\footnote{\url{https://www.npmjs.com/}}), καθώς και η οργάνωση και η διαχείρισή τους στα πλαίσια της ανάπτυξης μίας εφαρμογής που εξαρτάται από αυτά.

Το Node.js έχει το αποθετήριό του στο GitHub (\url{https://github.com/nodejs/node}).