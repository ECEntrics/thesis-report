\chapter*{Περίληψη}
\addcontentsline{toc}{chapter}{Περίληψη}

Τις τελευταίες δεκαετίες, η ραγδαία ανάπτυξη του διαδικτύου μετέβαλε ριζικά τις ανθρώπινες
κοινωνίες, μέσω μίας πληθώρας ψηφιακών εφαρμογών. Αυτές, στη συντριπτική τους πλειοψηφία, προσφέρονται από παρόχους υπηρεσιών υπολογιστικού νέφους, ακολουθώντας την αρχιτεκτονική πελάτη-εξυπηρετητή.

Μολονότι αυτό το μοντέλο υλοποίησης έχει αποδειχθεί ιδιαίτερα λειτουργικό και έχει βελτιωθεί αξιοσημείωτα ανά τα χρόνια, η συγκεντρωτική του λογική συνοδεύεται από ορισμένα προβλήματα. Καταρχάς, οι χρήστες καλούνται να εμπιστευθούν τα προσωπικά τους δεδομένα στη διαχειριστική αρχή της εκάστοτε εφαρμογής. Εκείνη, διατηρώντας πλήρη έλεγχο επί αυτών, αποκτάει τη δυνατότητα να τα επεξεργάζεται, να τα διαμοιράζεται και να τα λογοκρίνει, είτε για να εξυπηρετήσει τα συμφέροντά της, είτε για να συμμορφωθεί με άλλες αρχές που της ασκούν εξουσία. Επιπλέον, απουσιάζει η εγγύηση της διαθεσιμότητας των δεδομένων, καθώς, ανά πάσα στιγμή, ο εξυπηρετητής μπορεί να αποσυνδεθεί για αόριστο χρονικό διάστημα και λόγω ποικίλων αιτιών, όπως κάποιας κυβερνοεπίθεσης ή κάποιας φυσικής καταστροφής.

Αυτοί είναι μερικοί βασικοί λόγοι που συνετέλεσαν στην ταχεία ανάπτυξη ενός συνόλου καινοτόμων λογισμικών ανοιχτού κώδικα, τα οποία βασίζονται σε τεχνολογίες όπως το blockchain και τα δίκτυα ομότιμων κόμβων. Τα παραπάνω, αν και βρίσκονται σε σχετικά πρώιμο στάδιο, αποτελούν ήδη ισχυρά εργαλεία δημιουργίας κατανεμημένων και αποκεντρωμένων εφαρμογών.

Στόχος της παρούσας διπλωματικής εργασίας είναι η υλοποίηση μίας αυτόνομης κοινωνικής πλατφόρμας,
η οποία, αξιοποιώντας τεχνολογίες αποκέντρωσης, αφενός θα επιστρέφει την κυριότητα των προσωπικών
δεδομένων στον χρήστη, αφετέρου θα παρέχει τη δυνατότητα διενέργειας διαφανών δημοκρατικών ψηφοφοριών. Αυτά μέσα σε ένα πλαίσιο ανθεκτικό, τόσο σε σφάλματα και επιθέσεις, όσο και σε απόπειρες λογοκρισίας και παραποίησης.

Η αναπτυχθείσα πιλοτική εφαρμογή "Concordia" προσεγγίζει τον παραπάνω στόχο συνδυάζοντας τις τεχνολογίες Ethereum και IPFS, ώστε να ορίσει έναν ψηφιακό χώρο, αποκεντρωμένο τόσο σε αρχιτεκτονικό, όσο και σε πολιτικό επίπεδο.
\\[2\baselineskip]
\textbf{Λέξεις-Κλειδιά}: Αποκέντρωση, Ψηφιακή δημοκρατία, Ethereum, Blockchain, Έξυπνα συμβόλαια, Αποκεντρωμένες εφαρμογές, IPFS