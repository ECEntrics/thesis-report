\chapter*{Σύνοψη}
\addcontentsline{toc}{chapter}{Σύνοψη}
Τις τελευταίες δεκαετίες, η ανάπτυξη του διαδικτύου μετέβαλε ριζικά τις ανθρώπινες
κοινωνίες, μέσω της παροχής μίας πληθώρας ψηφιακών υπηρεσιών. Τα μέσα κοινωνικής δικτύωσης,
το ηλεκτρονικό ταχυδρομείο, η ψηφιακή ειδησεογραφία, ο διαμοιρασμός αρχείων και
οι υπηρεσίες πολυμέσων ροής, αποτελούν ορισμένα από τα σημαντικότερα - και πλέον αναπόσπαστα - κομμάτια,
που συνθέτουν την ψηφιακή πτυχή της σύγχρονης καθημερινότητας. Στη συντριπτική τους πλειοψηφία,
τεχνολογίες όπως οι παραπάνω, προσφέρονται από παρόχους υπηρεσιών υπολογιστικού νέφους, ακολουθώντας
ως μοντέλο υλοποίησης εκείνο της αρχιτεκτονικής πελάτη-εξυπηρετητή.

Μολονότι η αρχιτεκτονική αυτή έχει αποδειχθεί ιδιαίτερα λειτουργική και έχει βελτιωθεί αξιοσημείωτα ανά τα χρόνια, η συγκεντρωτική λογική της συνοδεύεται από μία σειρά προβλημάτων. Πρώτα απ' όλα, ο χρήστης καλείται να εμπιστευθεί τα προσωπικά του δεδομένα σε μία τρίτη οντότητα. Εκείνη, διατηρώντας τον πλήρη έλεγχο επί αυτών, διαθέτει και τη δυνατότητα
να τα επεξεργάζεται, να τα διαμοιράζεται και να τα λογοκρίνει, είτε εξυπηρετώντας τα συμφέροντά της, είτε για να 
συμμορφωθεί με άλλες αρχές που της ασκούν εξουσία. Δεύτερον, απουσιάζει η εγγύηση της διαθεσιμότητας των δεδομένων, καθώς,
 ανά πάσα στιγμή, ο εξυπηρετητής μπορεί να αποσυνδεθεί για αόριστο χρονικό διάστημα και εξαιτίας ποικίλων αιτιών, όπως λόγω κάποιας
κυβερνοεπίθεσης ή κάποιας φυσικής καταστροφής.

Αυτοί είναι μερικοί βασικοί λόγοι που συντέλεσαν στην ταχεία υλοποίηση ενός συνόλου λογισμικών ανοιχτού κώδικα, όπως το Ethereum blockchain και το IPFS, τα οποία, αν και βρίσκονται σε σχετικά πρώιμο στάδιο, αποτελούν ήδη ικανά εργαλεία για τη δημιουργία κατανεμημένων και αποκεντρωμένων εφαρμογών.

Στόχος της παρούσας διπλωματικής εργασίας είναι η υλοποίηση μίας αυτόνομης κοινωνικής πλατφόρμας,
η οποία, αξιοποιώντας τεχνολογίες αποκέντρωσης, αφενός θα επιστρέφει την κυριότητα των προσωπικών
δεδομένων στον χρήστη, αφετέρου θα παρέχει τη δυνατότητα για διαφανείς δημοκρατικές ψηφοφορίες. Αυτά μέσα σε ένα πλαίσιο ανθεκτικό, τόσο σε σφάλματα και επιθέσεις, όσο και σε απόπειρες λογοκρισίας και παραποίησης.
\\[2\baselineskip]
\textbf{Λέξεις-Κλειδιά}: Αποκεντροποίηση, Ethereum, Blockchain, Έξυπνο Συμβόλαιο, Αποκεντρωμένη Εφαρμογή, IPFS, OrbitDB