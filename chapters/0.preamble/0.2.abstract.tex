\chapter*{Abstract}
\addcontentsline{toc}{chapter}{Abstract}

\textenglish{In recent decades, the rapid growth of the internet has radically transformed society, through a plethora of digital applications. The vast majority of these are offered by cloud computing service providers, using the client-server architecture.

Although this implementation model has proven to be highly functional and has improved significantly over the years, its centralized logic is accompanied by certain problems. Users are required to trust their personal data to an administrative entity, granting it full control over them. This entity gains the ability to process, share and censor the data, either to serve its own interests, or to comply with other authorities in power. Furthermore, there is no guarantee of data availability, as, at any time, the server can be disconnected indefinitely for a variety of reasons, such as a cyberattack or a natural disaster.

These are some of the key factors that have led to the rapid development of a wide range of innovative open-source software, which is based on technologies such as the blockchain and peer-to-peer networks. The aforementioned technologies, although at a relatively early stage, are already powerful tools for creating distributed and decentralized applications.

The goal of this thesis is the implementation of an autonomous social platform by utilizing decentralization technologies. This implementation will firstly return the ownership of the data to the end user and secondly provide transparent democratic voting processes. The above are provided in a context resistant to both faults and attacks, as well as attempts at censorship and falsification.

The developed proof-of-concept application “Concordia” approaches the above goal by combining Ethereum and IPFS, in order to define a digital space, that is decentralized both at architectural and political level. 
\\[2\baselineskip]
\textbf {Keywords}: Decentralization, Digital democracy, Ethereum, Blockchain, Smart contracts, Decentralized applications, IPFS}