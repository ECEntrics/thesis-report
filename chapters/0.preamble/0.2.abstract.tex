\chapter*{Abstract}
\addcontentsline{toc}{chapter}{Abstract}

In recent decades, the rapid growth of the internet has radically changed human
societies, through a plethora of digital applications, the vast majority of which are offered by cloud computing service providers, following the client-server architecture.

Although this implementation model has proven to be highly functional and has improved significantly over the years, its centralized logic is accompanied by a number of problems. First of all, the user is required to trust his personal data to an external entity. Maintaining full control over them, the latter gains the ability to process, share and censor them, either to serve its own interests or to comply with other authorities in power. In addition, there is no guarantee of data availability, as, at any time, the server can be disconnected indefinitely and for a variety of reasons, such as a cyber attack or a natural disaster.

These are some of the key factors that have led to the rapid development of a wide range of innovating open source software, that are based on technologiew such as blockchain and peer-to-peer networks. The above, although at a relatively early stage, are already powerful tools for creating distributed and decentralized applications.

The aim of this thesis is the implementation of an autonomous social platform,
which, by utilizing decentralization technologies, on the one hand will return the ownership of the staff
user data, on the other hand, will enable transparent democratic voting processes. These in a context resistant to both faults and attacks, as well as attempts at censorship and falsification.

The developed proof of concept application "Concordia" approaches the above goal by combining Ethereum and IPFS, in order to define a decentralized digital space, both at architectural and political level. 
\\[2\baselineskip]
\textbf {Keywords}: Decentralization, Ethereum, Blockchain, Smart Contract, Decentralized Application, IPFS, OrbitDB, React, Redux, Jenkins 