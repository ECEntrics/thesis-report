\section{IPFS}

Το IPFS (InterPlanetary File System) είναι \textit{ένα P2P πρωτόκολλο υπερμέσων, σχεδιασμένο για να διατηρήσει και να αυξήσει τη γνώση της ανθρωπότητας κάνοντας το διαδίκτυο αναβαθμίσιμο, ανθεκτικό και πιο ανοιχτό}. %TODO: add [https://ipfs.io/]
Πρακτικά πρόκειται για ένα κατανεμημένο σύστημα για αποθήκευση και πρόσβαση σε αρχεία, ιστότοπους, εφαρμογές και δεδομένα. Το περιεχόμενο είναι προσβάσιμο μέσω ενός δικτύου ομότιμων κόμβων που βρίσκονται οπουδήποτε στον κόσμο, οι οποίοι ενδέχεται να μεταφέρουν πληροφορία, να την αποθηκεύσουν ή και τα δύο.  %TODO: add [https://docs.ipfs.io/]
%TODO: add IPFS logo image

Ο τρόπος λειτουργίας του IPFS βασίζεται στα εξής:

\begin{itemize}
	\item \textbf{Μοναδική ταυτοποίηση μέσω διευθυνσιοδότησης περιεχομένου (content addressing)}. Το περιεχόμενο δεν προσδιορίζεται από την τοποθεσία του (π.χ. https://...), αλλά από το τι περιλαμβάνει. Κάθε κομμάτι περιεχομένου έχει ένα μοναδικό αναγνωριστικό περιεχομένου (Content ID ή CID), το οποίο είναι το hash του.
	\item \textbf{Σύνδεση περιεχομένου μέσω κατευθυνόμενων άκυκλων γράφων (Directed Acyclic Graphs ή DAGs)}. Το IPFS αξιοποιεί DAGs (και συγκεκριμένα Merkle DAGs), μίας δομής δεδομένων της οποίας κάθε κόμβος έχει ως μοναδικό αναγνωριστικό το hash του περιεχομένου του (το CID).
	%TODO: add https://proto.school/tutorial-assets/T0008L04-complete-dag.svg 
	\item \textbf{Ανακάλυψη περιεχομένου μέσω κατανεμημένων πινάκων κατακερματισμού (Distributed hash tables ή DHTs)}. Ο DHT είναι ένα κατανεμημένο σύστημα για την αντιστοίχιση κλειδιών σε τιμές. Στο IPFS αποτελεί το θεμελιώδες συστατικό του συστήματος δρομολόγησης περιεχομένου και λειτουργεί ως διασταύρωση μεταξύ καταλόγου και συστήματος πλοήγησης. Πρακτικά πρόκειται για ένα πίνακα που αποθηκεύει ποιος έχει ποια δεδομένα και, μέσω του οποίου, ο χρήστης βρίσκει τον peer που έχει αποθηκευμένο το επιθυμητό περιεχόμενο.
\end{itemize}

Ο αποκεντρωτικός χαρακτήρας του IPFS δίνει τη δυνατότητα να παρέχεται το περιεχόμενο από πολλούς κόμβους, οι οποίοι βρίσκονται σε διαφορετικές τοποθεσίες και δεν υπάγονται σε κάποια συγκεκριμένη κεντρική αρχή. Με αυτόν τον τρόπο, τα δεδομένα είναι πιο ανθεκτικά τόσο από άποψη διαθεσιμότητας (αν ένας κόμβος αποσυνδεθεί, θα υπάρχει κάποιος άλλος), όσο και από άποψη αντοχής στη λογοκρισία. Μπορεί, επίσης, να ανακτώνται γρηγορότερα, εφόσον τα διαθέτουν κάποιοι κοντινοί peers, πράγμα ιδιαίτερα πολύτιμο εάν για κοινότητες που είναι καλά δικτυωμένες τοπικά αλλά δεν έχουν καλή σύνδεση με το ευρύτερο διαδίκτυο.

%TODO: add stuff about lack of fees, pinning etc
