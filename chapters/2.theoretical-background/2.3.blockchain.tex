\section{Blockchain}

% Παλιό από Drive
Πρακτικά το blockchain είναι μία διανεμημένη δημόσια βάση δεδομένων που διατηρεί έναν αμετάβλητο κατάλογο (immutable ledger) ψηφιακών συναλλαγών (digital transactions) ενός νομίσματος (token). Ο κατάλογος αυτός παίρνει τη μορφή μιας αλυσίδας (chain) από blocks συναλλαγών που διαδέχονται το ένα το άλλο.

Ως προς την κυριότητα επί αυτής, η βάση δεν ελέγχεται από κάποια κεντρική οντότητα, αλλά αναπαράγεται σε όλους τους (πλήρεις) κόμβους (full nodes) που απαρτίζουν συλλογικά το δίκτυο. Με το δικό του κόμβο μπορεί να συμμετάσχει οποιοσδήποτε το επιθυμεί, ωστόσο δεν είναι αναγκαίο για τον χρήστη που θέλει απλά να συναλλαγεί στο δίκτυο να διαθέτει τον κόμβο του (διαθέτει απλά έναν light node).

Η λειτουργία των κόμβων (ενν. full) είναι η επικύρωση των συναλλαγών που επιθυμούν να πραγματοποιήσουν οι χρήστες. Για την επικύρωση, ανταγωνίζονται ο ένας τον άλλον για το ποιος θα λύσει πρώτος ένα σύνθετο αλγοριθμικό πρόβλημα που συσχετίζεται με το εκάστοτε block. Το κίνητρο για τη δαπάνη της επεξεργαστικής ισχύος που απαιτείται για αυτό είναι η ανταμοιβή του κόμβου που θα καταλήξει πρώτος στη λύση του προβλήματος  με ένα ποσό από tokens. Ανάλογα με τον αλγόριθμο που έχει προσυμφωνηθεί να χρησιμοποιείται στο εκάστοτε blockchain (Proof of Work, Proof of Stake κτλ.), οι κόμβοι του χαρακτηρίζονται ως miners/stakers/κ.ά. .

 Έτσι, με το που προκύψει λύση, δημιουργείται ένα νέο block στο οποίο καταγράφονται -μεταξύ άλλων- οι συναλλαγές που πραγματοποιούν οι χρήστες, μια παραπομπή στο προηγούμενο block και η λύση του αλγορίθμου.  O κόμβος που επέλυσε πρώτος το πρόβλημα επιβραβεύεται με ένα ποσό νομισμάτων, που απαρτίζεται τόσο από μια προσυμφωνηθείσα ποσότητα νομισμάτων που δημιορυργείται σε κάθε block, όσο και με τα τέλη συναλλαγής (transaction fees) που κατέβαλαν στο δίκτυο οι χρήστες που πραγματοποίησαν τις συναλλαγές.

% TODO: insert diagram

Η διευθυνσιοδότηση σε ένα blockchain επιτυγχάνεται αξιοποιώντας το PGP. Δηλαδή  αποστολείς και παραλήπτες είναι πάντα κάτοχοι των ιδιωτικών κλειδιών κάποιων δημοσίων διευθύνσεων. Ο οποιοσδήποτε μπορεί να παράγει ανά πάσα στιγμή στον υπολογιστή του ένα valid PGP ζεύγος και να λαμβάνει tokens στην παραχθείσα δημόσια διεύθυνση (το πλήθος των έγκυρων δημοσίων διευθύνσεων είναι πρακτικά ανεξάντλητο π.χ. $2^{160}$ για το Bitcoin).
