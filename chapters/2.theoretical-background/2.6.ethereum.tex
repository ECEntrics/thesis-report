\section{Ethereum} \label{section:2-6-ethereum}

\logo{chapter-2/2.6.ethereum-logo}{Ethereum logo}

Το Ethereum είναι ένα δημόσιο blockchain ανοιχτού κώδικα με εγγενές κρυπτονόμισμα το Ether (ETH). Παρέχει μία προγραμματιστική πλατφόρμα με ενσωματωμένη μία Turing-complete γλώσσα προγραμματισμού, που μπορεί να χρησιμοποιηθεί για τη δημιουργία αποκεντρωμένων εφαρμογών (Decentralized Applications ή DApps) μέσω της χρήσης "έξυπνων συμβολαίων" (smart contracts).\cite{2.6-ethereum-whitepaper}

\subsection{Λογαριασμοί} \label{subsection:2-6-1-ethereum-accounts}
Στο Ethereum blockchain οι λογαριασμοί αποτελούν οντότητες οι οποίες μπορούν να δέχονται, να κρατούν και να στέλνουν ETH και tokens, καθώς και να αλληλεπιδρούν με smart contracts.\cite{2.6-ethereum-documentation} Χωρίζονται σε δύο κατηγορίες: στους λογαριασμούς εξωτερικής κατοχής (\textenglish{externally owned accounts} ή EOAs) και στους λογαριασμούς συμβολαίων (contract accounts).

Οι λογαριασμοί εξωτερικής κατοχής δημιουργούνται χωρίς κόστος και ελέγχονται μέσω ιδιωτικών κλειδιών. Μπορούν να πραγματοποιούν μόνο συναλλαγές μεταφοράς ETH ή κάποιου token.

Από την άλλη, οι λογαριασμοί συμβολαίων απαιτούν κάποιο κόστος δημιουργίας, καθώς χρησιμοποιούν αποθηκευτικό χώρο επί του blockchain, ενώ ελέγχονται αποκλειστικά από τον κώδικά τους. Οι συναλλαγές που πραγματοποιούν προς άλλους λογαριασμούς είναι μόνο σαν αντίδραση σε μία εισερχόμενη συναλλαγή, όπως ορίζει ο κώδικας του smart contract τους.

Πιο αναλυτικά, τα πεδία που διαθέτουν οι λογαριασμοί στο Ethereum είναι τα εξής:

\begin{itemize}
	\item Το \textbf{nonce}: ένας μετρητής που υποδεικνύει τον αριθμό των απεσταλμένων συναλλαγών του λογαριασμού. Αυτό διασφαλίζει ότι οι συναλλαγές διεκπεραιώνονται μόνο μία φορά. Σε έναν λογαριασμό συμβολαίου, αυτός ο αριθμός αντιπροσωπεύει τον αριθμό των συμβολαίων που δημιουργήθηκαν από εκείνον.

	\item Το \textbf{balance}: το ποσό σε ETH που διαθέτει ο λογαριασμός, μετρημένο σε wei (1 ETH = $10^{18}$ wei).

	\item Το \textbf{codeHash}: ένα hash που αναφέρεται στον κώδικα του λογαριασμού (contract code). Είναι χαρακτηριστικό των λογαριασμών συμβολαίων (στους λογαριασμούς εξωτερικής κατοχής είναι hash μίας κενής συμβολοσειράς) και, σε αντίθεση με τα άλλα πεδία, αφότου οριστεί παραμένει αμετάβλητο.

	\item Το \textbf{storageRoot}: το root hash του δένδρου Merkle των αποθηκευμένων δεδομένων του smart contract, εφόσον πρόκειται για έναν λογαριασμό συμβολαίου (δε χρησιμοποιείται σε λογαριασμούς εξωτερικής κατοχής).
\end{itemize}

Η δημιουργία των λογαριασμών επιτυγχάνεται μέσω της παραγωγής ενός ζεύγους κλειδιών, αξιοποιώντας τον 
ECDSA (βλ. ενότητα \ref{section:2-2-asymmetric-cryptography}). Έτσι, το ιδιωτικό κλειδί χρησιμοποιείται για να υπογράφονται ψηφιακά οι συναλλαγές, ενώ το δημόσιο ορίζει τη δημόσια διεύθυνση του λογαριασμού (είναι της μορφής "0x + τα 20 τελευταία bytes του Keccak-256\footnote{Ο Keccak-256 αποτελεί τον κρυπτογραφικό αλγόριθμο κατακερματισμού που χρησιμοποειίται στο Ethereum. Πρόκειται για τον αλγόριθμο SHA3-256 πριν την τυποποίησή του από το NIST.} hash του δημόσιου κλειδιού").

Κατά κύριο λόγο, οι χρήστες παράγουν και διαχειρίζονται τα ιδιωτικά κλειδιά των λογαριασμών εξωτερικής κατοχής μέσω ενός πορτοφολιού (wallet). Τα wallets αποτελούν εφαρμογές, οι οποίες δείχνουν το balance του λογαριασμού και παρέχουν τη δυνατότητα αποστολής/ λήψης ETH και tokens από/ σε αυτόν. Ορισμένα προτοφόλια προσφέρουν περαιτέρω λειτουργίες, σημαντικότερη εκ των οποίων είναι η διασύνδεση του λογαριασμού με αποκεντρωμένες εφαρμογές. Επί του παρόντος, το πιο διαδεδομένο πορτοφόλι τέτοιου τύπου είναι το MetaMask\footnote{\url{https://metamask.io/}}.

\subsection{Smart Contracts}
Με λίγα λόγια, τα smart contracts αποτελούν αυτόνομα κομμάτια κώδικα, τα οποία είναι αποθηκευμένα στο blockchain και ενεργοποιούνται μέσω συναλλαγών. Κληρονομούν ιδιότητες του blockchain, όπως τη διαφάνεια (transparency), την επαληθευσιμότητα (verifiability) και την αμεταβλητότητα (immutability).

Ένα παράδειγμα της καθημερινότητας που μπορεί να παρομοιασθεί λειτουργικά με smart contract είναι ο αυτόματος πωλητής.\cite{2.6-ethereum-smart-contracts} Ένας αυτόματος πωλητής ορίζεται ως ένα αυτόνομο μηχάνημα που διανέμει αγαθά ή παρέχει υπηρεσίες, όταν εισαχθεί σε αυτόν κάποιο χρηματικό ποσό και επιλεχθεί ένα διαθέσιμο προϊόν. Οι αυτόματοι πωλητές είναι προγραμματισμένοι να εκτελούν συγκεκριμένους κανόνες, που θα μπορούσαν να οριστούν σε ένα συμβόλαιο. Με όμοιο τρόπο, σε ένα smart contract μπορούν να κωδικοποιηθούν αυθαίρετες συναρτήσεις μετάβασης, επιτρέποντας τη δημιουργία μίας πληθώρας αποκεντρωμένων εφαρμογών.

Όπως προαναφέρθηκε στην υποενότητα \ref{subsection:2-6-1-ethereum-accounts}, τα smart contracts εντάσσονται σε contract accounts και απαιτούν την καταβολή τελών για τη δημιουργία τους, αφού χρειάζεται να εγγράψουν επί του blockchain τον κώδικα και τα αρχικά δεδομένα τους.

Επιπλέον, ο κώδικάς τους ενεργοποιείται μονάχα όταν δεχθούν μία έγκυρη συναλλαγή από κάποιον άλλον λογαριασμό (είτε εξωτερικής κατοχής, είτε συμβολαίου). Η συναλλαγή αυτή εμπεριέχει, πέρα από το απαιτούμενο fee, ένα συνοδευτικό μήνυμα με πληροφορίες σχετικές με τη συνάρτηση που πρέπει να εκτελεστεί. Η δε συνάρτηση μπορεί να πραγματοποιεί ποικίλες διαφορετικές ενέργειες, όπως την ανάγνωση και την εγγραφή στην εσωτερική αποθήκευση, την πραγματοποίηση νέων συναλλαγών, ή, ακόμα, τη δημιουργία νέων συμβολαίων.

Η σύνταξη των smart contracts γίνεται κατά βάση σε γλώσσες υψηλού επιπέδου και, συγκεκριμένα, στις Solidity, Vyper και Fe\footnote{\url{https://soliditylang.org/}, \url{https://github.com/vyperlang/vyper} και \url{https://fe-lang.org/}}. Πριν την εγγραφή τους στο blockchain, μεταγλωττίζονται σε εμηνεύσιμο από την EVM bytecode, η οποία είναι υπεύθυνη για την αποθήκευση και την εκτέλεσή του (βλ. υποενότητα \ref{subsection:2-6-5-evm}).

\subsection{DApps}
Οι DApps στο οικοσύστημα του Ethereum είναι εφαρμογές οι οποίες συνδυάζουν \textenglish{smart contracts} και  \textenglish{frontend UIs}. Είναι ντετερμινιστικές, Turing-complete και εκτελούνται απομονωμένα στην EVM.\cite{2.6-ethereum-documentation}

Πέρα από τα θετικά χαρακτηριστικά των DApps που αναλύθηκαν στην ενότητα \ref{section:1-2-decentralization} (ανοχή σε σφάλματα, αντοχή σε επιθέσεις, απουσία ανάγκης εκχώρησης εμπιστοσύνης, αντίσταση σε συμπαιγνίες), τα Ethereum DApps διαθέτουν επιπλέον όλα τα πλεονεκτήματα των blockchain και των smart contract, όπως μηδενικό downtime, πλήρη ακεραιότητα δεδομένων και επαληθεύσιμη συμπεριφορά.

Ωστόσο, χαρακτηρίζονται και από μία σειρά αξιοσημείωτων μειονεκτημάτων, όπως τα παρακάτω:
\begin{itemize}
	\item Δυσκολία συντήρησης: Συντηρούνται δυσκολότερα από τις συγκεντρωτικές εφαρμογές, εξαιτίας της αμεταβλητότητας του κώδικα και των δεδομένων επί του blockchain. 
	\item Επιβάρυνση απόδοσης: Υπάρχει τεράστια επιβάρυνση απόδοσης (performance overhead) και η κλιμάκωση (scaling) είναι πολύ δύσκολη, καθώς απαιτείται όλοι οι κόμβοι να εκτελούν και να αποθηκεύουν όλες τις συναλλαγές.
	\item Συμφόρηση δικτύου: Επί του παρόντος, το δίκτυο μπορεί να επεξεργαστεί μόνο περίπου 10-15 συναλλαγές ανά δευτερόλεπτο. Εάν οι συναλλαγές αποστέλλονται με ταχύτερο ρυθμό από αυτόν, θα αυξάνονται παράλληλα και οι μη επιβεβαιωμένες συναλλαγές που αναμένουν να εκτελεστούν.
	\item Κακή εμπειρία χρήστη: Επί του παρόντος, είναι δύσκολο για τον μέσο τελικό χρήστη να αλληλεπιδράσει με το blockchain με ευκολία και ασφάλεια, καθώς απαιτούνται ενέργειες όπως η εγκατάσταση ειδικών εργαλείων για τη διασύνδεση με αυτό, η δημιουργία wallet, η διαφύλαξη του ιδιωτικού του κλειδιού και η προσθήκη ETH για την εξόφληση των τελών.
\end{itemize}

Παρ' όλα τα μειονεκτήματα, τα οποία μετριάζονται με τον καιρό μέσω συνεχών αναβαθμίσεων της πλατφόρμας, υπάρχει ήδη ένα ευρύ φάσμα εφαρμογών που μπορούν να υλοποιηθούν στο Ethereum, αξιοποιώντας τα ισχυρά του πλεονεκτήματα. Οι εφαρμογές αυτές μπορούν να διακριθούν σε τρεις κατηγορίες:
\begin{enumerate}
	\item Οικονομικές εφαρμογές, οι οποίες παρέχουν στους χρήστες ισχυρούς τρόπους διαχείρισης και σύναψης συμβάσεων χρησιμοποιώντας τα χρήματά τους. Αυτό περιλαμβάνει υπονομίσματα, χρηματοοικονομικά παράγωγα, συμβάσεις αντιστάθμισης κινδύνου, πορτοφόλια αποταμίευσης, διαθήκες και ακόμα και ορισμένες κατηγορίες συμβάσεων εργασίας πλήρους κλίμακας.
	
	\item Ημι-οικονομικές εφαρμογές, όπου εμπλέκονται χρήματα, αλλά η λειτουργία τους εμπεριέχει παράλληλα και μία αξιοσημείωτη μη νομισματική πλευρά. Ένα τέτοιο παράδειγμα είναι οι αυτόματες πληρωμές για λύσεις σε υπολογιστικά προβλήματα (βλ. Gitcoin).
	
	\item Μη οικονομικές εφαρμογές, όπως η αποκεντρωμένη αποθήκευση δεδομένων, συστήματα ταυτότητας (identity) και φήμης (reputation), διαδικτυακές ψηφοφορίες και αποκεντρωμένη διακυβέρνηση. Οι τελευταίες εισάγουν και την έννοια των Αποκεντρωμένων Αυτόνομων Οργανώσεων (Decentralized Autonomous Organizations ή DAOs), οντοτήτων που ενεργούν αυτόνομα, χωρίς την ανάγκη διαμεσολάβησης κάποιας συγκεντρωτικής (\textenglish{centralized}) αρχής.\cite{2.6-ethereum-whitepaper}
\end{enumerate}

\subsection{Tokens}

Η λέξη "token" προέρχεται από τη λέξη "tācen" των Παλαιών Αγγλικών και σημαίνει σημάδι ή
σύμβολο. Στην καθημερινότητα, ο όρος χρησιμοποιείται, κατά κύριο λόγο, για την περιγραφή αντικειμένων ειδικού σκοπού, τα οποία μοιάζουν με κέρματα και έχουν ασήμαντη εγγενή αξία (π.χ. μάρκες παιχνιδιών). Συνήθως, η χρήση των tokens περιορίζεται σε συγκεκριμένες επιχειρήσεις, οργανισμούς ή τοποθεσίες, πράγμα το οποίο σημαίνει ότι δεν ανταλλάσσονται εύκολα και τυπικά έχουν μόνο μία λειτουργία.\cite{2.6-ethereum-mastering}

Ωστόσο, στο Ethereum blockchain η έννοια των tokens επεκτείνεται και επαναπροσδιορίζεται. Πλέον δεν υπάρχουν περιορισμοί χρήσης και ιδιοκτησίας, ενώ η αξία τους ποικίλει, ανάλογα με το τι αντιπροσωπεύουν (π.χ. νομίσματα, περιουσιακά στοιχεία, ή δικαιώματα πρόσβασης). Μπορούν, δε, να ανταλλαχθούν σε παγκόσμιο επίπεδο, για άλλα tokens ή για άλλα νομίσματα.

Τα tokens που ορίζονται σε Ethereum smart contracts μπορούν να έχουν μία ή περισσότερες από τις παρακάτω χρήσεις:

\begin{itemize}
	\item Νόμισμα (currency)
	\item Πόρος (resource), π.χ. για το διαμοιρασμό CPU ή storage εντός ενός δικτύου
	\item Περιουσιακό στοιχείο (asset), εγγενούς ή εξωγενούς, υλικού
ή άυλου (π.χ. χρυσός, πετρέλαιο, ενέργεια, ακίνητο)
	\item Πρόσβαση (access), ψηφιακή ή
φυσική (π.χ. forum, δωμάτιο ξενοδοχείου)
	\item Μετοχικό κεφάλαιο (equity) ενός ψηφιακού οργανισμού (π.χ. μίας DAO) ή μίας νομικής οντότητας (π.χ. μίας εταιρείας)
	\item Ψηφοφορία (voting), παρέχοντας δικαιώματα ψήφου σε ένα ψηφιακό ή νομικό σύστημα
	\item Συλλεκτικό (collectible), ψηφιακό ή φυσικό
	\item Ταυτότητα (identity), ψηφιακής ή νομικής φύσεως
	\item Πιστοποίηση (attestation), π.χ. πτυχίο, πιστοποιητικό γέννησης
	\item Χρησιμότητα (utility), για πρόσβαση ή πληρωμή μίας υπηρεσίας
\end{itemize}

Ένα σημαντικό κριτήριο κατηγοριοποίησης των tokens είναι η εναλλαξιμότητα (fungibility) τους. Ένα token είναι εναλλάξιμο όταν οποιαδήποτε μονάδα του μπορεί να αντικατασταθεί με μία άλλη χωρίς καμία διαφορά στην αξία ή τη λειτουργία του. Από την άλλη πλευρά, ένα non-fungible token (NFT) αντιπροσωπεύει ένα μοναδικό υλικό ή άυλο στοιχείο και, επομένως, είναι μη εναλλάξιμο.

Τέλος, τα tokens συχνά ακολουθούν κάποια καθορισμένα standards στην υλοποίησή τους. Τα δημοφιλέστερα από αυτά είναι τα ERC-20 και ERC-777 (για fungible tokens), το ERC-721 (για NFTs) και το ERC-1155 (για semi-fungible tokens ή SFTs).

\subsection{EVM} \label{subsection:2-6-5-evm}
Τα smart contracts (και, κατ' επέκταση, οι DApps) εκτελούνται από την εικονική μηχανή του Ethereum (Ethereum Virtual Machine ή EVM). Η EVM αποτελεί μία quasi\footnote{"Quasi" ("σχεδόν") επειδή όλες οι διαδικασίες εκτέλεσης περιορίζονται σε έναν πεπερασμένο αριθμό υπολογιστικών βημάτων από την ποσότητα gas που είναι διαθέσιμη για οποιαδήποτε εκτέλεση ενός smart contract.}–Turing-complete μηχανή καταστάσεων αρχιτεκτονικής βασισμένης σε στοίβα (stack-based architecture). Σε υψηλό επίπεδο, η EVM μπορεί να θεωρηθεί ως ένας παγκόσμιος αποκεντρωμένος υπολογιστής που περιέχει εκατομμύρια εκτελέσιμα αντικείμενα, το καθένα με τη δική του μόνιμη αποθήκη δεδομένων.

Η EVM αποθηκεύει όλες τις τιμές της μνήμης σε μια στοίβα και λειτουργεί με μέγεθος λέξης 256 bit, κυρίως για τη διευκόλυνση των εγγενών λειτουργιών κατακερματισμού και ελλειπτικής καμπύλης. Διαθέτει ένα σύνολο διευθυνσιοδοτήσιμων στοιχείων δεδομένων:

\begin{itemize}
	\item Έναν αμετάβλητο κώδικα προγράμματος σε εικονική μνήμη ROM, φορτωμένο με τον \textenglish{bytecode} του smart contract προς εκτέλεση.
	\item Μία πτητική (volatile) μνήμη, με κάθε θέση ρητά αρχικοποιημένη στο μηδέν.
	\item Ένας χώρος μόνιμης αποθήκευσης, που είναι μέρος του Ethereum state, επίσης αρχικά μηδενισμένος.
\end{itemize}

Υπάρχει, επίσης, ένα σύνολο μεταβλητών περιβάλλοντος και δεδομένων, τα οποία είναι διαθέσιμα κατά την εκτέλεση.\cite{2.6-ethereum-mastering}
