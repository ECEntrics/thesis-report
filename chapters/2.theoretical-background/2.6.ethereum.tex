\section{Ethereum}

\begin{figure}[H]
	\centering
	\includegraphics[width=2cm]{ethereum-logo.png}
	\caption{Ethereum logo}
\end{figure}

Το Ethereum είναι ένα δημόσιο blockchain ανοιχτού κώδικα με εγγενές κρυπτονόμισμα το Ether (ETH). Παρέχει μία προγραμματιστική πλατφόρμα με ενσωματωμένη μία Turing-complete γλώσσα προγραμματισμού, που μπορεί να χρησιμοποιηθεί για τη δημιουργία αποκεντρωμένων εφαρμογών (Decentralized Applications ή DApps) μέσω της χρήσης "έξυπνων συμβολαίων" (smart contracts).\cite{2.6-ethereum-whitepaper}

\subsection{Smart Contracts \& DApps}
Με απλά λόγια, τα smart contracts αποτελούν αυτόνομα κομμάτια κώδικα τα οποία είναι αποθηκευμένα στο blockchain και ενεργοποιούνται μέσω συναλλαγών. Μπορούν να διαβάζουν και να γράφουν δεδομένα επί του blockchain, κληρονομώντας ιδιότητες όπως τη διαφάνεια (transparency), την εγκυρότητα (validability) και την αμεταβλητότητα (immutability).

Ένα παράδειγμα της καθημερινότητας που μπορεί να παρομοιασθεί λειτουργικά με smart contract είναι ο αυτόματος πωλητής\cite{2.6-ethereum-smart-contracts}. Ένας αυτόματος πωλητής ορίζεται ως ένα αυτόνομο μηχάνημα που διανέμει αγαθά ή παρέχει υπηρεσίες όταν εισάγονται σε αυτόν κέρματα ή κάποια ηλεκτρονική πληρωμή. Οι αυτόματοι πωλητές είναι προγραμματισμένοι να εκτελούν συγκεκριμένους κανόνες που θα μπορούσαν να οριστούν σε ένα συμβόλαιο.

Με όμοιο τρόπο, σε ένα smart contract μπορούν να κωδικοποιηθούν αυθαίρετες συναρτήσεις μετάβασης, επιτρέποντας τη δημιουργία ποικίλων αποκεντρωμένων εφαρμογών. Οι εφαρμογές αυτές μπορούν να χωριστούν σε τρεις κατηγορίες:
\begin{itemize}
	\item Οικονομικές εφαρμογές, οι οποίες παρέχουν στους χρήστες ισχυρότερους τρόπους διαχείρισης και σύναψης συμβάσεων χρησιμοποιώντας τα χρήματά τους. Αυτό περιλαμβάνει υπο-νομίσματα, χρηματοοικονομικά παράγωγα, συμβάσεις αντιστάθμισης κινδύνου, πορτοφόλια αποταμίευσης, διαθήκες, και, τελικά, ακόμη και ορισμένες κατηγορίες συμβάσεων εργασίας πλήρους κλίμακας.
	
	\item Ημι-οικονομικές εφαρμογές, όπου εμπλέκονται χρήματα, αλλά η λειτουργία τους εμπεριέχει παράλληλα και  μία αξιοσημείωτη μη νομισματική πλευρά. Ένα τέτοιο παράδειγμα είναι οι αυτόματες πληρωμές για λύσεις σε υπολογιστικά προβλήματα (βλ. Gitcoin).
	
	\item Μη οικονομικές εφαρμογές, όπως η αποκεντρωμένη αποθήκευση δεδομένων, συστήματα ταυτότητας (identity) και φήμης (reputation), διαδικτυακές ψηφοφορίες και αποκεντρωμένη διακυβέρνηση. Οι τελευταίες εισάγουν και την έννοια των Αποκεντρωμένων Αυτόνομων Οργανώσεων (Decentralized Autonomous Organizations ή DAOs), οντοτήτων που ενεργούν αυτόνομα, χωρίς την ανάγκη διαμεσολάβησης κάποιας συγκεντρωτικής (centralized) αρχής.\cite{2.6-ethereum-whitepaper}
\end{itemize}

\subsection{EVM}
Τα smart contracts εκτελούνται από την εικονική μηχανή του Ethereum (Ethereum Virtual Machine ή EVM). Η EVM αποτελεί μία quasi\footnote{"quasi" επειδή όλες οι διαδικασίες εκτέλεσης περιορίζονται σε έναν πεπερασμένο αριθμό υπολογιστικών βημάτων από την ποσότητα gas που είναι διαθέσιμη για οποιαδήποτε εκτέλεση ενός smart contract.}–Turing-complete μηχανή καταστάσεων αρχιτεκτονικής βασισμένης σε στοίβα (stack-based architecture). Σε υψηλό επίπεδο, η EVM μπορεί να θεωρηθεί ως ένας παγκόσμιος αποκεντρωμένος υπολογιστής που περιέχει εκατομμύρια εκτελέσιμα αντικείμενα, το καθένα με τη δική του μόνιμη αποθήκη δεδομένων.

Η EVM αποθηκεύει όλες τις τιμές της μνήμης σε μια στοίβα και λειτουργεί με μέγεθος λέξης 256 bit, κυρίως για τη διευκόλυνση των εγγενών λειτουργιών κατακερματισμού και ελλειπτικής καμπύλης. Διαθέτει ένα σύνολο διευθυνσιοδοτήσιμων στοιχείων δεδομένων:

\begin{itemize}
	\item Έναν αμετάβλητο κώδικα προγράμματος σε εικονική μνήμη ROM, φορτωμένο με τον bytecode του smart contract προς εκτέλεση.
	\item Μία πτητική (volatile) μνήμη, με κάθε θέση ρητά αρχικοποιημένη στο μηδέν.
	\item Ένας χώρος μόνιμης αποθήκευσης, που είναι μέρος του Ethereum state, επίσης αρχικά μηδενισμένος.
\end{itemize}

Υπάρχει, επίσης, ένα σύνολο μεταβλητών περιβάλλοντος και δεδομένων, που είναι διαθέσιμα κατά την εκτέλεση.\cite{2.6-ethereum-mastering}

% TODO: Add account types, addressing, token types, fees,...
