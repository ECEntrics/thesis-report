\section{Ethereum}

Το Ethereum είναι ένα δημόσιο blockchain ανοιχτού κώδικα με εγγενές κρυπτονόμισμα το Ether (ETH). Παρέχει μία προγραμματιστική πλατφόρμα με ενσωματωμένη μία Turing-complete γλώσσα προγραμματισμού, που μπορεί να χρησιμοποιηθεί για τη δημιουργία αποκεντρωμένων εφαρμογών (Decentralized Applications ή DApps) μέσω της χρήσης "έξυπνων συμβολαίων" (smart contracts).

%TODO: add [https://ethereum.org/en/whitepaper/]
%TODO: add Ethereum logo image

Με απλά λόγια, τα smart contracts αποτελούν αυτόνομα κομμάτια κώδικα τα οποία είναι αποθηκευμένα στο blockchain και ενεργοποιούνται μέσω συναλλαγών. Μπορούν να διαβάζουν και να γράφουν δεδομένα επί του blockchain, κληρονομώντας ιδιότητες όπως τη διαφάνεια (transparency), την εγκυρότητα (validability) και την αμεταβλητότητα (immutability).

Ένα παράδειγμα της καθημερινότητας που μπορεί να παρομοιασθεί λειτουργικά με smart contract είναι ο αυτόματος πωλητής %TODO: add [https://publications.hse.ru/mirror/pubs/share/folder/8uijw5g6qm/direct/199179205.pdf]
. Ένας αυτόματος πωλητής ορίζεται ως ένα αυτόνομο μηχάνημα που διανέμει αγαθά ή παρέχει υπηρεσίες όταν εισάγονται σε αυτόν κέρματα ή κάποια ηλεκτρονική πληρωμή. Οι αυτόματοι πωλητές είναι προγραμματισμένοι να εκτελούν συγκεκριμένους κανόνες που θα μπορούσαν να οριστούν σε ένα συμβόλαιο.

Με όμοιο τρόπο σε ένα smart contract μπορούν να κωδικοποιηθούν αυθαίρετες συναρτήσεις μετάβασης, επιτρέποντας τη δημιουργία ποικίλων αποκεντρωμένων εφαρμογών. Οι εφαρμογές αυτές μπορούν να χωριστούν σε τρεις κατηγορίες:
\begin{itemize}
	\item Οικονομικές εφαρμογές, οι οποίες παρέχοντας στους χρήστες ισχυρότερους τρόπους διαχείρισης και σύναψης συμβάσεων χρησιμοποιώντας τα χρήματά τους. Αυτό περιλαμβάνει υπο-νομίσματα, χρηματοοικονομικά παράγωγα, συμβάσεις αντιστάθμισης κινδύνου, πορτοφόλια αποταμίευσης, διαθήκες, και, τελικά, ακόμη και ορισμένες κατηγορίες συμβάσεων εργασίας πλήρους κλίμακας.
	
	\item Ημι-οικονομικές εφαρμογές, όπου εμπλέκονται χρήματα, αλλά η λειτουργία τους εμπεριέχει παράλληλα και  μία αξιοσημείωτη μη νομισματική πλευρά. Ένα τέτοιο παράδειγμα είναι οι αυτόματες πληρωμές για λύσεις σε υπολογιστικά προβλήματα (βλ. Gitcoin).
	
	\item Μη οικονομικές εφαρμογές, όπως η αποκεντρωμένη αποθήκευση δεδομένων, συστήματα ταυτότητας (identity) και φήμης (reputation), διαδικτυακές ψηφοφορίες και αποκεντρωμένη διακυβέρνηση. Οι τελευταίες εισάγουν και την έννοια των Αποκεντρωμένων Αυτόνομων Οργανώσεων (Decentralized Autonomous Organizations ή DAOs), οντοτήτων που ενεργούν αυτόνομα, χωρίς την ανάγκη διαμεσολάβησης κάποιας συγκεντρωτικής (centralized) αρχής.
\end{itemize}
%TODO: add [https://ethereum.org/en/whitepaper/]
