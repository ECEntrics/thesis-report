\section{Ethereum} \label{section:2-6-ethereum}

\begin{figure}[H]
	\centering
	\includegraphics[width=2cm]{assets/figures/ethereum-logo.png}
	\caption{Ethereum logo}
\end{figure}

Το Ethereum είναι ένα δημόσιο blockchain ανοιχτού κώδικα με εγγενές κρυπτονόμισμα το Ether (ETH). Παρέχει μία προγραμματιστική πλατφόρμα με ενσωματωμένη μία Turing-complete γλώσσα προγραμματισμού, που μπορεί να χρησιμοποιηθεί για τη δημιουργία αποκεντρωμένων εφαρμογών (Decentralized Applications ή DApps) μέσω της χρήσης "έξυπνων συμβολαίων" (smart contracts).\cite{2.6-ethereum-whitepaper}

\subsection{Λογαριασμοί} \label{subsection:2-6-1-ethereum-accounts}
Στο Ethereum blockchain οι λογαριασμοί αποτελούν οντότητες οι οποίες μπορούν να δέχονται, να κρατούν και να στέλνουν ETH και tokens, καθώς και να αλληλεπιδρούν με smart contracts.\cite{2.6-ethereum-documentation} Χωρίζονται σε δύο κατηγορίες: στους λογαριασμούς εξωτερικής κατοχής (externally owned accounts ή EOAs) και στους λογαριασμούς συμβολαίων (contract accounts).

Οι λογαριασμοί εξωτερικής κατοχής δημιουργούνται χωρίς κόστος και ελέγχονται μέσω ιδιωτικών κλειδιών. Μπορούν να πραγματοποιούν μόνο συναλλαγές μεταφοράς ETH ή κάποιου token.

Από την άλλη, οι λογαριασμοί συμβολαίων απαιτούν κάποιο κόστος δημιουργίας, καθώς χρησιμοποιούν αποθηκευτικό χώρο επί του blockchain, ενώ ελέγχονται αποκλειστικά από τον κώδικά τους. Οι συναλλαγές που πραγματοποιούν προς άλλους λογαριασμούς είναι μόνο σαν αντίδραση σε μία εισερχόμενη συναλλαγή, όπως ορίζει ο κώδικας του smart contract τους.

Πιο αναλυτικά, τα πεδία που διαθέτουν οι λογαριασμοί στο Ethereum είναι τα εξής:

\begin{itemize}
	\item Το \textbf{nonce}: ένας μετρητής που υποδεικνύει τον αριθμό των απεσταλμένων συναλλαγών του λογαριασμού. Αυτό διασφαλίζει ότι οι συναλλαγές διεκπεραιώνονται μόνο μία φορά. Σε έναν λογαριασμό συμβολαίου, αυτός ο αριθμός αντιπροσωπεύει τον αριθμό των συμβολαίων που δημιουργήθηκαν από εκείνον.

	\item Το \textbf{balance}: το ποσό σε ETH που διαθέτει ο λογαριασμός, μετρημένο σε wei (1 ETH = $10^{18}$ wei).

	\item Το \textbf{codeHash}: ένα hash που αναφέρεται στον κώδικα του λογαριασμού (contract code). Είναι χαρακτηριστικό των λογαριασμών συμβολαίων (στους λογαριασμούς εξωτερικής κατοχής είναι hash μίας κενής συμβολοσειράς) και, σε αντίθεση με τα άλλα πεδία, αφότου οριστεί παραμένει αμετάβλητο.

	\item Το \textbf{storageRoot}: το root hash του δένδρου Merkle των αποθηκευμένων δεδομένων του smart contract σε έναν λογαριασμό συμβολαίου (δε χρησιμοποιείται σε λογαριασμούς εξωτερικής κατοχής).
\end{itemize}

Η δημιουργία των λογαριασμών επιτυγχάνεται μέσω της παραγωγής ενός ζεύγους κλειδιών, αξιοποιώντας τον 
ECDSA (βλ. ενότητα \ref{section:2-2-asymmetric-cryptography}). Έτσι, το ιδιωτικό κλειδί χρησιμοποιείται για να υπογράφονται ψηφιακά οι συναλλαγές, ενώ το δημόσιο ορίζει τη δημόσια διεύθυνση του λογαριασμού (είναι της μορφής "0x + τα 20 τελευταία bytes του Keccak-256 hash του δημόσιου κλειδιού").

\subsection{Smart Contracts}
Με λίγα λόγια, τα smart contracts αποτελούν αυτόνομα κομμάτια κώδικα, τα οποία είναι αποθηκευμένα στο blockchain και ενεργοποιούνται μέσω συναλλαγών. Κληρονομούν ιδιότητες του blockchain, όπως τη διαφάνεια (transparency), την εγκυρότητα (validability) και την αμεταβλητότητα (immutability).

Ένα παράδειγμα της καθημερινότητας που μπορεί να παρομοιασθεί λειτουργικά με smart contract είναι ο αυτόματος πωλητής.\cite{2.6-ethereum-smart-contracts} Ένας αυτόματος πωλητής ορίζεται ως ένα αυτόνομο μηχάνημα που διανέμει αγαθά ή παρέχει υπηρεσίες, όταν εισαχθεί σε αυτόν κάποιο χρηματικό ποσό και επιλεχθεί ένα διαθέσιμο προϊόν. Οι αυτόματοι πωλητές είναι προγραμματισμένοι να εκτελούν συγκεκριμένους κανόνες που θα μπορούσαν να οριστούν σε ένα συμβόλαιο. Με όμοιο τρόπο, σε ένα smart contract μπορούν να κωδικοποιηθούν αυθαίρετες συναρτήσεις μετάβασης, επιτρέποντας τη δημιουργία μίας πληθώρας αποκεντρωμένων εφαρμογών.

Όπως προαναφέρθηκε στην υποενότητα \ref{subsection:2-6-1-ethereum-accounts}, τα smart contracts εντάσσονται σε contract accounts και απαιτούν την καταβολή τελών για τη δημιουργία τους, αφού χρειάζεται να εγγράψουν επί του blockchain τον κώδικα και τα αρχικά δεδομένα τους.

Επιπλέον, ο κώδικάς τους ενεργοποιείται μονάχα όταν δεχθούν μία έγκυρη συναλλαγή από κάποιον άλλον λογαριασμό (είτε εξωτερικής κατοχής, είτε συμβολαίου). Η συναλλαγή αυτή εμπεριέχει, πέρα από το απαιτούμενο fee, ένα συνοδευτικό μήνυμα με πληροφορίες σχετικές με τη συνάρτηση που πρέπει να εκτελεστεί. Η δε συνάρτηση μπορεί να πραγματοποιεί ποικίλες διαφορετικές ενέργειες, όπως την ανάγνωση και την εγγραφή στην εσωτερική αποθήκευση, την πραγματοποίηση νέων συναλλαγών, ή, ακόμα, τη δημιουργία νέων συμβολαίων.

Η σύνταξη των smart contracts γίνεται κατά βάση σε γλώσσες υψηλού επιπέδου και, συγκεκριμένα, στις Solidity και Vyper. Πριν την εγγραφή τους στο blockchain, μεταγλωττίζονται σε εμηνεύσιμο από την EVM bytecode, η οποία είναι υπεύθυνη για την αποθήκευση και την εκτέλεσή του (βλ. υποενότητα \ref{subsection:2-6-5-evm}).

\subsection{DApps}
Οι DApps στο οικοσύστημα του Ethereum είναι εφαρμογές οι οποίες συνδυάζουν smart contracts και frontend UIs.\cite{2.6-ethereum-documentation}

% TODO: add stuff from https://ethereum.org/en/developers/docs/dapps/

Μπορούν να χωριστούν σε τρεις κατηγορίες:
\begin{enumerate}
	\item Οικονομικές εφαρμογές, οι οποίες παρέχουν στους χρήστες ισχυρούς τρόπους διαχείρισης και σύναψης συμβάσεων χρησιμοποιώντας τα χρήματά τους. Αυτό περιλαμβάνει υπο-νομίσματα, χρηματοοικονομικά παράγωγα, συμβάσεις αντιστάθμισης κινδύνου, πορτοφόλια αποταμίευσης, διαθήκες, και, τελικά, ακόμη και ορισμένες κατηγορίες συμβάσεων εργασίας πλήρους κλίμακας.
	
	\item Ημι-οικονομικές εφαρμογές, όπου εμπλέκονται χρήματα, αλλά η λειτουργία τους εμπεριέχει παράλληλα και  μία αξιοσημείωτη μη νομισματική πλευρά. Ένα τέτοιο παράδειγμα είναι οι αυτόματες πληρωμές για λύσεις σε υπολογιστικά προβλήματα (βλ. Gitcoin).
	
	\item Μη οικονομικές εφαρμογές, όπως η αποκεντρωμένη αποθήκευση δεδομένων, συστήματα ταυτότητας (identity) και φήμης (reputation), διαδικτυακές ψηφοφορίες και αποκεντρωμένη διακυβέρνηση. Οι τελευταίες εισάγουν και την έννοια των Αποκεντρωμένων Αυτόνομων Οργανώσεων (Decentralized Autonomous Organizations ή DAOs), οντοτήτων που ενεργούν αυτόνομα, χωρίς την ανάγκη διαμεσολάβησης κάποιας συγκεντρωτικής (\textenglish{centralized}) αρχής.\cite{2.6-ethereum-whitepaper}
\end{enumerate}

\subsection{Tokens}
% TODO

\subsection{EVM} \label{subsection:2-6-5-evm}
Τα smart contracts (και, κατ' επέκταση, οι DApps) εκτελούνται από την εικονική μηχανή του Ethereum (Ethereum Virtual Machine ή EVM). Η EVM αποτελεί μία quasi\footnote{"Quasi" επειδή όλες οι διαδικασίες εκτέλεσης περιορίζονται σε έναν πεπερασμένο αριθμό υπολογιστικών βημάτων από την ποσότητα gas που είναι διαθέσιμη για οποιαδήποτε εκτέλεση ενός smart contract.}–Turing-complete μηχανή καταστάσεων αρχιτεκτονικής βασισμένης σε στοίβα (stack-based architecture). Σε υψηλό επίπεδο, η EVM μπορεί να θεωρηθεί ως ένας παγκόσμιος αποκεντρωμένος υπολογιστής που περιέχει εκατομμύρια εκτελέσιμα αντικείμενα, το καθένα με τη δική του μόνιμη αποθήκη δεδομένων.

Η EVM αποθηκεύει όλες τις τιμές της μνήμης σε μια στοίβα και λειτουργεί με μέγεθος λέξης 256 bit, κυρίως για τη διευκόλυνση των εγγενών λειτουργιών κατακερματισμού και ελλειπτικής καμπύλης. Διαθέτει ένα σύνολο διευθυνσιοδοτήσιμων στοιχείων δεδομένων:

\begin{itemize}
	\item Έναν αμετάβλητο κώδικα προγράμματος σε εικονική μνήμη ROM, φορτωμένο με τον \textenglish{bytecode} του smart contract προς εκτέλεση.
	\item Μία πτητική (volatile) μνήμη, με κάθε θέση ρητά αρχικοποιημένη στο μηδέν.
	\item Ένας χώρος μόνιμης αποθήκευσης, που είναι μέρος του Ethereum state, επίσης αρχικά μηδενισμένος.
\end{itemize}

Υπάρχει, επίσης, ένα σύνολο μεταβλητών περιβάλλοντος και δεδομένων, τα οποία είναι διαθέσιμα κατά την εκτέλεση.\cite{2.6-ethereum-mastering}
