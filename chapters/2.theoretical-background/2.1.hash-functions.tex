\section{Συναρτήσεις κατακερματισμού}

% Παλιό από Drive
Οι κρυπτογραφικές συναρτήσεις κατακερματισμού (cryptographic hash functions) είναι ειδική κατηγορία συναρτήσεων κατακερματισμού σχεδιασμένες για χρήση στην κρυπτογραφία. Οι συναρτήσεις κατακερματισμού είναι μαθηματικές συναρτήσεις που δέχονται ως είσοδο δεδομένα τυχαίου μεγέθους και επιστρέφουν συμβολοσειρές σταθερού μήκους αναπαράστασης.

% TODO: insert diagram

Οι τιμές που επιστρέφει η συνάρτηση κατακερματισμού ονομάζονται τιμές κατακερματισμού (hash values ή πιο απλά hashes). Μια ιδανική κρυπτογραφική συνάρτηση κατακερματισμού έχει τις εξής βασικές ιδιότητες:

\begin{itemize}
    \item Είναι ντετερμινιστική, έτσι η ίδια είσοδος παράγει πάντα την ίδια έξοδο.
    \item Ο υπολογισμός των hashes για οποιοδήποτε μήνυμα είναι γρήγορος.
    \item Δεν είναι εφικτό να βρεις την είσοδο από το hash, παρά μόνο δοκιμάζοντας όλα τα πιθανά μηνύματα.
    \item Μία μικρή αλλαγή στο μήνυμα επιφέρει τόσο μεγάλες αλλαγές στο νέο παραγόμενο hash ώστε να φαίνεται ασυσχέτιστο με το παλιό.
    \item Δεν είναι εφικτό να βρεθούν δύο διαφορετικές είσοδοι που δίνουν το ίδιο hash.
\end{itemize}

% TODO: insert diagram
