\section{Ανοιχτά θέματα}\label{section:5-4-open-areas}

\subsection{Τέλη Ethereum}\label{subsection:5-4-1-ethereum-fees}

Οι ανάγκες κάθε υπολογιστικού συστήματος σε πόρους που σχετίζονται με τις διάφορες λειτουργίες του (π.χ. επεξεργασία, αποθήκευση δεδομένων, δίκτυα) μεταφράζονται σε κάποιο οικονομικό κόστος. Στην περίπτωση της παρούσας εφαρμογής, ενώ η αποθήκευση των δεδομένων διαμοιράζεται αυτοβούλως ανάμεσα στους συμμετέχοντες κόμβους, η χρήση του Ethereum απαιτεί από τα μέλη την καταβολή τελών για τη δημιουργία συναλλαγών. Αν και αυτά τα τέλη είναι απαραίτητα για τη λειτουργία του blockchain και την προάσπισή του από επιθέσεις, αποτελούν ισχυρό εμπόδιο για την ένταξη των τελικών χρηστών στο οικοσύστημα των αποκεντρωμένων εφαρμογών του Ethereum.

Στα πλαίσια της εφαρμογής "Concordia", η λήψη μέτρων για τη διαχείριση των τελών θεωρείται υψίστης σημασίας. Ωστόσο, η συμπερίληψη ενός τέτοιου μηχανισμού θα περιέπλεκε εξαιρετικά τον σχεδιασμό της και, ως εκ τούτου, λήφθηκε η απόφαση να συμπεριληφθεί ως πρόταση για μελλοντική της επέκταση. Ένας τέτοιος μηχανισμός θα παρείχε τη δυνατότητα στα μέλη της πλατφόρμας να τη χρησιμοποιούν χωρίς να κατέχουν ή να δαπανούν ETH. Αυτό θα μπορούσε να ήταν εφικτό μέσω της δημιουργίας μίας meta transaction, υπογεγραμμένης από τον Ethereum λογαριασμό τους, η οποία θα μεταβίβαζε την αποπληρωμή των τελών στις κοινότητες που ανήκουν οι χρήστες.

Αυτή τη στιγμή υπάρχουν ήδη προσεγγίσεις υλοποιήσεων τέτοιου είδους μηχανισμών, όπως το Gas Station Network\footnote{\url{https://opengsn.org/}}, ενώ η προγραμματιστική ομάδα του Ethereum εργάζεται ενεργά για την εγγενή υποστήριξη αυτή της δυνατότητας από την ίδια πλατφόρμα.

%TODO: add
%1. reputation system
%2. voting types
%3. token distribution
%4. ethereum, ipfs, move to proof of stake, remove of rendezvous server
