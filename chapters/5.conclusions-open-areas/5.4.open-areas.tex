\section{Ανοιχτά θέματα}\label{section:5-4-open-areas}

\subsection{Τέλη Ethereum}\label{subsection:5-4-1-ethereum-fees}

Οι ανάγκες κάθε υπολογιστικού συστήματος σε πόρους που σχετίζονται με τις διάφορες λειτουργίες του (π.χ. επεξεργασία, αποθήκευση δεδομένων, δίκτυα) μεταφράζονται σε κάποιο οικονομικό κόστος. Στην περίπτωση της παρούσας εφαρμογής, ενώ η αποθήκευση των δεδομένων διαμοιράζεται αυτοβούλως ανάμεσα στους συμμετέχοντες κόμβους, η χρήση του Ethereum απαιτεί από τα μέλη την καταβολή τελών για τη δημιουργία συναλλαγών. Αν και αυτά τα τέλη είναι απαραίτητα για τη λειτουργία του blockchain και την προάσπισή του από επιθέσεις, αποτελούν ισχυρό εμπόδιο για την ένταξη των τελικών χρηστών στο οικοσύστημα των αποκεντρωμένων εφαρμογών του Ethereum.

Στα πλαίσια της εφαρμογής Concordia, η λήψη μέτρων για τη διαχείριση των τελών θεωρείται υψίστης σημασίας. Ωστόσο, η συμπερίληψη ενός τέτοιου μηχανισμού θα περιέπλεκε εξαιρετικά τον σχεδιασμό της και, ως εκ τούτου, λήφθηκε η απόφαση να συμπεριληφθεί ως πρόταση για μελλοντική της επέκταση. Ένας τέτοιος μηχανισμός θα παρείχε τη δυνατότητα στα μέλη της πλατφόρμας να τη χρησιμοποιούν χωρίς να κατέχουν ή να δαπανούν ETH. Αυτό θα μπορούσε να ήταν εφικτό μέσω της δημιουργίας μίας meta transaction, υπογεγραμμένης από τον Ethereum λογαριασμό τους, η οποία θα μεταβίβαζε την αποπληρωμή των τελών στις κοινότητες που ανήκουν οι χρήστες.

Αυτή τη στιγμή υπάρχουν ήδη προσεγγίσεις υλοποιήσεων τέτοιου είδους μηχανισμών, όπως το Gas Station Network\footnote{\url{https://opengsn.org/}}, ενώ η προγραμματιστική ομάδα του Ethereum εργάζεται ενεργά για την εγγενή υποστήριξη αυτή της δυνατότητας από την ίδια πλατφόρμα.

\subsection{Διανομή των Ethereum token}\label{subsection:5-4-1-ethereum-fees}

Στον φυσικό κόσμο, η έγκυρη και ανώνυμη διανομή ενός συνόλου μοναδικών πιστοποιητικών αυθεντικοποίησης στα μέλη μίας κοινότητας θα μπορούσε να ήταν μία διαδικασία, η οποία θα απαιτούσε την φυσική παρουσία των χρηστών και την επιλογή ενός λαχνού-πιστοποιητικού από μία κληρωτίδα. Σε αυτήν την περίπτωση θα έπρεπε είτε να υπήρχε ολομέλεια και, έτσι, διαμοιρασμός της εμπιστοσύνης σε όλα τα μέλη, είτε να υπήρχε μεταβίβαση της εμπιστοσύνης σε μία επιτροπή.

Στον ψηφιακό κόσμο, το παραπάνω ζήτημα αποτελεί μία ιδιαίτερη πρόκληση με ποικίλες προσεγγίσεις σχετικά με την επιλογή των συστημάτων που θα χρησιμοποιηθούν, καθώς και των οντοτήτων στις οποίες θα εκχωρηθεί εμπιστοσύνη.

Στην παρούσα εφαρμογή, η υλοποίηση μηχανισμών για την ανώνυμη διανομή των Ethereum token των κοινοτήτων με τρόπο που να πληροί τα παραπάνω, τέθηκε εκτός του πλαισίου της εργασίας, εξαιτίας της παρέκκλισης από το κεντρικό θέμα και της πολυπλοκότητας της. Όπως είναι σχεδιασμένη αυτήν τη στιγμή, η Concordia δύναται να υποστηρίξει ποικίλες αφηρημένες διαδικασίες οι οποίες να κατοχυρώνουν την εγκυρότητα των εκάστοτε μελών, αλλά όχι την ανωνυμία τους. Εκείνη, όσο η διαδικασία βασίζεται σε κάποια κεντρική οντότητα αυθεντικοποίησης, δε μπορεί να διασφαλιστεί, καιώς θα απαιτεί πάντα την εκχώρηση εμπιστοσύνης από τον τελικό χρήστη στα υπολογιστικά συστήματα της πρώτης. Η εμφάνιση του προβλήματος οφείλεται στο γεγονός ότι η ανωνυμοποίηση των πιστοποιητικών θα πρέπει να λάβει χώρα εντός των των προαναφερθέντων συστημάτων, τα οποία, ως επί το πλείστον, θα είναι συγκεντρωτικής λογικής.

Για παράδειγμα, έστω ότι μία κεντρική αρχή με δικό της σύστημα αυθεντικοποίησης αρχιτεκτονικής πελάτη-εξυπηρετητή αποφασίζει να συμμετάσχει στην πλατφόρμα της Concordia, δημιουργώντας μία κοινότητα και ορίζοντας ένα εξωτερικό smart contract για τα token των μελών της τελευταίας. Ο μηχανισμός διανομής των token θα μπορούσε να ήταν η εγγραφή του χρήστη στο κεντρικό σύστημα, η εξαγωγή ενός πιστοποιητικού αυθεντικότητας από εκεί και, τέλος, η εξαργύρωση του τελευταίου για ένα αντίστοιχο Ethereum token, μέσω του smart contract.

Λύση στο παραπάνω πρόβλημα μπορεί να επέλθει μόνο με τη μετακίνηση της διαδικασίας ανωνυμοποίησης των πιστοποιητικών επί του blockchain. Αυτό απαιτεί την ύπαρξη ενός μηχανισμού στο οικοσύστημα του Ethereum, ο οποίος να παρέχει τη δυνατότητα μεταφοράς token αποκρύπτοντας τις διευθύνσεις προέλευσης και προορισμού. Έτσι, οι χρήστες απλώς θα μετακινούσαν τα token που αρχικά παρέλαβαν σε μία διεύθυνση μη προσδιορίσιμη από τρίτους.

Στο ευρύτερο οικοσύστημα των blockchain υπάρχουν ήδη υλοποιήσεις που προσφέρουν αυτήν την δυνατότητα επί του εγγενούς τους νομίσματος (π.χ. Zcash, Monero), ενώ διάφορες ομάδες εργάζονται ενεργά για την ανάπτυξη τέτοιων μηχανισμών και στο Ethereum. Αν και υπάρχουν διαφοροποιήσεις στις προσεγγίσεις τους, η κύρια τεχνολογία στην οποία βασίζονται είναι αυτή των λεγόμενων "zero knowledge proof", με επικρατέστερα πρωτόκολλα τα zk-SNARK και zk-STARK. Ως μία ήδη λειτουργική λύση τύπου μίκτη συναλλαγών θα μπορούσε να θεωρηθεί ο Tornado\footnote{\url{https://tornado.cash/}}, ο οποίος παρέχει τη δυνατότητα ανώνυμης μεταφοράς ETH ή ERC20 token αξιοποιώντας τα zk-SNARK.\cite{5.4-privacy-on-ethereum} 

%TODO: add
%1. reputation system
%2. voting types
%3. ethereum, ipfs, move to proof of stake, remove of rendezvous server
